\documentclass[xcolor=dvipsnames,t,aspectratio=169]{beamer}

\usecolortheme{rose}
\usecolortheme{dolphin}
\usetheme{Boadilla}

% Encoding packages MUST come first
\usepackage[utf8]{inputenc}
\usepackage[brazil]{babel}
\usepackage[T1]{fontenc}

\usepackage{amsfonts}
\usepackage{amsmath}
\usepackage{amssymb}
% \usepackage{latexsym}
% \usepackage{graphicx}
% \usepackage{listings} 
\usepackage{xcolor}
\usepackage{amsthm}
\usepackage{url}
\usepackage{textpos}
% \usepackage{amssymb}
% \usepackage{caption}
% \usepackage{subcaption}
% \usepackage{multicol}
% \usepackage{mathrsfs}
\usepackage{float}
\setbeamersize{text margin left=1cm,text margin right=1cm,} 
\usepackage{subcaption}
\usepackage{dsfont}
\usepackage{physics}
\usepackage{mathpazo}
% \usepackage{tikz}
\usepackage{lipsum}
\usepackage{minted}
\usepackage[minted, most]{tcolorbox}
\tcbuselibrary{theorems}
\setbeamertemplate{enumerate items}[default]
\usefonttheme{professionalfonts}

\setlength{\parindent}{0pt} % zero indent

\setbeamertemplate{itemize items}[circle]
\setbeamertemplate{itemize subitem}{$\blacktriangleright$}
\definecolor{nes_dark_purple}{HTML}{4E006B}
\definecolor{nes_dark_orange}{HTML}{EE4F09}
\setbeamercolor{normal text}{fg=nes_dark_purple}\usebeamercolor*{normal text}
\setbeamercolor{math text}{fg=nes_dark_purple}\usebeamercolor[fg]{math text}


\setbeamercolor{title}{fg=nes_dark_purple}
\setbeamercolor{frametitle}{fg=nes_dark_purple}
\setbeamercolor{structure}{fg=nes_dark_purple}
\setbeamercolor{author}{fg=nes_dark_orange}
\setbeamercolor{footline}{fg=nes_dark_purple} 

\setbeamertemplate{footline}[frame number]
\setbeamertemplate{navigation symbols}{}

\definecolor{codegreen}{rgb}{0,0.6,0.2}
\definecolor{codegray}{rgb}{0.5,0.5,0.5}
\definecolor{codepurple}{rgb}{0.58,0,0.82}
\definecolor{backcolour}{rgb}{0.95,0.95,0.92}

\lstdefinestyle{mystyle}{
    backgroundcolor=\color{backcolour},   
    commentstyle=\color{codegreen},
    keywordstyle=\color{magenta},
    numberstyle=\tiny\color{codegray},
    stringstyle=\color{codepurple},
    basicstyle=\ttfamily\footnotesize,
    breakatwhitespace=false,  
    breaklines=true,                 
    captionpos=b,                    
    keepspaces=true,                 
    numbers=left,                    
    numbersep=5pt,                  
    showspaces=false,                
    showstringspaces=false,
    showtabs=false,                  
    tabsize=2
}
\input{../../templates/slides/commands}

\usepackage{tikz}
\usetikzlibrary{arrows.meta, shapes, positioning, calc}
\usepackage{algorithmic}
\usepackage{graphicx}
\usepackage{amsmath}
\usepackage{listings}
\usepackage{multicol}

% Configuração para código Python
\lstset{
    language=Python,
    basicstyle=\tiny\ttfamily,
    keywordstyle=\color{blue},
    commentstyle=\color{green!60!black},
    stringstyle=\color{orange},
    showstringspaces=false,
    breaklines=true,
    frame=single,
    numbers=left,
    numberstyle=\tiny\color{gray}
}

\titlegraphic{
    \includegraphics[scale = 0.5]{../../templates/slides/logo}
}

\logo{
\begin{tikzpicture}[overlay,remember picture]
\node[below left = 0.2cm] at (current page.30) {
    \includegraphics[width=0.1\textwidth]{../../templates/slides/logo}};
\end{tikzpicture}
}

\newcommand{\highlight}[1]{{\color{nes_dark_orange} #1}}

\title{Técnicas Avançadas de Redes Neurais Convolucionais} 

\author{
    Eduardo Adame
}

\date{{\color{nes_dark_purple}  \textbf{Redes Neurais}\\[0.5em] 29 de outubro de 2025 }}


\begin{document}

\frame[plain]{\titlepage}
\setcounter{framenumber}{0}

% Slides apenas com os frames - adicionar ao documento principal

\begin{frame}[c]
\frametitle{Introdução}

\textbf{Objetivos da Aula}
\begin{itemize}
    \item \textbf{Data Augmentation}: Técnicas para expandir datasets artificialmente
    \item \textbf{Pipelines de Dados}: Como usar geradores de dados eficientemente
    \item \textbf{API Funcional do Keras}: Arquiteturas complexas e flexíveis
    \item \textbf{Transfer Learning}: Reutilizar conhecimento de modelos pré-treinados
    \item \textbf{Técnicas Avançadas}: Callbacks, regularização e otimização
\end{itemize}

\vspace{0.5cm}

\textbf{Por que estas técnicas são importantes?}
\begin{itemize}
    \item \textbf{Dados limitados}: Nem sempre temos milhões de imagens rotuladas
    \item \textbf{Eficiência}: Treinar do zero é caro e demorado
    \item \textbf{Generalização}: Evitar overfitting e melhorar performance
    \item \textbf{Produção}: Técnicas essenciais para sistemas reais
\end{itemize}

\end{frame}

\begin{frame}[c]
\frametitle{Data Augmentation: O Problema}

\textbf{Desafio Principal: Escassez de Dados}

\begin{itemize}
    \item \textbf{Obter dados rotulados é:}
        \begin{itemize}
            \item Caro (tempo e recursos humanos)
            \item Demorado (processo manual de rotulação)
            \item Difícil (alguns domínios têm poucos exemplos)
        \end{itemize}
    
    \item \textbf{Consequências da falta de dados:}
        \begin{itemize}
            \item \highlight{Overfitting}: Modelo memoriza treino, não generaliza
            \item \highlight{Baixa robustez}: Sensível a pequenas variações
            \item \highlight{Viés}: Não representa diversidade do mundo real
        \end{itemize}
\end{itemize}

\vspace{0.3cm}

\begin{attention}[Pergunta Fundamental]
Como fazer mais com menos dados?
\end{attention}

\vspace{0.3cm}

\textbf{Solução}: Data Augmentation - criar variações sintéticas dos dados existentes

\end{frame}

\begin{frame}[c]
\frametitle{Data Augmentation: O Conceito}

\textbf{Ideia Central}

\begin{itemize}
    \item Se uma imagem de cadeira rotacionada ainda é uma cadeira...
    \item Se uma cadeira espelhada horizontalmente ainda é uma cadeira...
    \item Se uma cadeira com zoom ainda é uma cadeira...
    \item \textbf{Então podemos criar múltiplos exemplos de treino a partir de uma única imagem!}
\end{itemize}

\vspace{0.5cm}

\textbf{Benefícios}
\begin{itemize}
    \item \textbf{Aumenta o dataset efetivo}: De $N$ para $N \times k$ amostras
    \item \textbf{Regularização implícita}: Rede aprende invariâncias úteis
    \item \textbf{Melhor generalização}: Exposição a variações realistas
    \item \textbf{Reduz overfitting}: Menos provável memorizar os dados
\end{itemize}


\begin{attention}[Cuidado!]
Nem todas as transformações preservam o significado semântico!
\end{attention}

\end{frame}

\begin{frame}[c]
\frametitle{Data Augmentation: O Conceito}

\begin{figure}[h]
\centering
\includegraphics[width=.85\textwidth]{gato.png}
\end{figure}

\end{frame}


\begin{frame}[c]
\frametitle{Data Augmentation: Cuidados Necessários}

\textbf{Exemplo: Placas de Trânsito}

\begin{columns}
\begin{column}{0.5\textwidth}
\textbf{$\checkmark$ Transformações Válidas}
\begin{itemize}
    \item Pequenas rotações ($\pm 15°$)
    \item Mudanças de brilho
    \item Pequenos crops
    \item Blur leve
    \item Ruído moderado
\end{itemize}
\end{column}

\begin{column}{0.5\textwidth}
\textbf{$\cross$ Transformações Inválidas}
\begin{itemize}
    \item \highlight{Flip horizontal/vertical}
    \item Rotações grandes
    \item Distorções severas
    \item Inversão de cores
    \item Crops agressivos
\end{itemize}
\end{column}
\end{columns}

\vspace{0.5cm}

\textbf{Por que?} Placa de "Proibido virar à esquerda" $\neq$ "Proibido virar à direita"

\vspace{0.3cm}

\textbf{Princípio Geral}
\begin{itemize}
    \item Aplicar apenas transformações que \textbf{preservam a classe}
    \item Considerar o \textbf{contexto da aplicação}
    \item Testar se as augmentations \textbf{fazem sentido no domínio}
\end{itemize}

\end{frame}

\begin{frame}[c]
\frametitle{Técnicas Comuns de Data Augmentation}

\textbf{1. Transformações Geométricas}
\begin{itemize}
    \item \textbf{Rotação}: Girar a imagem por um ângulo aleatório
    \item \textbf{Translação}: Deslocar horizontal/verticalmente
    \item \textbf{Escala/Zoom}: Ampliar ou reduzir partes da imagem
    \item \textbf{Flip}: Espelhamento horizontal ou vertical
    \item \textbf{Cisalhamento (shear)}: Deformação angular
\end{itemize}

\vspace{0.3cm}

\textbf{2. Transformações de Cor}
\begin{itemize}
    \item \textbf{Brilho}: Ajustar iluminação global
    \item \textbf{Contraste}: Aumentar/diminuir diferença entre pixels
    \item \textbf{Saturação}: Intensidade das cores
    \item \textbf{Hue}: Mudança de tonalidade
    \item \textbf{Normalização}: Ajuste de canais RGB
\end{itemize}

\end{frame}

\begin{frame}[c]
\frametitle{Técnicas Avançadas de Data Augmentation}

\textbf{3. Transformações Baseadas em Recortes}
\begin{itemize}
    \item \textbf{Random Crop}: Recortar região aleatória da imagem
    \item \textbf{Center Crop}: Recortar região central
    \item \textbf{Cutout (2017)}: Mascarar regiões aleatórias com zeros
    \item \textbf{Random Erasing}: Similar ao Cutout, com pixels aleatórios
\end{itemize}

\vspace{0.3cm}

\textbf{4. Transformações Modernas}
\begin{itemize}
    \item \textbf{MixUp (2018)}: Combinar duas imagens linearmente
        \begin{itemize}
            \item $x' = \lambda x_1 + (1-\lambda) x_2$
            \item $y' = \lambda y_1 + (1-\lambda) y_2$
        \end{itemize}
    \item \textbf{CutMix (2019)}: Recortar e colar patches entre imagens
    \item \textbf{AutoAugment (2019)}: Busca automática de políticas de augmentation
    \item \textbf{RandAugment (2020)}: Versão simplificada e mais eficiente
\end{itemize}

\end{frame}

\begin{frame}[c,fragile]
\frametitle{Data Augmentation em Keras 3 (2025)}

\textbf{Atualização Importante: Keras 3}
\begin{itemize}
    \item \textbf{Keras 3} (lançado em 2023, estável em 2024-2025)
    \item API moderna e multi-backend (TensorFlow, JAX, PyTorch)
    \item \texttt{ImageDataGenerator} está \highlight{deprecated}
    \item Nova abordagem recomendada: \texttt{keras.layers} preprocessing
\end{itemize}


\begin{code}[Abordagem Moderna com Keras 3]{python}
import keras
from keras import layers
data_augmentation = keras.Sequential([
    layers.RandomFlip("horizontal"),
    layers.RandomRotation(0.1),  # ±10% = ±36°
    layers.RandomZoom(0.1),
    layers.RandomContrast(0.1),
    layers.RandomBrightness(0.1),
])
\end{code}

\end{frame}

\begin{frame}[c,fragile]
\frametitle{Pipeline Completo com tf.data}

\begin{code}[Pipeline Moderno de Dados]{python}
import tensorflow as tf
train_ds = tf.keras.utils.image_dataset_from_directory(
    'data/train',
    image_size=(224, 224),
    batch_size=32,
    label_mode='categorical'
)
AUTOTUNE = tf.data.AUTOTUNE
train_ds = (train_ds
    .map(lambda x, y: (data_augmentation(x, training=True), y),
         num_parallel_calls=AUTOTUNE)
    .prefetch(AUTOTUNE)
)
model.fit(train_ds, epochs=50, validation_data=val_ds)
\end{code}

\textbf{Vantagens}: Mais eficiente, integrado ao modelo, funciona em GPU

\end{frame}

\begin{frame}[c,fragile]
\frametitle{Comparação: Abordagem Legada vs Moderna}

\textbf{Abordagem Antiga (ImageDataGenerator - deprecated)}

\begin{code}[Método Antigo]{python}
from keras.preprocessing.image import ImageDataGenerator

datagen = ImageDataGenerator(
    rotation_range=20,
    width_shift_range=0.2,
    height_shift_range=0.2,
    horizontal_flip=True
)

# Treinar com fit_generator (também deprecated)
model.fit_generator(
    datagen.flow(x_train, y_train, batch_size=32),
    epochs=50
)
\end{code}

\end{frame}
\begin{frame}[c,fragile]
\frametitle{Comparação: Abordagem Legada vs Moderna}
\textbf{Problemas}:
\begin{itemize}
    \item Executa em CPU (lento)
    \item API inconsistente
    \item Difícil integrar com tf.data
\end{itemize}

\end{frame}

\begin{frame}[c,fragile]
\frametitle{Augmentation como Camada do Modelo}

\textbf{Melhor Prática 2025: Augmentation dentro do Modelo}

\begin{code}[Integração no Modelo]{python}
# Definir augmentation
augmentation = keras.Sequential([
    layers.RandomFlip("horizontal"),
    layers.RandomRotation(0.2),
    layers.RandomZoom(0.2),
], name="data_augmentation")
# Criar modelo com augmentation integrado
inputs = keras.Input(shape=(224, 224, 3))
x = augmentation(inputs)  # Só aplicado durante treino
x = layers.Rescaling(1./255)(x)
# Resto do modelo
x = layers.Conv2D(32, 3, activation='relu')(x)
# ... mais camadas ...
outputs = layers.Dense(10, activation='softmax')(x)
model = keras.Model(inputs, outputs)
\end{code}

\textbf{Vantagens}: GPU, reprodutível, exportável, treino vs inferência automático 

\end{frame}

\begin{frame}[c]
\frametitle{Boas Práticas de Data Augmentation}

\textbf{Diretrizes Gerais}

\begin{itemize}
    \item \textbf{Comece simples}: Flip + pequenas rotações + zoom
    \item \textbf{Aumente gradualmente}: Adicione complexidade se necessário
    \item \textbf{Valide visualmente}: Inspecione exemplos augmentados
    \item \textbf{Considere o domínio}: Nem tudo faz sentido para todos os problemas
    \item \textbf{Monitore overfitting}: Se persiste, aumente augmentation
\end{itemize}

\vspace{0.3cm}

\textbf{Augmentation vs Tamanho do Dataset}

\begin{itemize}
    \item \textbf{Dataset pequeno (<1000)}: Augmentation agressiva é crucial
    \item \textbf{Dataset médio (1K-100K)}: Augmentation moderada ajuda
    \item \textbf{Dataset grande (>100K)}: Augmentation leve já suficiente
    \item \textbf{Dataset enorme (>1M)}: Augmentation menos crítica
\end{itemize}

\vspace{0.3cm}

\textbf{Cuidado}: Augmentation muito agressiva pode prejudicar!

\end{frame}

\begin{frame}[c]
\frametitle{API Funcional do Keras: Motivação}

\textbf{Limitações do Modelo Sequential}

\begin{itemize}
    \item \textbf{Apenas topologias lineares}: Uma camada após a outra
    \item \textbf{Uma entrada, uma saída}: Não suporta múltiplos inputs/outputs
    \item \textbf{Sem compartilhamento de camadas}: Cada camada usada uma vez
    \item \textbf{Sem skip connections}: Impossível fazer ResNet, U-Net, etc.
\end{itemize}

\vspace{0.5cm}

\textbf{Quando Usar API Funcional?}

\begin{itemize}
    \item \textbf{Múltiplas entradas}: Ex: imagem + metadados
    \item \textbf{Múltiplas saídas}: Ex: classificação + regressão simultâneas
    \item \textbf{Grafos complexos}: ResNet, Inception, U-Net
    \item \textbf{Compartilhamento de pesos}: Siamese networks
    \item \textbf{Skip connections}: Qualquer arquitetura moderna
\end{itemize}

\end{frame}

\begin{frame}[c,fragile]
\frametitle{API Funcional: Conceitos Básicos}

\textbf{Filosofia}
\begin{itemize}
    \item Camadas são \textbf{chamáveis} (callable) em tensores
    \item Cada camada retorna um tensor
    \item Construir o grafo conectando camadas
    \item Especificar inputs e outputs explicitamente
\end{itemize}

\vspace{0.3cm}

\begin{code}[Exemplo Simples]{python}
import keras
from keras import layers
# 1. Definir entrada
inputs = keras.Input(shape=(28, 28, 1))
# 2. Construir grafo aplicando camadas
x = layers.Flatten()(inputs)
x = layers.Dense(128, activation='relu')(x)
x = layers.Dropout(0.5)(x)
outputs = layers.Dense(10, activation='softmax')(x)
# 3. Criar modelo especificando inputs e outputs
model = keras.Model(inputs=inputs, outputs=outputs)
\end{code}

\end{frame}

\begin{frame}[c,fragile]
\frametitle{Exemplo: Múltiplas Saídas}

\textbf{Caso de Uso}: Classificar idade e gênero simultaneamente

\begin{code}[Modelo com Duas Saídas]{python}
# Entrada: imagem de rosto
inputs = keras.Input(shape=(224, 224, 3))
# Feature extraction compartilhado
x = layers.Conv2D(32, 3, activation='relu')(inputs)
x = layers.MaxPooling2D()(x)
x = layers.Conv2D(64, 3, activation='relu')(x)
x = layers.MaxPooling2D()(x)
x = layers.Flatten()(x)
x = layers.Dense(256, activation='relu')(x)
# Saída 1: Idade (regressão)
age_output = layers.Dense(1, name='age')(x)
# Saída 2: Gênero (classificação binária)
gender_output = layers.Dense(1, activation='sigmoid', 
                             name='gender')(x)
model = keras.Model(inputs=inputs, 
                   outputs=[age_output, gender_output])
\end{code}

\end{frame}

\begin{frame}[c,fragile]
\frametitle{Compilando Modelos com Múltiplas Saídas}

\begin{code}[Especificar Losses e Métricas por Saída]{python}
model.compile(
    optimizer='adam',
    loss={
        'age': 'mse',           # Mean Squared Error para idade
        'gender': 'binary_crossentropy'  # BCE para gênero
    },
    loss_weights={
        'age': 0.5,             # Peso relativo de cada loss
        'gender': 0.5
    },
    metrics={
        'age': ['mae'],         # Mean Absolute Error
        'gender': ['accuracy']  # Acurácia
    }
)


\end{code}

\end{frame}
\begin{frame}[c,fragile]
\frametitle{Compilando Modelos com Múltiplas Saídas}

\begin{code}[Especificar Losses e Métricas por Saída]{python}
# Treinar
history = model.fit(
    x_train,
    {'age': y_age_train, 'gender': y_gender_train},
    validation_data=(x_val, {'age': y_age_val, 
                             'gender': y_gender_val}),
    epochs=50
)
\end{code}

\end{frame}

\begin{frame}[c,fragile]
\frametitle{Exemplo: Múltiplas Entradas}

\textbf{Caso de Uso}: Classificação com imagem + metadados

\begin{code}[Modelo com Duas Entradas]{python}
# Entrada 1: Imagem
image_input = keras.Input(shape=(224, 224, 3), name='image')
x1 = layers.Conv2D(32, 3, activation='relu')(image_input)
x1 = layers.Flatten()(x1)
# Entrada 2: Metadados (ex: características numéricas)
metadata_input = keras.Input(shape=(10,), name='metadata')
x2 = layers.Dense(32, activation='relu')(metadata_input)
# Concatenar features de ambas entradas
concatenated = layers.Concatenate()([x1, x2])
# Camadas de classificação final
x = layers.Dense(128, activation='relu')(concatenated)
outputs = layers.Dense(5, activation='softmax')(x)

model = keras.Model(
    inputs=[image_input, metadata_input],
    outputs=outputs
)
\end{code}

\end{frame}

\begin{frame}[c,fragile]
\frametitle{Implementando ResNet com API Funcional}

\begin{code}[Bloco Residual]{python}
def residual_block(x, filters, stride=1):
    # Salvar identidade para skip connection
    shortcut = x
    # Caminho principal
    x = layers.Conv2D(filters, 3, strides=stride, padding='same')(x)
    x = layers.BatchNormalization()(x)
    x = layers.ReLU()(x)
    
    x = layers.Conv2D(filters, 3, padding='same')(x)
    x = layers.BatchNormalization()(x)
    # Ajustar shortcut se necessário
    if stride != 1:
        shortcut = layers.Conv2D(filters, 1, strides=stride)(shortcut)
        shortcut = layers.BatchNormalization()(shortcut)
    # Adicionar skip connection
    x = layers.Add()([x, shortcut])
    x = layers.ReLU()(x)
    return x
\end{code}

\end{frame}

\begin{frame}[c,fragile]
\frametitle{Construindo ResNet Completo}

\begin{code}[Mini-ResNet]{python}
def build_mini_resnet(input_shape, num_classes):
    inputs = keras.Input(shape=input_shape)
    # Stem
    x = layers.Conv2D(64, 7, strides=2, padding='same')(inputs)
    x = layers.BatchNormalization()(x)
    x = layers.ReLU()(x)
    x = layers.MaxPooling2D(3, strides=2, padding='same')(x)
    # Blocos residuais
    x = residual_block(x, 64)
    x = residual_block(x, 64)
    x = residual_block(x, 128, stride=2)
    x = residual_block(x, 128)
    x = residual_block(x, 256, stride=2)
    x = residual_block(x, 256)
    # Cabeça de classificação
    x = layers.GlobalAveragePooling2D()(x)
    outputs = layers.Dense(num_classes, activation='softmax')(x)
    return keras.Model(inputs, outputs, name='mini_resnet')
\end{code}

\end{frame}

\begin{frame}[c]
\frametitle{Transfer Learning: Conceito}

\textbf{Ideia Central}

\begin{itemize}
    \item Treinar CNN do zero requer:
        \begin{itemize}
            \item Milhões de imagens rotuladas
            \item Muito poder computacional
            \item Dias/semanas de treinamento
        \end{itemize}
    \item \textbf{Solução}: Reutilizar conhecimento de modelos pré-treinados!
\end{itemize}

\vspace{0.3cm}

\textbf{Por que funciona?}

\begin{itemize}
    \item \textbf{Features iniciais são genéricas}:
        \begin{itemize}
            \item Bordas, texturas, cores
            \item Padrões básicos presentes em todas as imagens
        \end{itemize}
    \item \textbf{Features finais são específicas}:
        \begin{itemize}
            \item Objetos complexos da tarefa original
            \item Podem ser adaptadas para nova tarefa
        \end{itemize}
\end{itemize}

\end{frame}

\begin{frame}[c]
\frametitle{Transfer Learning: O Conceito}

\begin{figure}[h]
\centering
\includegraphics[width=.8\textwidth]{transfer.png}
\end{figure}

\end{frame}


\begin{frame}[c]
\frametitle{Transfer Learning: Estratégias}

\textbf{1. Feature Extractor (Congelar Base)}

\begin{itemize}
    \item \textbf{Quando usar}: Dataset pequeno, domínio similar
    \item \textbf{Método}: Congelar camadas convolucionais, treinar apenas topo
    \item \textbf{Vantagem}: Rápido, menos risco de overfitting
\end{itemize}

\vspace{0.3cm}

\textbf{2. Fine-tuning (Ajustar Parcialmente)}

\begin{itemize}
    \item \textbf{Quando usar}: Dataset médio/grande
    \item \textbf{Método}: Congelar camadas iniciais, treinar camadas finais
    \item \textbf{Vantagem}: Adapta features para tarefa específica
\end{itemize}

\vspace{0.3cm}

\textbf{3. Treino Completo com Pesos Iniciais}

\begin{itemize}
    \item \textbf{Quando usar}: Dataset grande, domínio diferente
    \item \textbf{Método}: Usar pesos pré-treinados como inicialização
    \item \textbf{Vantagem}: Convergência mais rápida
\end{itemize}

\end{frame}

\begin{frame}[c,fragile]
\frametitle{Transfer Learning: Implementação}

\begin{code}[Feature Extraction]{python}
# Carregar modelo pré-treinado
base_model = keras.applications.ResNet50(
    weights='imagenet',        # Pesos do ImageNet
    include_top=False,         # Remover classificador
    input_shape=(224, 224, 3)
)
# Congelar base
base_model.trainable = False
# Adicionar novo classificador
inputs = keras.Input(shape=(224, 224, 3))
x = base_model(inputs, training=False)
x = layers.GlobalAveragePooling2D()(x)
x = layers.Dense(256, activation='relu')(x)
x = layers.Dropout(0.5)(x)
outputs = layers.Dense(10, activation='softmax')(x)
model = keras.Model(inputs, outputs)
\end{code}

\end{frame}

\begin{frame}[c,fragile]
\frametitle{Transfer Learning: Fine-tuning}

\begin{code}[Fine-tuning em Duas Etapas]{python}
# Etapa 1: Treinar apenas o topo
model.compile(optimizer='adam', loss='categorical_crossentropy',
              metrics=['accuracy'])
model.fit(train_ds, epochs=10, validation_data=val_ds)
# Etapa 2: Descongelar algumas camadas finais
base_model.trainable = True
# Congelar todas exceto últimas 20 camadas
for layer in base_model.layers[:-20]:
    layer.trainable = False
# Recompilar com learning rate menor
model.compile(
    optimizer=keras.optimizers.Adam(1e-5),  # LR baixo!
    loss='categorical_crossentropy',
    metrics=['accuracy']
)
# Continuar treinamento
history = model.fit(train_ds, epochs=20, validation_data=val_ds)
\end{code}
\end{frame}

\begin{frame}[c]
\frametitle{Modelos Pré-treinados Disponíveis (2025)}

\textbf{Opções em keras.applications}

\begin{table}[h]
\centering
\footnotesize
\begin{tabular}{|l|c|c|l|}
\hline
\textbf{Modelo} & \textbf{Parâmetros} & \textbf{Top-1 Acc} & \textbf{Melhor Para} \\
\hline
MobileNetV3 & 5M & 75.6\% & Mobile, edge devices \\
\hline
EfficientNetB0 & 5M & 77.1\% & Balance geral \\
\hline
ResNet50 & 26M & 80.4\% & Baseline confiável \\
\hline
EfficientNetV2-M & 54M & 85.1\% & Alta performance \\
\hline
ConvNeXt-Large & 198M & 86.6\% & Estado da arte \\
\hline
\end{tabular}
\end{table}

\vspace{0.3cm}

\textbf{Considerações para Escolha}

\begin{itemize}
    \item \textbf{Recursos limitados}: MobileNet, EfficientNetB0
    \item \textbf{Produção padrão}: ResNet50, EfficientNetB3
    \item \textbf{Máxima acurácia}: ConvNeXt, EfficientNetV2
    \item \textbf{Domínio específico}: Buscar modelos especializados
\end{itemize}

\end{frame}

\begin{frame}[c,fragile]
\frametitle{Callbacks: Controle Avançado do Treinamento}

\textbf{O que são Callbacks?}

\begin{itemize}
    \item Funções chamadas em pontos específicos do treinamento
    \item Permitem monitorar e controlar o processo
    \item Essenciais para treino em produção
\end{itemize}

\end{frame}
\begin{frame}[c,fragile]
\frametitle{Callbacks: Controle Avançado do Treinamento}

\begin{code}[Callbacks Essenciais]{python}
from keras.callbacks import (
    ModelCheckpoint, EarlyStopping, ReduceLROnPlateau,
    TensorBoard, CSVLogger
)
callbacks = [
    # Salvar melhor modelo
    ModelCheckpoint('best_model.keras', 
                   monitor='val_accuracy',
                   save_best_only=True),
    # Parar se não melhorar
    EarlyStopping(monitor='val_loss', patience=10,
                 restore_best_weights=True),
    # Reduzir LR se estagnar
    ReduceLROnPlateau(monitor='val_loss', factor=0.5,
                     patience=5, min_lr=1e-7)
]
\end{code}

\end{frame}

\begin{frame}[c,fragile]
\frametitle{Callbacks: TensorBoard}

\textbf{Visualização Interativa do Treinamento}

\begin{code}[Configurar TensorBoard]{python}
import datetime
# Criar diretório com timestamp
log_dir = f"logs/fit/{datetime.datetime.now():%Y%m%d-%H%M%S}"
tensorboard_callback = TensorBoard(
    log_dir=log_dir,
    histogram_freq=1,        # Histogramas de pesos
    write_graph=True,        # Visualizar grafo
    write_images=True,       # Visualizar augmentations
    update_freq='epoch',     # Frequência de atualização
    profile_batch='10,20'    # Profiling de performance
)
# Treinar
model.fit(train_ds, epochs=50,
         validation_data=val_ds,
         callbacks=[tensorboard_callback])
# Visualizar: tensorboard --logdir=logs/fit
\end{code}
\end{frame}

\begin{frame}[c,fragile]
\frametitle{Callbacks Customizados}

\begin{code}[Criar Callback Personalizado]{python}
class CustomCallback(keras.callbacks.Callback):
    def on_epoch_end(self, epoch, logs=None):
        # Executado ao fim de cada época
        if logs['val_accuracy'] > 0.95:
            print(f"\nAcurácia alvo atingida em {epoch}!")
            self.model.stop_training = True
    
    def on_train_begin(self, logs=None):
        print("Iniciando treinamento...")

    def on_train_end(self, logs=None):
        print("Treinamento finalizado!")
# Uso
custom_cb = CustomCallback()
model.fit(train_ds, epochs=100, 
         callbacks=[custom_cb])
\end{code}

\textbf{Casos de Uso}: Logging customizado, salvamento periódico, visualizações

\end{frame}

\begin{frame}[c]
\frametitle{Regularização: Dropout e Batch Normalization}

\textbf{Dropout}

\begin{itemize}
    \item \textbf{Conceito}: Desativar neurônios aleatoriamente durante treino
    \item \textbf{Taxa típica}: 0.2 a 0.5
    \item \textbf{Onde usar}: Antes de camadas Dense, após GlobalPooling
    \item \textbf{Efeito}: Força rede a aprender features redundantes
\end{itemize}

\vspace{0.3cm}

\textbf{Batch Normalization}

\begin{itemize}
    \item \textbf{Conceito}: Normalizar ativações entre mini-batches
    \item \textbf{Onde usar}: Após Conv2D, antes ou depois da ativação
    \item \textbf{Benefícios}:
        \begin{itemize}
            \item Acelera convergência
            \item Permite learning rates maiores
            \item Regularização implícita
            \item Reduz sensibilidade à inicialização
        \end{itemize}
\end{itemize}

\end{frame}

\begin{frame}[c,fragile]
\frametitle{Boas Práticas de Regularização}

\begin{code}[CNN Moderna com Regularização]{python}
def build_regularized_model():
    inputs = keras.Input(shape=(224, 224, 3))
    
    # Stem com BatchNorm
    x = layers.Conv2D(64, 7, strides=2, padding='same')(inputs)
    x = layers.BatchNormalization()(x)
    x = layers.ReLU()(x)
    
    # Blocos com BN e Dropout
    for filters in [64, 128, 256]:
        x = layers.Conv2D(filters, 3, padding='same')(x)
        x = layers.BatchNormalization()(x)
        x = layers.ReLU()(x)
        x = layers.MaxPooling2D()(x)
        x = layers.Dropout(0.25)(x)  # Spatial dropout
\end{code}

\end{frame}       
\begin{frame}[c,fragile]
\frametitle{Boas Práticas de Regularização}

\begin{code}[CNN Moderna com Regularização]{python} 
    # Classificador com Dropout
    x = layers.GlobalAveragePooling2D()(x)
    x = layers.Dense(512)(x)
    x = layers.BatchNormalization()(x)
    x = layers.ReLU()(x)
    x = layers.Dropout(0.5)(x)
    outputs = layers.Dense(10, activation='softmax')(x)
    
    return keras.Model(inputs, outputs)
\end{code}

\end{frame}

\begin{frame}[c]
\frametitle{Otimizadores Modernos}

\textbf{Evolução dos Otimizadores}

\begin{itemize}
    \item \textbf{SGD}: Baseline, requer tuning cuidadoso
    \item \textbf{Momentum}: SGD + memória de gradientes
    \item \textbf{Adam (2014)}: Adaptive learning rate, popular
    \item \textbf{AdamW (2017)}: Adam com weight decay correto
    \item \textbf{Lion (2023)}: Eficiente, menos memória
\end{itemize}

\vspace{0.3cm}

\textbf{Recomendações 2025}

\begin{itemize}
    \item \textbf{Primeiro experimento}: Adam com LR=1e-3
    \item \textbf{Transfer learning}: Adam com LR=1e-5 (fine-tuning)
    \item \textbf{Treino longo}: AdamW com cosine decay
    \item \textbf{Recursos limitados}: Lion (menor memória)
    \item \textbf{Máxima performance}: SGD com momentum + scheduler
\end{itemize}

\end{frame}

\begin{frame}[c,fragile]
\frametitle{Learning Rate Scheduling}

\begin{code}[Schedulers de Learning Rate]{python}
from keras.optimizers.schedules import (
    ExponentialDecay, CosineDecay, PolynomialDecay
)
# 1. Decaimento Exponencial
lr_schedule = ExponentialDecay(
    initial_learning_rate=1e-3,
    decay_steps=1000,
    decay_rate=0.96
)
# 2. Cosine Decay (popular em 2025)
lr_schedule = CosineDecay(
    initial_learning_rate=1e-3,
    decay_steps=10000,
    alpha=0.01  # LR mínimo
)
\end{code}

\end{frame}
\begin{frame}[c,fragile]
\frametitle{Learning Rate Scheduling}

\begin{code}[Schedulers de Learning Rate]{python}
# 3. Warm-up + Decay
def create_warmup_schedule(warmup_steps, total_steps):
    def schedule(step):
        if step < warmup_steps:
            return step / warmup_steps
        else:
            return (total_steps - step) / (total_steps - warmup_steps)
    return schedule

optimizer = keras.optimizers.AdamW(learning_rate=lr_schedule)
\end{code}

\end{frame}

\begin{frame}[c,fragile]
\frametitle{Mixed Precision Training (2025)}

\textbf{O que é?}

\begin{itemize}
    \item Usar float16 para aceleração, float32 para estabilidade
    \item \textbf{Benefícios}: 2-3x mais rápido, usa menos memória
    \item \textbf{Requisito}: GPU com Tensor Cores (Volta+)
\end{itemize}
\end{frame}

\begin{frame}[c,fragile]
\frametitle{Mixed Precision Training (2025)}

\begin{code}[Habilitar Mixed Precision]{python}
# Configurar política de mixed precision
keras.mixed_precision.set_global_policy('mixed_float16')
# Modelo normal - Keras cuida do resto automaticamente
model = build_model()
# Última camada deve ser float32 para estabilidade
outputs = layers.Dense(10, activation='softmax', 
                      dtype='float32', name='predictions')(x)
# Compilar normalmente
model.compile(
    optimizer='adam',
    loss='categorical_crossentropy',
    metrics=['accuracy']
)
# Treinar - 2-3x mais rápido!
model.fit(train_ds, epochs=50)
\end{code}

\end{frame}

\begin{frame}[c]
\frametitle{Técnicas Avançadas de 2025}

\textbf{1. Gradient Accumulation}

\begin{itemize}
    \item Simular batches maiores em hardware limitado
    \item Acumular gradientes por N steps antes de atualizar
    \item Útil para modelos muito grandes
\end{itemize}

\vspace{0.3cm}

\textbf{2. Stochastic Weight Averaging (SWA)}

\begin{itemize}
    \item Média de checkpoints durante final do treino
    \item Melhor generalização com custo mínimo
    \item Implementado como callback em Keras 3
\end{itemize}

\vspace{0.3cm}

\textbf{3. Knowledge Distillation}

\begin{itemize}
    \item Treinar modelo pequeno (student) com modelo grande (teacher)
    \item Transferir conhecimento via soft targets
    \item Conseguir 90\%+ da performance com modelo 10x menor
\end{itemize}

\end{frame}

\begin{frame}[c]
\frametitle{Knowledge Distillation}

\begin{figure}[h]
\centering
\includegraphics[width=.8\textwidth]{destill.jpg}
\end{figure}

\end{frame}

\begin{frame}[c,fragile]
\frametitle{Monitoramento e Debugging}

\textbf{Ferramentas Essenciais}

\begin{code}[Debugging e Monitoramento]{python}
# 1. Verificar dataset
for images, labels in train_ds.take(1):
    print(f"Batch shape: {images.shape}")
    print(f"Label shape: {labels.shape}")
    print(f"Min/Max: {images.numpy().min():.2f}, {images.numpy().max():.2f}")
# 2. Visualizar modelo
model.summary()
keras.utils.plot_model(model, show_shapes=True, to_file='model.png')
# 3. Gradient checking
with tf.GradientTape() as tape:
    predictions = model(images, training=True)
    loss = loss_fn(labels, predictions)
gradients = tape.gradient(loss, model.trainable_weights)
print(f"Número de gradientes: {len(gradients)}")
print(f"Gradiente médio: {tf.reduce_mean([tf.reduce_mean(tf.abs(g))
        for g in gradients]):.6f}")
\end{code}

\end{frame}

\begin{frame}[c]
\frametitle{Checklist: Treinamento Efetivo}

\textbf{Antes de Treinar}

\begin{itemize}
    \item[$\square$] Dataset balanceado e representativo
    \item[$\square$] Normalização/padronização aplicada
    \item[$\square$] Data augmentation apropriada ao domínio
    \item[$\square$] Train/val/test splits bem definidos
    \item[$\square$] Baseline estabelecida
\end{itemize}

\vspace{0.3cm}

\textbf{Durante Treinamento}

\begin{itemize}
    \item[$\square$] Monitorar overfitting (train vs val loss)
    \item[$\square$] Verificar convergência do learning rate
    \item[$\square$] Usar callbacks (early stopping, checkpoints)
    \item[$\square$] Visualizar métricas no TensorBoard
    \item[$\square$] Salvar checkpoints regularmente
\end{itemize}

\end{frame}

\begin{frame}[c]
\frametitle{Checklist: Treinamento Efetivo (cont.)}

\textbf{Debugging}

\begin{itemize}
    \item[$\square$] Verificar se modelo consegue overfit em batch pequeno
    \item[$\square$] Inspecionar predições em exemplos individuais
    \item[$\square$] Visualizar ativações e feature maps
    \item[$\square$] Checar distribuição de gradientes
    \item[$\square$] Validar pipeline de dados
\end{itemize}

\vspace{0.3cm}

\textbf{Otimização}

\begin{itemize}
    \item[$\square$] Tunar hyperparâmetros sistematicamente
    \item[$\square$] Experimentar diferentes arquiteturas
    \item[$\square$] Testar diferentes augmentations
    \item[$\square$] Considerar ensemble de modelos
    \item[$\square$] Avaliar no conjunto de teste apenas no final
\end{itemize}

\end{frame}

\begin{frame}[c]
\frametitle{Exemplo Completo: Pipeline de Produção}

\textbf{Componentes de um Pipeline Robusto}

\begin{enumerate}
    \item \textbf{Preparação de Dados}
        \begin{itemize}
            \item tf.data.Dataset com prefetching e caching
            \item Augmentation integrada ao modelo
            \item Validação cruzada estratificada
        \end{itemize}
    
    \item \textbf{Modelo}
        \begin{itemize}
            \item Transfer learning com fine-tuning
            \item Regularização (Dropout, BatchNorm, L2)
            \item Mixed precision para performance
        \end{itemize}
    
    \item \textbf{Treinamento}
        \begin{itemize}
            \item Callbacks completos (checkpoint, early stopping, LR reduction)
            \item TensorBoard para monitoramento
            \item Salvamento de métricas e logs
        \end{itemize}
    
    \item \textbf{Validação e Deploy}
        \begin{itemize}
            \item Avaliação no test set
            \item Análise de erros
            \item Otimização para inferência (TFLite, ONNX)
        \end{itemize}
\end{enumerate}

\end{frame}

\begin{frame}[c,fragile]
\frametitle{Template: Pipeline Completo}

\begin{code}[Pipeline de Produção]{python}
import keras
from keras import layers
import tensorflow as tf
# 1. Preparar dados
train_ds = tf.keras.utils.image_dataset_from_directory('data/train', image_size=(224,224), batch_size=32).cache().prefetch(tf.data.AUTOTUNE)
# 2. Augmentation
augmentation = keras.Sequential([
    layers.RandomFlip(), layers.RandomRotation(0.2),
    layers.RandomZoom(0.2), layers.RandomContrast(0.2)
])

\end{code}

\end{frame}
\begin{frame}[c,fragile]
\frametitle{Template: Pipeline Completo (cont.)}

\begin{code}[Pipeline de Produção - Continuação]{python}
# 3. Transfer learning
base = keras.applications.EfficientNetB3(
    include_top=False, weights='imagenet'
)
base.trainable = False

inputs = keras.Input(shape=(224, 224, 3))
x = augmentation(inputs)
x = layers.Rescaling(1./255)(x)
x = base(x, training=False)
x = layers.GlobalAveragePooling2D()(x)
x = layers.Dense(256, activation='relu')(x)
x = layers.BatchNormalization()(x)
x = layers.Dropout(0.5)(x)
outputs = layers.Dense(10, activation='softmax',
                      dtype='float32')(x)
model = keras.Model(inputs, outputs)
\end{code}

\end{frame}

\begin{frame}[c,fragile]
\frametitle{Template: Pipeline Completo (cont.)}

\begin{code}[Pipeline de Produção - Continuação]{python}
# 4. Compilar
model.compile(
    optimizer=keras.optimizers.Adam(1e-3),
    loss='sparse_categorical_crossentropy',
    metrics=['accuracy']
)
# 5. Callbacks
callbacks = [
    keras.callbacks.ModelCheckpoint('best.keras', 
                                   save_best_only=True),
    keras.callbacks.EarlyStopping(patience=10, 
                                 restore_best_weights=True),
    keras.callbacks.ReduceLROnPlateau(factor=0.5, patience=5),
    keras.callbacks.TensorBoard(log_dir='logs')
]
# 6. Treinar
model.fit(train_ds, epochs=100, validation_data=val_ds,
         callbacks=callbacks)
\end{code}

\end{frame}

\begin{frame}[c]
\frametitle{Recursos e Próximos Passos}

\textbf{Documentação Oficial}

\begin{itemize}
    \item \textbf{Keras 3}: \texttt{keras.io} - API reference completa
    \item \textbf{TensorFlow}: \texttt{tensorflow.org/tutorials}
    \item \textbf{Papers}: arxiv.org/list/cs.CV/recent
\end{itemize}

\vspace{0.3cm}

\textbf{Práticas Recomendadas}

\begin{itemize}
    \item Começar com modelos pré-treinados
    \item Experimentar sistematicamente (controle de versão, MLflow)
    \item Investir tempo em preparação de dados
    \item Monitorar tudo com TensorBoard
    \item Validar no mundo real, não apenas métricas
\end{itemize}

\vspace{0.3cm}

\textbf{Próxima Aula}

\begin{itemize}
    \item \textbf{Tópico}: Texto + Word Vectors
\end{itemize}

\end{frame}

\begin{frame}[c]
\frametitle{Resumo da Aula}

\textbf{Principais Conceitos}

\begin{itemize}
    \item \textbf{Data Augmentation}
        \begin{itemize}
            \item Expandir datasets artificialmente
            \item Usar camadas de preprocessing em Keras 3
            \item Escolher augmentations apropriadas ao domínio
        \end{itemize}
    
    \item \textbf{API Funcional}
        \begin{itemize}
            \item Flexibilidade para grafos complexos
            \item Múltiplas entradas/saídas
            \item Skip connections e arquiteturas modernas
        \end{itemize}
    
    \item \textbf{Transfer Learning}
        \begin{itemize}
            \item Reutilizar conhecimento pré-treinado
            \item Feature extraction vs fine-tuning
            \item Escolher modelo base apropriado
        \end{itemize}
    
    \item \textbf{Boas Práticas}
        \begin{itemize}
            \item Callbacks para controle de treino
            \item Regularização (Dropout, BatchNorm)
            \item Mixed precision para performance
        \end{itemize}
\end{itemize}

\end{frame}

\begin{frame}[c]
\frametitle{Perguntas}

\textbf{Q: Ainda posso usar ImageDataGenerator?}

A: Funciona, mas está deprecated. Use keras.layers para augmentation.

\vspace{0.3cm}

\textbf{Q: Quando usar Sequential vs Functional API?}

A: Sequential para modelos simples lineares. Functional para tudo mais.

\vspace{0.3cm}

\textbf{Q: Quanto data augmentation é demais?}

A: Se val loss $>$ train loss e não melhora, reduza augmentation.

\vspace{0.3cm}

\textbf{Q: Sempre usar transfer learning?}

A: Quase sempre! Treinar do zero só se: dataset imenso OU domínio muito diferente.

\vspace{0.3cm}

\textbf{Q: Mixed precision quebra meu modelo?}

A: Raro. Se acontecer, verifique overflow em loss (use loss scaling automático).

\end{frame}

\begin{frame}[c, noframenumbering, plain]
    \frametitle{~}
    \vfill
    \begin{center}
        {\Huge Obrigado!}\vspace{1.5em}\\
        {\Large \highlight{Dúvidas?}}\\
    \end{center}
    \vfill
    % \begin{center}
    %     {\small Próxima aula: 22/10/2025}\\
    %     {\small Tema: Arquiteturas Modernas de CNNs}
    % \end{center}
\end{frame}

\end{document}