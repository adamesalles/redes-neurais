\documentclass[a4paper,12pt]{article}

% Pacotes essenciais
\usepackage[portuguese]{babel}
\usepackage[utf8]{inputenc}
\usepackage[T1]{fontenc}
\usepackage[margin=2.5cm]{geometry}
\usepackage{amsmath}
\usepackage{amsfonts}
\usepackage{amssymb}
\usepackage{graphicx}
\usepackage{booktabs}
\usepackage{array}
\usepackage{tikz}
\usepackage{listings}
\usepackage{xcolor}
\usepackage{hyperref}
\usepackage{fancyhdr}
\usepackage{enumitem}
\usepackage{float}

% Configuração de cabeçalho e rodapé
\pagestyle{fancy}
\fancyhf{}
\fancyhead[L]{Curso de Redes Neurais - Semana X}
\fancyhead[R]{\today}
\fancyfoot[C]{\thepage}

% Configuração para código Python
\lstdefinestyle{python}{
    language=Python,
    basicstyle=\ttfamily\small,
    keywordstyle=\color{blue}\bfseries,
    commentstyle=\color{gray}\itshape,
    stringstyle=\color{red},
    showstringspaces=false,
    breaklines=true,
    frame=single,
    backgroundcolor=\color{gray!5},
    numbers=left,
    numberstyle=\tiny\color{gray},
    stepnumber=1
}

% Configuração de links
\hypersetup{
    colorlinks=true,
    linkcolor=blue,
    urlcolor=blue,
    citecolor=blue
}

% Definição de ambientes personalizados
\newtheorem{definicao}{Definição}[section]
\newtheorem{teorema}{Teorema}[section]
\newtheorem{exemplo}{Exemplo}[section]

% Título do documento
\title{Notas de Aula - Semana X \\
       \large Título do Tópico}
\author{Nome do Instrutor}
\date{\today}

\begin{document}

\maketitle

\tableofcontents
\newpage

% Seção 1: Introdução
\section{Introdução}

Breve introdução ao tópico da semana, contextualizando com o conteúdo anterior e estabelecendo os objetivos de aprendizagem.

\subsection{Objetivos de Aprendizagem}
Ao final desta aula, o estudante será capaz de:
\begin{itemize}
    \item Objetivo 1
    \item Objetivo 2
    \item Objetivo 3
\end{itemize}

% Seção 2: Fundamentos Teóricos
\section{Fundamentos Teóricos}

\subsection{Conceitos Fundamentais}

\begin{definicao}
Definição importante do tópico.
\end{definicao}

Explicação detalhada dos conceitos fundamentais, incluindo:
\begin{enumerate}
    \item Primeiro conceito
    \item Segundo conceito
    \item Terceiro conceito
\end{enumerate}

\subsection{Formulação Matemática}

Apresentação das equações e fórmulas relevantes:

\begin{equation}
    \text{Equação importante} = \text{formulação}
    \label{eq:importante}
\end{equation}

Onde cada termo representa...

% Seção 3: Implementação Prática
\section{Implementação Prática}

\subsection{Algoritmo Passo a Passo}
\begin{enumerate}
    \item Passo 1: Descrição
    \item Passo 2: Descrição
    \item Passo 3: Descrição
\end{enumerate}

\subsection{Exemplo de Código}
\begin{lstlisting}[style=python, caption=Exemplo de implementação em Python]
import numpy as np
import matplotlib.pyplot as plt

# Código de exemplo
def funcao_exemplo(x):
    """
    Função de exemplo para demonstrar conceitos
    """
    return x**2

# Uso da função
x = np.linspace(-10, 10, 100)
y = funcao_exemplo(x)
plt.plot(x, y)
plt.title('Exemplo de Função')
plt.show()
\end{lstlisting}

% Seção 4: Exercícios Propostos
\section{Exercícios Propostos}

\subsection{Exercício 1}
Descrição do primeiro exercício com instruções claras.

\textbf{Dica:} Sugestão para resolver o exercício.

\subsection{Exercício 2}
Descrição do segundo exercício.

\subsection{Exercício 3}
Exercício mais desafiador que integra os conceitos.

% Seção 5: Leitura Complementar
\section{Leitura Complementar}

\begin{itemize}
    \item Referência 1: Livro ou artigo relevante
    \item Referência 2: Tutorial online
    \item Referência 3: Paper científico (opcional)
\end{itemize}

% Seção 6: Próxima Aula
\section{Próxima Aula}

Breve preview dos tópicos que serão abordados na próxima semana e como eles se conectam com o conteúdo atual.

\end{document}