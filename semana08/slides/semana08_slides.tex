\documentclass[xcolor=dvipsnames,t,aspectratio=169]{beamer}

\usecolortheme{rose}
\usecolortheme{dolphin}
\usetheme{Boadilla}

% Encoding packages MUST come first
\usepackage[utf8]{inputenc}
\usepackage[brazil]{babel}
\usepackage[T1]{fontenc}

\usepackage{amsfonts}
\usepackage{amsmath}
\usepackage{amssymb}
% \usepackage{latexsym}
% \usepackage{graphicx}
% \usepackage{listings} 
\usepackage{xcolor}
\usepackage{amsthm}
\usepackage{url}
\usepackage{textpos}
% \usepackage{amssymb}
% \usepackage{caption}
% \usepackage{subcaption}
% \usepackage{multicol}
% \usepackage{mathrsfs}
\usepackage{float}
\setbeamersize{text margin left=1cm,text margin right=1cm,} 
\usepackage{subcaption}
\usepackage{dsfont}
\usepackage{physics}
\usepackage{mathpazo}
% \usepackage{tikz}
\usepackage{lipsum}
\usepackage{minted}
\usepackage[minted, most]{tcolorbox}
\tcbuselibrary{theorems}
\setbeamertemplate{enumerate items}[default]
\usefonttheme{professionalfonts}

\setlength{\parindent}{0pt} % zero indent

\setbeamertemplate{itemize items}[circle]
\setbeamertemplate{itemize subitem}{$\blacktriangleright$}
\definecolor{nes_dark_purple}{HTML}{4E006B}
\definecolor{nes_dark_orange}{HTML}{EE4F09}
\setbeamercolor{normal text}{fg=nes_dark_purple}\usebeamercolor*{normal text}
\setbeamercolor{math text}{fg=nes_dark_purple}\usebeamercolor[fg]{math text}


\setbeamercolor{title}{fg=nes_dark_purple}
\setbeamercolor{frametitle}{fg=nes_dark_purple}
\setbeamercolor{structure}{fg=nes_dark_purple}
\setbeamercolor{author}{fg=nes_dark_orange}
\setbeamercolor{footline}{fg=nes_dark_purple} 

\setbeamertemplate{footline}[frame number]
\setbeamertemplate{navigation symbols}{}

\definecolor{codegreen}{rgb}{0,0.6,0.2}
\definecolor{codegray}{rgb}{0.5,0.5,0.5}
\definecolor{codepurple}{rgb}{0.58,0,0.82}
\definecolor{backcolour}{rgb}{0.95,0.95,0.92}

\lstdefinestyle{mystyle}{
    backgroundcolor=\color{backcolour},   
    commentstyle=\color{codegreen},
    keywordstyle=\color{magenta},
    numberstyle=\tiny\color{codegray},
    stringstyle=\color{codepurple},
    basicstyle=\ttfamily\footnotesize,
    breakatwhitespace=false,  
    breaklines=true,                 
    captionpos=b,                    
    keepspaces=true,                 
    numbers=left,                    
    numbersep=5pt,                  
    showspaces=false,                
    showstringspaces=false,
    showtabs=false,                  
    tabsize=2
}
\input{../../templates/slides/commands}

\usepackage{tikz}
\usetikzlibrary{arrows.meta, shapes, positioning, calc}
\usepackage{algorithmic}
\usepackage{graphicx}
\usepackage{amsmath}
\usepackage{listings}
\usepackage{multicol}

% Configuração para código Python
\lstset{
    language=Python,
    basicstyle=\tiny\ttfamily,
    keywordstyle=\color{blue},
    commentstyle=\color{green!60!black},
    stringstyle=\color{orange},
    showstringspaces=false,
    breaklines=true,
    frame=single,
    numbers=left,
    numberstyle=\tiny\color{gray}
}

\titlegraphic{
    \includegraphics[scale = 0.5]{../../templates/slides/logo}
}

\logo{
\begin{tikzpicture}[overlay,remember picture]
\node[below left = 0.2cm] at (current page.30) {
    \includegraphics[width=0.1\textwidth]{../../templates/slides/logo}};
\end{tikzpicture}
}

\newcommand{\highlight}[1]{{\color{nes_dark_orange} #1}}

\title{Arquiteturas de Redes Neurais Convolucionais} 

\author{
    Eduardo Adame
}

\date{{\color{nes_dark_purple}  \textbf{Redes Neurais}\\[0.5em] 22 de outubro de 2025 }}


\begin{document}

\frame[plain]{\titlepage}
\setcounter{framenumber}{0}

\begin{frame}[c]
\begin{itemize}
    \item \textbf{Objetivo}: Compreender a evolução das arquiteturas CNN
    \item \textbf{Marcos históricos}: AlexNet, VGG, Inception, ResNet
    \item \textbf{Tendências atuais}: Vision Transformers e arquiteturas híbridas
    \item \textbf{Aplicações práticas}: Como escolher a arquitetura adequada
\end{itemize}

\end{frame}

\begin{frame}[c]
\frametitle{AlexNet (2012): O Marco Inicial}

\textbf{Contexto Histórico}
\begin{itemize}
    \item Criada para o ImageNet Large Scale Visual Recognition Challenge (ILSVRC)
    \item \textbf{Tarefa}: Classificar entre 1000 classes
    \item \textbf{Dataset}: Aproximadamente 1,2 milhão de imagens
    \item Considerada o "ponto de ignição" do deep learning moderno
\end{itemize}

\vspace{0.5cm}

\textbf{Resultados Revolucionários}
\begin{itemize}
    \item \highlight{Top-5 error rate: 15,4\%}
    \item Segundo colocado: 26,2\%
    \item \textbf{Diferença de mais de 10\%} - resultado devastador
\end{itemize}

\vspace{0.5cm}

\textbf{Por que foi tão importante?}
\begin{itemize}
    \item Primeira demonstração clara do poder das CNNs profundas
    \item Uso de GPUs para acelerar o treinamento
    \item Popularizou técnicas como ReLU e Dropout
\end{itemize}

\end{frame}

\begin{frame}[c]
\frametitle{Arquitetura da AlexNet}

\begin{figure}[h]
\centering
\includegraphics[width=.7\textwidth]{alexNet-architecture-2135888065.png}
\end{figure}

\textbf{Características Principais}
\begin{itemize}
    \item \textbf{8 camadas}: 5 convolucionais + 3 totalmente conectadas
    \item \textbf{60 milhões de parâmetros}
    \item \textbf{Data augmentation}: Rotação, espelhamento, crops aleatórios
    \item \textbf{Dropout}: Regularização nas camadas FC
    \item \textbf{ReLU}: Primeira aplicação em larga escala
\end{itemize}

\end{frame}

\begin{frame}[c, fragile]
\frametitle{Template Básico da AlexNet}

\begin{code}[Implementação Simplificada]{python}
model = Sequential([
    Conv2D(96, (11,11), strides=4, activation='relu'),
    MaxPooling2D((3,3), strides=2),
    Conv2D(256, (5,5), activation='relu'),
    MaxPooling2D((3,3), strides=2),
    Conv2D(384, (3,3), activation='relu'),
    Conv2D(384, (3,3), activation='relu'),
    Conv2D(256, (3,3), activation='relu'),
    MaxPooling2D((3,3), strides=2),
    Flatten(),
    Dense(4096, activation='relu'),
    Dropout(0.5),
    Dense(4096, activation='relu'),
    Dropout(0.5),
    Dense(1000, activation='softmax')
])
\end{code}

\end{frame}

\begin{frame}[c]
\frametitle{VGG (2014): Simplicidade e Profundidade}

\textbf{Filosofia de Design}
\begin{itemize}
    \item \textbf{Simplificar a estrutura da rede}
    \item \textbf{Evitar escolhas manuais} do tamanho de convolução
    \item \textbf{Redes muito profundas} com convoluções 3$\times$3
    \item Convoluções menores \textbf{"simulam" convoluções maiores}
\end{itemize}

\vspace{0.5cm}

\textbf{Insight Principal: Campo Receptivo}
\begin{center}
\begin{tikzpicture}[scale=0.6]
    % Layer 1
    \draw[step=0.3] (0,0) grid (2.1,2.1);
    \node at (1.05,-0.3) {\footnotesize Camada 1};
    
    % Highlight 3x3 region
    \draw[thick, red] (0.3,0.3) rectangle (1.2,1.2);
    
    % Arrow
    \draw[->, thick] (2.3,1.05) -- (2.7,1.05);
    
    % Layer 2
    \draw[step=0.3] (3,0) grid (5.1,2.1);
    \node at (4.05,-0.3) {\footnotesize Camada 2};
    
    % Highlight single cell
    \fill[blue!30] (3.6,0.9) rectangle (3.9,1.2);
    
    % Arrow
    \draw[->, thick] (5.3,1.05) -- (5.7,1.05);
    
    % Layer 3
    \draw[step=0.3] (6,0) grid (8.1,2.1);
    \node at (7.05,-0.3) {\footnotesize Camada 3};
    
    % Show 5x5 receptive field
    \draw[thick, blue] (6.0,0.6) rectangle (7.5,2.1);
    \node at (6.75,2.4) {\footnotesize Campo 5$\times$5};
\end{tikzpicture}
\end{center}

Duas convoluções 3$\times$3 consecutivas = Uma convolução 5$\times$5

\end{frame}

\begin{frame}[c]
\frametitle{VGG: Vantagem dos Filtros Pequenos}

\textbf{Comparação de Parâmetros}

\begin{columns}
\begin{column}{0.5\textwidth}
\textbf{Uma camada 7$\times$7}
\begin{align}
\text{Parâmetros} &= 7 \times 7 \times C \times C\\
&= 49C^2
\end{align}
\end{column}

\begin{column}{0.5\textwidth}
\textbf{Três camadas 3$\times$3}
\begin{align}
\text{Parâmetros} &= 3 \times (3 \times 3 \times C \times C)\\
&= 27C^2
\end{align}
\end{column}
\end{columns}

\begin{attention}[Redução de Parâmetros]
\begin{center}
$49C^2 \rightarrow 27C^2 \approx \textbf{45\% de redução!}$
\end{center}
\end{attention}

\textbf{Vantagens Adicionais}
\begin{itemize}
    \item \textbf{Mais não-linearidades}: 3 ReLUs vs 1 ReLU
    \item \textbf{Maior expressividade}: Função mais complexa
    \item \textbf{Melhor gradiente}: Propagação mais eficiente
\end{itemize}

\end{frame}

\begin{frame}[c]
\frametitle{Arquitetura VGG16}

\begin{figure}[h]
\centering
\includegraphics[width=0.7\textwidth]{vgg16-1739283886.png}
\end{figure}

\textbf{Padrão de Arquitetura}
\begin{itemize}
    \item \textbf{Duplicação progressiva}: 64 $\to$ 128 $\to$ 256 $\to$ 512 $\to$ 512
    \item \textbf{Redução espacial}: MaxPooling 2$\times$2 após cada bloco
    \item \textbf{16 camadas com pesos}: 13 conv + 3 FC
    \item \textbf{138 milhões de parâmetros}
\end{itemize}

\end{frame}

\begin{frame}[c]
\frametitle{Legado da VGG}

\textbf{Contribuições Principais}
\begin{itemize}
    \item \textbf{Primeira arquitetura} a experimentar com muitas camadas
    \item \textbf{Filosofia}: "Mais profundo é melhor"
    \item \textbf{Padronização}: Uso consistente de filtros 3$\times$3
    \item \textbf{Modelo base}: Serviu como backbone para trabalhos futuros
\end{itemize}

\vspace{0.5cm}

\textbf{Limitações}
\begin{itemize}
    \item \textbf{Muito pesada}: 138M parâmetros, arquivo de 500MB+
    \item \textbf{Lenta}: Muitas operações nas camadas FC
    \item \textbf{Consumo de memória}: Mapas de características grandes
\end{itemize}

\vspace{0.5cm}

\textbf{Influência Atual}
\begin{itemize}
    \item Backbone em detecção de objetos (Faster R-CNN, etc.)
    \item Base para transfer learning
    \item Inspiração para arquiteturas modernas
\end{itemize}

\end{frame}

\begin{frame}[c]
\frametitle{Inception (GoogLeNet, 2014): Pensando Diferente}

\textbf{Motivação}
\begin{itemize}
    \item \textbf{Problema}: Qual tamanho de filtro usar? 1$\times$1, 3$\times$3, 5$\times$5?
    \item \textbf{Insight}: Por que não usar \textit{todos}?
    \item \textbf{Eficiência computacional}: Como manter baixo custo?
    \item \textbf{Principio Hebbiano}: "Neurônios que disparam juntos, se conectam"
\end{itemize}

\vspace{0.5cm}

\textbf{Solução: Módulos Inception}
\begin{itemize}
    \item \textbf{Ramificação paralela}: Diferentes operações simultâneas
    \item \textbf{Concatenação}: Combinar todas as saídas
    \item \textbf{1$\times$1 convoluções}: Redução de dimensionalidade
    \item \textbf{Especialização}: Cada ramo processa uma porção do trabalho
\end{itemize}

\end{frame}

\begin{frame}[c]
\frametitle{Módulo Inception: Versão Ingênua}

\begin{figure}[h]
\centering
\includegraphics[width=0.7\textwidth]{image.png}
\end{figure}

\textbf{Problema}: Muitas operações caras!
\begin{itemize}
    \item Convoluções 3$\times$3 e 5$\times$5 são computacionalmente intensivas
    \item Número de canais cresce rapidamente
    \item Explosão de parâmetros
\end{itemize}

\end{frame}

\begin{frame}[c]
\frametitle{Inception com Gargalos 1$\times$1}

\begin{figure}[h]
\centering
\includegraphics[width=0.7\textwidth]{image2.png}
\end{figure}

\textbf{Vantagens dos Gargalos 1$\times$1}
\begin{itemize}
    \item \textbf{Redução de canais}: Antes das operações caras
    \item \textbf{Menos parâmetros}: Significativa economia computacional
    \item \textbf{Não-linearidade extra}: ReLU após cada 1$\times$1
\end{itemize}

\end{frame}

\begin{frame}[c]
\frametitle{Evolução do Inception}

\begin{multicols}{2}

\textbf{Inception v1 (GoogLeNet, 2014)}
\begin{itemize}
    \item Módulos básicos com gargalos 1$\times$1
    \item 22 camadas, apenas 5M parâmetros
    \item Saídas auxiliares para treinamento
\end{itemize}

\vspace{0.3cm}

\textbf{Inception v2/v3 (2015-2016)}
\begin{itemize}
    \item \textbf{Factorização}: 5$\times$5 $\to$ duas 3$\times$3; 3$\times$3 $\to$ 1$\times$3 + 3$\times$1
    \item \textbf{Batch Normalization}: Acelera convergência
    \item \textbf{Label smoothing}: Regularização no treinamento
\end{itemize}

\vspace{0.3cm}

\textbf{Inception v4 e Inception-ResNet (2016)}
\begin{itemize}
    \item Arquitetura mais uniforme e simples
    \item Combinação com conexões residuais
    \item Estado da arte até 2017
\end{itemize}

\vspace{0.3cm}

\textbf{Xception (2017)}
\begin{itemize}
    \item Separable convolutions: Espacial + Canal
    \item Ainda mais eficiente que Inception v3
\end{itemize}

\end{multicols}
\end{frame}

\begin{frame}[c]
\frametitle{ResNet (2015): Resolvendo o Problema da Profundidade}

\textbf{O Problema Observado}

\begin{columns}
\begin{column}{0.5\textwidth}
\begin{figure}[h]
\centering
\begin{tikzpicture}[scale=0.6]
    \draw[->] (0,0) -- (0,4);
    \draw[->] (0,0) -- (4,0);
    \node at (-0.3,4.5) {\footnotesize Erro};
    \node at (4,-0.5) {\footnotesize Época};
    
    % 20-layer network
    \draw[thick, green] (0.5,3.5) .. controls (1,2.5) and (2,1.5) .. (3.5,1);
    \node at (3.7,0.9) {\footnotesize 20 camadas};
    
    % 56-layer network
    \draw[thick, red] (0.5,3.8) .. controls (1,3) and (2,2.5) .. (3.5,2);
    \node at (3.7,1.9) {\footnotesize 56 camadas};
\end{tikzpicture}
\end{figure}
\end{column}

\begin{column}{0.5\textwidth}
\textbf{Observação Surpreendente}
\begin{itemize}
    \item Redes mais profundas performam \textit{pior}
    \item Tanto em treino quanto em teste
    \item Não é overfitting!
    \item Problema de otimização
\end{itemize}
\end{column}
\end{columns}

\vspace{0.5cm}

\textbf{Diagnóstico}
\begin{itemize}
    \item \textbf{Desvanecimento de gradiente}: Camadas iniciais param de aprender
    \item \textbf{Degradação}: Performance satura e depois decai
    \item \textbf{Dificuldade de otimização}: Funções de identidade são difíceis de aprender
\end{itemize}

\end{frame}

\begin{frame}[c]
\frametitle{A Solução ResNet: Conexões Residuais}

\begin{columns}
\begin{column}{0.5\textwidth}
\textbf{Bloco Tradicional}
\begin{figure}[h]
\centering
\begin{tikzpicture}[scale=0.8]
    \draw[fill=blue!20] (0,0) rectangle (2,0.5);
    \node at (1,0.25) {\footnotesize $x$};
    
    \draw[fill=orange!30] (0,1) rectangle (2,1.5);
    \node at (1,1.25) {\footnotesize Weight Lyr};
    
    \draw[fill=orange!30] (0,2) rectangle (2,2.5);
    \node at (1,2.25) {\footnotesize Weight Lyr};
    
    \draw[fill=green!20] (0,3) rectangle (2,3.5);
    \node at (1,3.25) {\footnotesize $\mathcal{H}(x)$};
    
    \draw[->, thick] (1,0.5) -- (1,1);
    \draw[->, thick] (1,1.5) -- (1,2);
    \draw[->, thick] (1,2.5) -- (1,3);
\end{tikzpicture}
\end{figure}

Aprende: $\mathcal{H}(x)$
\end{column}

\begin{column}{0.5\textwidth}
\textbf{Bloco Residual}
\begin{figure}[h]
\centering
\begin{tikzpicture}[scale=0.8]
    \draw[fill=blue!20] (0,0) rectangle (2,0.5);
    \node at (1,0.25) {\footnotesize $x$};
    
    \draw[fill=orange!30] (0,1) rectangle (2,1.5);
    \node at (1,1.25) {\footnotesize Weight Lyr};
    
    \draw[fill=orange!30] (0,2) rectangle (2,2.5);
    \node at (1,2.25) {\footnotesize Weight Lyr};
    
    % Addition
    \draw[circle, fill=white] (1,3.25) circle (0.2);
    \node at (1,3.25) {\footnotesize +};
    
    \draw[fill=green!20] (0,3.8) rectangle (2,4.3);
    \node at (1,4.05) {\footnotesize $\mathcal{H}(x)$};
    
    \draw[->, thick] (1,0.5) -- (1,1);
    \draw[->, thick] (1,1.5) -- (1,2);
    \draw[->, thick] (1,2.5) -- (1,3.05);
    \draw[->, thick] (1,3.45) -- (1,3.8);
    
    % Skip connection
    \draw[->, thick] (0.2,0.25) -- (-0.5,0.25) -- (-0.5,3.25) -- (0.8,3.25);
\end{tikzpicture}
\end{figure}

Aprende: $\mathcal{F}(x) = \mathcal{H}(x) - x$
\end{column}
\end{columns}

\vspace{0.5cm}

\textbf{Insight Chave}
\begin{itemize}
    \item \textbf{Função residual}: $\mathcal{F}(x)$ é mais fácil de aprender que $\mathcal{H}(x)$
    \item \textbf{Identidade}: Se a melhor função é identidade, $\mathcal{F}(x) = 0$
    \item \textbf{Gradientes}: Fluxo direto através das skip connections
\end{itemize}

\end{frame}

\begin{frame}[c]
\frametitle{Blocos ResNet em Detalhes}

\begin{figure}[h]
\centering
\includegraphics[width=0.7\textwidth]{image3.png}
\end{figure}

\textbf{Bottleneck Design}
\begin{itemize}
    \item \textbf{1$\times$1 $\to$ 3$\times$3 $\to$ 1$\times$1}: Reduz complexidade computacional
    \item \textbf{Menos parâmetros}: Para redes muito profundas (50+ camadas)
    \item \textbf{Eficiência}: Mantém expressividade com menor custo
\end{itemize}

\end{frame}

\begin{frame}[c]
\frametitle{Família ResNet e Impacto}

\textbf{Variantes Principais}
\begin{itemize}
    \item \textbf{ResNet-18/34}: Blocos básicos, até 34 camadas
    \item \textbf{ResNet-50/101/152}: Blocos bottleneck, até 152 camadas
    \item \textbf{ResNet-200+}: Experimentos com redes extremamente profundas
\end{itemize}

\vspace{0.5cm}

\textbf{Resultados no ImageNet (2015)}
\begin{itemize}
    \item \textbf{ResNet-152}: 3.57\% top-5 error (melhor que humanos!)
    \item \textbf{Primeiro a superar performance humana} em ImageNet
    \item \textbf{Vencedor}: ILSVRC 2015 Classification, Detection, Localization
\end{itemize}

\vspace{0.5cm}

\textbf{Impacto Transformador}
\begin{itemize}
    \item \textbf{Arquitetura padrão}: Backbone em quase todos os modelos
    \item \textbf{Transfer learning}: Base para inúmeras aplicações
    \item \textbf{Conceito residual}: Adoptado em Transformers, GAN, etc.
    \item \textbf{Possibilitou redes profundas}: 1000+ camadas funcionais
\end{itemize}

\end{frame}

\begin{frame}[c]
\frametitle{Evoluções Pós-ResNet (2016-2018)}

\textbf{DenseNet (2017)}
\begin{itemize}
    \item \textbf{Conexões densas}: Cada camada conecta a todas as anteriores
    \item \textbf{Reutilização de features}: Máximo aproveitamento de informação
    \item \textbf{Menos parâmetros}: Mais eficiente que ResNet
\end{itemize}

\vspace{0.3cm}

\textbf{MobileNets (2017-2019)}
\begin{itemize}
    \item \textbf{Depthwise Separable Convolutions}: Factorização espacial/canal
    \item \textbf{Dispositivos móveis}: Otimizado para baixo consumo
    \item \textbf{MobileNetV2}: Inverted residuals + linear bottlenecks
    \item \textbf{MobileNetV3}: Neural Architecture Search + squeeze-excitation
\end{itemize}

\vspace{0.3cm}

\textbf{EfficientNet (2019)}
\begin{itemize}
    \item \textbf{Compound scaling}: Balancea width, depth, resolution
    \item \textbf{Neural Architecture Search}: Busca automatizada de arquiteturas
    \item \textbf{Estado da arte}: Melhor trade-off precisão/eficiência
\end{itemize}

\end{frame}

\begin{frame}[c]
\frametitle{Revolução dos Vision Transformers (2020-2025)}

\textbf{Vision Transformer (ViT) - 2021}
\begin{itemize}
    \item \textbf{Patches como tokens}: Divide imagem em patches 16$\times$16
    \item \textbf{Self-attention}: Mecanismo dos Transformers para visão
    \item \textbf{Sem convoluções}: Primeira arquitetura puramente attention-based
    \item \textbf{Escalabilidade}: Performa melhor com datasets muito grandes
\end{itemize}

\vspace{0.3cm}

\textbf{Arquiteturas Híbridas}
\begin{itemize}
    \item \textbf{ConvNet}: CNN early layers + Transformer later layers
    \item \textbf{Swin Transformer}: Attention hierárquico com janelas
    \item \textbf{CoAtNet}: Combinação otimizada de Conv + Attention
\end{itemize}

\vspace{0.3cm}

\textbf{Estado Atual (2025)}
\begin{itemize}
    \item \textbf{ConvNext}: CNNs modernizadas competem com ViTs
    \item \textbf{Foundation models}: CLIP, DALL-E, modelos multimodais
    \item \textbf{Trend}: Arquiteturas que combinam o melhor dos dois mundos
\end{itemize}

\end{frame}

\begin{frame}[c]
\frametitle{Comparativo de Arquiteturas}

\begin{table}[h]
\centering
\footnotesize
\begin{tabular}{|l|c|c|c|c|}
\hline
\textbf{Arquitetura} & \textbf{Ano} & \textbf{Parâmetros} & \textbf{ImageNet Top-1} & \textbf{Inovação Principal} \\
\hline
AlexNet & 2012 & 60M & 42.9\% & CNNs profundas + GPU \\
\hline
VGG-16 & 2014 & 138M & 28.1\% & Filtros 3$\times$3 uniformes \\
\hline
Inception v3 & 2016 & 24M & 21.2\% & Módulos paralelos \\
\hline
ResNet-50 & 2015 & 26M & 20.7\% & Skip connections \\
\hline
ResNet-152 & 2015 & 60M & 19.4\% & Redes muito profundas \\
\hline
DenseNet-201 & 2017 & 20M & 18.1\% & Conexões densas \\
\hline
EfficientNet-B7 & 2019 & 66M & 15.7\% & Compound scaling \\
\hline
ViT-Large & 2021 & 307M & 16.4\% & Pure attention \\
\hline
ConvNext-XL & 2022 & 350M & 13.9\% & CNNs modernizadas \\
\hline
\end{tabular}
\end{table}

\vspace{0.3cm}

\textbf{Observações}
\begin{itemize}
    \item \textbf{Evolução contínua}: Cada arquitetura trouxe insights únicos
    \item \textbf{Trade-offs}: Precisão vs eficiência vs interpretabilidade
    \item \textbf{Especialização}: Diferentes arquiteturas para diferentes tarefas
\end{itemize}

\end{frame}

\begin{frame}[c]
\frametitle{Como Escolher uma Arquitetura?}

\textbf{Fatores de Decisão}

\begin{columns}
\begin{column}{0.5\textwidth}
\textbf{Recursos Disponíveis}
\begin{itemize}
    \item \textbf{Poder computacional}
    \item \textbf{Memória GPU/RAM}
    \item \textbf{Tempo de treinamento}
    \item \textbf{Latência de inferência}
\end{itemize}

\vspace{0.3cm}

\textbf{Natureza dos Dados}
\begin{itemize}
    \item \textbf{Tamanho do dataset}
    \item \textbf{Resolução das imagens}
    \item \textbf{Número de classes}
    \item \textbf{Complexidade visual}
\end{itemize}
\end{column}

\begin{column}{0.5\textwidth}
\textbf{Requisitos da Aplicação}
\begin{itemize}
    \item \textbf{Precisão necessária}
    \item \textbf{Tempo real vs batch}
    \item \textbf{Dispositivo alvo}
    \item \textbf{Interpretabilidade}
\end{itemize}

\vspace{0.3cm}

\textbf{Estratégia de Desenvolvimento}
\begin{itemize}
    \item \textbf{Transfer learning}
    \item \textbf{Fine-tuning}
    \item \textbf{Treino do zero}
    \item \textbf{Ensemble methods}
\end{itemize}
\end{column}
\end{columns}

\end{frame}

\begin{frame}[c]
\frametitle{Recomendações por Cenário}

\textbf{Projetos Iniciantes / Prototipagem}
\begin{itemize}
    \item \textbf{ResNet-18/34}: Equilibrio simplicidade/performance
    \item \textbf{MobileNetV2}: Para dispositivos com restrições
    \item \textbf{EfficientNet-B0}: Boa baseline moderna
\end{itemize}

\vspace{0.3cm}

\textbf{Aplicações de Produção}
\begin{itemize}
    \item \textbf{ResNet-50}: Padrão da indústria, muitos modelos pré-treinados
    \item \textbf{EfficientNet-B3/B4}: Melhor custo-benefício
    \item \textbf{ConvNext}: Para máxima precisão
\end{itemize}

\vspace{0.3cm}

\textbf{Pesquisa / Estado da Arte}
\begin{itemize}
    \item \textbf{ViT-Large}: Para datasets muito grandes
    \item \textbf{Swin Transformer}: Tarefas hierárquicas (detecção, segmentação)
    \item \textbf{Foundation models}: CLIP para multimodalidade
\end{itemize}

\end{frame}

\begin{frame}[c,fragile]
\frametitle{Implementação Prática: ResNet Moderno}

\begin{code}[ResNet-18 com Melhorias 2025]{python}
class BasicBlock(nn.Module):
    def __init__(self, in_channels, out_channels, stride=1):
        super().__init__()
        self.conv1 = nn.Conv2d(in_channels, out_channels, 3, 
                              stride=stride, padding=1, bias=False)
        self.bn1 = nn.BatchNorm2d(out_channels)
        self.conv2 = nn.Conv2d(out_channels, out_channels, 3, 
                              padding=1, bias=False)
        self.bn2 = nn.BatchNorm2d(out_channels)
        self.skip = nn.Identity()
        if stride != 1 or in_channels != out_channels:
            self.skip = nn.Sequential(
                nn.Conv2d(in_channels, out_channels, 1, 
                         stride=stride, bias=False),
                nn.BatchNorm2d(out_channels)
            )
        self.relu = nn.ReLU(inplace=True)
\end{code}
\end{frame}  
\begin{frame}[c,fragile]
\frametitle{Implementação Prática: ResNet Moderno}

\begin{code}[ResNet-18 com Melhorias 2025]{python}
    def forward(self, x):
        identity = self.skip(x)
        
        out = self.relu(self.bn1(self.conv1(x)))
        out = self.bn2(self.conv2(out))
        
        out += identity
        return self.relu(out)
\end{code}
\end{frame}

\begin{frame}[c]
\frametitle{Tendências Futuras em Arquiteturas CNN}

\textbf{Direções de Pesquisa 2025+}

\begin{itemize}
    \item \textbf{Arquiteturas Neurais Buscadas (NAS)}
        \begin{itemize}
            \item Busca automatizada de arquiteturas otimizadas
            \item Especialização para hardware específico
            \item Consideração de multiple objetivos (precisão, latência, energia)
        \end{itemize}
    
    \item \textbf{Modelos Foundation Multimodais}
        \begin{itemize}
            \item Integração visão + linguagem + audio
            \item Transfer learning entre modalidades
            \item Zero-shot learning aprimorado
        \end{itemize}
    
    \item \textbf{Eficiência Energética}
        \begin{itemize}
            \item Green AI: redução da pegada de carbono
            \item Quantização e pruning automáticos
            \item Edge computing otimizado
        \end{itemize}
    
    \item \textbf{Interpretabilidade}
        \begin{itemize}
            \item Arquiteturas inherentemente interpretáveis
            \item Attention visualization melhorada
            \item Debugging automático de modelos
        \end{itemize}
\end{itemize}

\end{frame}

\begin{frame}[c]
\frametitle{Resumo e Conclusões}

\textbf{Evolução das CNNs: Lições Aprendidas}

\begin{itemize}
    \item \textbf{AlexNet (2012)}: Provou o poder das CNNs profundas
    \item \textbf{VGG (2014)}: Simplicidade e padronização são valiosas
    \item \textbf{Inception (2014)}: Paralelização e eficiência computacional
    \item \textbf{ResNet (2015)}: Skip connections resolvem problemas de otimização
    \item \textbf{Era moderna}: Híbridos CNN+Transformer dominam
\end{itemize}

\vspace{0.3cm}

\textbf{Princípios Fundamentais}
\begin{itemize}
    \item \textbf{Não existe ``melhor'' arquitetura}: Depende do contexto
    \item \textbf{Trade-offs são inevitáveis}: Precisão vs eficiência
    \item \textbf{Transfer learning é essencial}: Aproveite modelos pré-treinados
    \item \textbf{Experimente sistematicamente}: Dados empíricos são cruciais
\end{itemize}

\vspace{0.3cm}

\textbf{Próxima Aula}
\begin{itemize}
    \item \textbf{Tópico}: Técnicas Avançadas de Treinamento
    \item \textbf{Conteúdo}: Otimizadores, regularização, data augmentation
\end{itemize}

\end{frame}

\end{document}