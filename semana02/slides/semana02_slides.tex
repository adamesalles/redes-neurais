\documentclass[xcolor=dvipsnames,t,aspectratio=169]{beamer}

\usecolortheme{rose}
\usecolortheme{dolphin}
\usetheme{Boadilla}

% Encoding packages MUST come first
\usepackage[utf8]{inputenc}
\usepackage[brazil]{babel}
\usepackage[T1]{fontenc}

\usepackage{amsfonts}
\usepackage{amsmath}
\usepackage{amssymb}
% \usepackage{latexsym}
% \usepackage{graphicx}
% \usepackage{listings} 
\usepackage{xcolor}
\usepackage{amsthm}
\usepackage{url}
\usepackage{textpos}
% \usepackage{amssymb}
% \usepackage{caption}
% \usepackage{subcaption}
% \usepackage{multicol}
% \usepackage{mathrsfs}
\usepackage{float}
\setbeamersize{text margin left=1cm,text margin right=1cm,} 
\usepackage{subcaption}
\usepackage{dsfont}
\usepackage{physics}
\usepackage{mathpazo}
% \usepackage{tikz}
\usepackage{lipsum}
\usepackage{minted}
\usepackage[minted, most]{tcolorbox}
\input{../../templates/slides/settings}
\input{../../templates/slides/commands}

\usepackage{tikz}
\usetikzlibrary{arrows.meta, shapes}

\titlegraphic{
    \includegraphics[scale = 0.5]{../../templates/slides/logo}
}

\logo{
\begin{tikzpicture}[overlay,remember picture]
\node[left=1.1cm, below=0.2cm] at (current page.30){
    \includegraphics[width=0.1\textwidth]{../../templates/slides/logo}};
\end{tikzpicture}
}

\newcommand{\highlight}[1]{{\color{nes_dark_orange} #1}}

\title{Introdução às Redes Neurais} 

\author{
    Eduardo Adame
}

\date{{\color{nes_dark_purple}  \textbf{Redes Neurais}\\[0.5em] 20 de agosto de 2025 }}

\begin{document}

\frame[plain]{\titlepage}
\setcounter{framenumber}{0}

\section{Revisão da Aula Anterior}
\begin{frame}[c]{Revisão: Descida de Gradiente}
    \begin{columns}[c]
        \begin{column}{0.5\textwidth}
            \textbf{O que aprendemos:}
            \begin{itemize}
                \item Conceitos fundamentais de ML
                \item Regressão linear
                \item \highlight{Descida de gradiente}
                \item Taxa de aprendizado ($\alpha$)
                \item SGD vs. Batch GD
                \item Critérios de convergência
            \end{itemize}
        \end{column}
        \begin{column}{0.5\textwidth}
            \textbf{Algoritmo básico:}
            \begin{equation}
                \boldsymbol{\beta}^{(t+1)} = \boldsymbol{\beta}^{(t)} - \alpha \nabla J(\boldsymbol{\beta}^{(t)})
            \end{equation}
            
            \vspace{0.5cm}
            \textbf{Hoje veremos:}
            \begin{itemize}
                \item Como isso se conecta com \highlight{redes neurais}
                \item O papel do gradiente no \highlight{treinamento}
            \end{itemize}
        \end{column}
    \end{columns}
\end{frame}

\section{Motivação para Redes Neurais}
\begin{frame}[c]{Motivação para Redes Neurais}
    \begin{columns}[c]
        \begin{column}{0.5\textwidth}
            \textbf{Inspiração Biológica:}
            \begin{itemize}
                \item Usar biologia como inspiração para modelo matemático
                \item Receber sinais de neurônios anteriores
                \item Gerar sinais (ou não) de acordo com as entradas
                \item Passar sinais para próximos neurônios
                \item Ao empilhar muitos neurônios, podemos criar modelos complexos
            \end{itemize}
        \end{column}
        \begin{column}{0.5\textwidth}
            \centering
            \begin{tikzpicture}[scale=0.8]
                % Neurônios de entrada
                \node[circle, draw, fill=cyan!30] (x1) at (0, 2) {$x_1$};
                \node[circle, draw, fill=cyan!30] (x2) at (0, 0) {$x_2$};
                \node[circle, draw, fill=cyan!30] (x3) at (0, -2) {$x_3$};
                
                % Camada oculta
                \node[circle, draw, fill=orange!40] (h1) at (3, 1.5) {$\sigma$};
                \node[circle, draw, fill=orange!40] (h2) at (3, 0) {$\sigma$};
                \node[circle, draw, fill=orange!40] (h3) at (3, -1.5) {$\sigma$};
                
                % Neurônio de saída
                \node[circle, draw, fill=gray!50] (y) at (6, 0) {$y$};
                
                % Conexões
                \foreach \x in {1,2,3} {
                    \foreach \h in {1,2,3} {
                        \draw[-latex, thick, yellow!80!black] (x\x) -- (h\h);
                    }
                }
                
                \foreach \h in {1,2,3} {
                    \draw[-latex, thick, yellow!80!black] (h\h) -- (y);
                }
            \end{tikzpicture}
        \end{column}
    \end{columns}
\end{frame}

\begin{frame}[c]{Estrutura de uma Rede Neural (Perceptron)}
    \begin{center}
        \begin{tikzpicture}[scale=1.2]
            % Camada de entrada
            \node[circle, draw, fill=cyan!30] (x1) at (0, 2) {$x_1$};
            \node[circle, draw, fill=cyan!30] (x2) at (0, 0.5) {$x_2$};
            \node[circle, draw, fill=cyan!30] (x3) at (0, -1) {$x_3$};
            
            % Primeira camada oculta
            \foreach \i in {1,2,3,4} {
                \node[circle, draw, fill=orange!40] (h1\i) at (3, 3-\i) {$\sigma$};
            }
            
            % Segunda camada oculta
            \foreach \i in {1,2,3,4} {
                \node[circle, draw, fill=orange!40] (h2\i) at (6, 3-\i) {$\sigma$};
            }
            
            % Camada de saída
            \node[circle, draw, fill=gray!50] (y1) at (9, 1.5) {$\hat{y}_1$};
            \node[circle, draw, fill=gray!50] (y2) at (9, 0) {$\hat{y}_2$};
            \node[circle, draw, fill=gray!50] (y3) at (9, -1.5) {$\hat{y}_3$};
            
            % Conexões (simplificadas para clareza)
            \foreach \x in {1,2,3} {
                \foreach \h in {1,2,3,4} {
                    \draw[-latex, yellow!60!black] (x\x) -- (h1\h);
                }
            }
            
            \foreach \h in {1,2,3,4} {
                \foreach \hh in {1,2,3,4} {
                    \draw[-latex, yellow!60!black] (h1\h) -- (h2\hh);
                }
            }
            
            \foreach \h in {1,2,3,4} {
                \foreach \y in {1,2,3} {
                    \draw[-latex, yellow!60!black] (h2\h) -- (y\y);
                }
            }
            
            % Labels
            \node[below] at (0, -2) {\textbf{Entrada}};
            \node[below] at (4.5, -2) {\textbf{Camadas Ocultas}};
            \node[below] at (9, -2) {\textbf{Saída}};
        \end{tikzpicture}
    \end{center}
    
    \begin{itemize}
        \item Pode ser vista como um \highlight{motor de computação} complicado
        \item Vamos "treiná-la" usando nossos dados de treino
        \item Então (esperamos) ela dará boas respostas em novos dados
    \end{itemize}
\end{frame}

\section{O Neurônio Artificial}
\begin{frame}[c]{Visualização Básica do Neurônio}
    \begin{center}
        \begin{tikzpicture}[scale=1.5]
            % Neurônio central
            \node[circle, draw, align=center, fill=green!40, minimum size=2.5cm] (neuron) at (0,0) {
                \textbf{FUNÇÃO DE}\\
                \textbf{ATIVAÇÃO}
            };
            
            % Entradas
            \draw[-latex, ultra thick, orange!80!black] (-3, 1.5) -- (-1.3, 0.5);
            \draw[-latex, ultra thick, orange!80!black] (-3, 0) -- (-1.3, 0);
            \draw[-latex, ultra thick, orange!80!black] (-3, -1.5) -- (-1.3, -0.5);
            
            % Saída
            \draw[-latex, ultra thick, orange!80!black] (1.3, 0) -- (3, 0) node[midway, above] {Saída};
            
            % Textos explicativos
            \node[text width=4.5cm, align=left] at (-3, -2.5) {
                \small \textbf{Dados da camada anterior}
            };
            
            \node[text width=4cm, align=left] at (3, -1) {
                \small \textbf{O neurônio produz}\\
                \small \textbf{os dados transformados}
            };
            
            \node[text width=4cm, align=center] at (0, -2.5) {
                \small \textbf{Alguma forma de computação}\\
                \small \textbf{transforma as entradas}
            };
        \end{tikzpicture}
    \end{center}
\end{frame}

\begin{frame}[c]{Neurônio com Pesos e Bias}
    \begin{center}
        \begin{tikzpicture}[scale=1.3]
            % Neurônio central
            \node[circle, draw, align=center, fill=green!40, minimum size=2.5cm] (neuron) at (0,0) {
                \textbf{FUNÇÃO DE}\\
                \textbf{ATIVAÇÃO}
            };
            
            % Entradas com pesos
            \node (x1) at (-3, 1.5) {$x_1$};
            \node (x2) at (-3, 0.5) {$x_2$};
            \node (x3) at (-3, -0.5) {$x_3$};
            \node (b) at (-3, -1.5) {$1$};
            
            \draw[-latex, ultra thick, orange!80!black] (x1) -- (-1.2, 0.5) node[midway, above] {$w_1$};
            \draw[-latex, ultra thick, orange!80!black] (x2) -- (-1.2, 0.2) node[midway, above] {$w_2$};
            \draw[-latex, ultra thick, orange!80!black] (x3) -- (-1.2, -0.2) node[midway, above] {$w_3$};
            \draw[-latex, ultra thick, orange!80!black] (b) -- (-1.2, -0.5) node[midway, above] {$b$};
            
            % Saída
            \draw[-latex, ultra thick, orange!80!black] (1.2, 0) -- (3, 0) node[midway, above] {$f(z)$};
            
            % Equação
            \node[text width=6cm, align=center] at (0, -2.5) {
                $z = x_1w_1 + x_2w_2 + x_3w_3 + b$
            };
        \end{tikzpicture}
    \end{center}
\end{frame}

\begin{frame}[c]{Notação Vetorial}
    \begin{itemize}
        \item \bfseries Cálculo do Neurônio
        \begin{align}
            z &= b + \sum_{i=1}^{m} x_i w_i \\[0.5em]
            z &= b + \mathbf{x}^T \mathbf{w} \\[0.5em]
            a &= f(z)
        \end{align}
    \end{itemize}
    
    \vspace{0.5cm}
    
    \begin{columns}[c]
        \begin{column}{0.5\textwidth}
            \textbf{Nomenclatura:}
            \begin{itemize}
                \item $z$ = "net input" (entrada líquida)
                \item $b$ = "bias term" (termo de viés)
                \item $f$ = função de ativação
                \item $a$ = saída para próxima camada
            \end{itemize}
        \end{column}
        \begin{column}{0.5\textwidth}
            \textbf{Dimensões:}
            \begin{itemize}
                \item $\mathbf{x}$ = vetor de entrada
                \item $\mathbf{w}$ = vetor de pesos
                \item $b$ = escalar (bias)
                \item $a$ = escalar (ativação)
            \end{itemize}
        \end{column}
    \end{columns}
\end{frame}

\section{Relação com Regressão Logística}
\begin{frame}[c]{Relação com Regressão Logística}
    Quando escolhemos a função sigmoid: $f(z) = \frac{1}{1 + e^{-z}}$
    
    \begin{equation}
        z = b + \sum_{i=1}^{m} x_i w_i = x_1w_1 + x_2w_2 + \cdots + x_mw_m + b
    \end{equation}
    
    \vspace{0.5cm}
    
    \begin{display}[Então um neurônio é simplesmente uma "unidade" de regressão logística!]
        \begin{center}
            \begin{tabular}{lcl}
                pesos & $\Leftrightarrow$ & coeficientes \\
                entradas & $\Leftrightarrow$ & variáveis \\
                termo de bias & $\Leftrightarrow$ & termo constante
            \end{tabular}
        \end{center}
    \end{display}
    
    \begin{attention}[Insight Importante]
        Um neurônio com ativação sigmoid é exatamente uma regressão logística!
    \end{attention}
\end{frame}

\begin{frame}[c]{A Função Sigmoid}
    \begin{columns}[c]
        \begin{column}{0.5\textwidth}
            \begin{equation}
                \sigma(z) = \frac{1}{1 + e^{-z}}
            \end{equation}
            
            \vspace{0.5cm}
            
            \textbf{Propriedades:}
            \begin{itemize}
                \item Domínio: $(-\infty, +\infty)$
                \item Imagem: $(0, 1)$
                \item Sempre crescente
                \item Ponto de inflexão em $z=0$
                \item $\sigma(0) = 0.5$
            \end{itemize}
        \end{column}
        \begin{column}{0.5\textwidth}
            \begin{tikzpicture}[scale=0.8]
                \draw[-latex] (-3, 0) -- (3, 0) node[right] {$z$};
                \draw[-latex] (0, -0.5) -- (0, 2.5) node[above] {$\sigma(z)$};
                
                % Grid
                \draw[gray, very thin] (-3, 0) grid (3, 2);
                
                % Sigmoid curve
                \draw[blue, ultra thick, domain=-3:3, smooth, variable=\x] 
                    plot (\x, {2/(1 + exp(-2*\x))});
                
                % Labels
                \node at (0, 1) [left] {$0.5$};
                \node at (0, 2) [left] {$1$};
                \node at (0, 0) [below left] {$0$};
                
                % Dotted lines
                \draw[dotted] (-3, 2) -- (3, 2);
                \draw[dotted] (0, 1) -- (3, 1);
            \end{tikzpicture}
        \end{column}
    \end{columns}
\end{frame}

\begin{frame}[c]{Propriedade Importante da Sigmoid}
    \begin{display}[Derivada da Função Sigmoid]
        \vspace{-.5cm}
        \begin{align}
            \sigma(z) &= \frac{1}{1 + e^{-z}} \\[0.5em]
            \sigma'(z) &= \frac{e^{-z}}{(1 + e^{-z})^2} \\[0.5em]
            &= \frac{1}{1 + e^{-z}} \cdot \frac{e^{-z}}{1 + e^{-z}} \\[0.5em]
            &= \frac{1}{1 + e^{-z}} \cdot \left(1 - \frac{1}{1 + e^{-z}}\right) \\[0.5em]
            &= \highlight{\sigma(z) \cdot (1 - \sigma(z))}
        \end{align}
        \vspace{-.5cm}
    \end{display}
    
\end{frame}

\begin{frame}[c]{Exemplo de Computação do Neurônio}
    \begin{columns}[c]
        \begin{column}{0.4\textwidth}
            \textbf{Dados de entrada:}
            \begin{itemize}
                \item $x_1 = 0.9$
                \item $x_2 = 0.2$
                \item $x_3 = 0.3$
            \end{itemize}
            
            \textbf{Pesos:}
            \begin{itemize}
                \item $w_1 = 2$
                \item $w_2 = 3$
                \item $w_3 = -1$
                \item $b = 0.5$
            \end{itemize}
        \end{column}
        \begin{column}{0.6\textwidth}
            \textbf{Cálculo:}
            \begin{align}
                z &= 0.9(2) + 0.2(3) + 0.3(-1) + 0.5 \\
                &= 1.8 + 0.6 - 0.3 + 0.5 \\
                &= 2.6 \\[0.5em]
                f(z) &= \frac{1}{1 + e^{-2.6}} \\
                &= 0.93
            \end{align}
            
            \highlight{O neurônio produziria o valor 0.93}
        \end{column}
    \end{columns}
\end{frame}

\section{Por que Redes Neurais?}
\begin{frame}[c]{Por que Redes Neurais?}
    \begin{columns}[c]
        \begin{column}{0.5\textwidth}
            \textbf{Um único neurônio (como regressão logística):}
            \begin{itemize}
                \item Só permite uma \highlight{fronteira de decisão linear}
                \item Limitado a problemas linearmente separáveis
                \item Não consegue capturar relações complexas
            \end{itemize}
        \end{column}
        \begin{column}{0.5\textwidth}
            \textbf{Redes com múltiplos neurônios:}
            \begin{itemize}
                \item Podem criar \highlight{fronteiras não-lineares}
                \item Capturam interações complexas
                \item Aproximam qualquer função contínua
                \item Resolvem problemas do mundo real!
            \end{itemize}
        \end{column}
    \end{columns}
    
    \vspace{0.5cm}
    
    \begin{attention}[Teorema da Aproximação Universal]
        Uma rede neural com uma camada oculta e suficientes neurônios pode aproximar qualquer função contínua!
    \end{attention}
\end{frame}

\section{Rede Neural Feedforward}
\begin{frame}[c]{Rede Neural Feedforward}
    \begin{center}
        \begin{tikzpicture}[scale=0.9]
            % Definir estilos
            \tikzstyle{neuron}=[circle, draw, minimum size=0.8cm]
            \tikzstyle{input}=[neuron, fill=cyan!30]
            \tikzstyle{hidden}=[neuron, fill=orange!40]
            \tikzstyle{output}=[neuron, fill=gray!50]
            
            % Camada de entrada
            \node[input] (x1) at (0, 2) {$x_1$};
            \node[input] (x2) at (0, 0) {$x_2$};
            \node[input] (x3) at (0, -2) {$x_3$};
            
            % Primeira camada oculta
            \foreach \i in {1,...,4} {
                \node[hidden] (h1\i) at (3, 3-\i) {$\sigma$};
            }
            
            % Segunda camada oculta
            \foreach \i in {1,...,4} {
                \node[hidden] (h2\i) at (6, 3-\i) {$\sigma$};
            }
            
            % Camada de saída
            \node[output] (y1) at (9, 1.5) {$\hat{y}_1$};
            \node[output] (y2) at (9, 0) {$\hat{y}_2$};
            \node[output] (y3) at (9, -1.5) {$\hat{y}_3$};
            
            % Desenhar conexões
            \foreach \x in {1,2,3} {
                \foreach \h in {1,2,3,4} {
                    \draw[-latex, yellow!70!black] (x\x) -- (h1\h);
                }
            }
            
            \foreach \h in {1,2,3,4} {
                \foreach \hh in {1,2,3,4} {
                    \draw[-latex, yellow!70!black] (h1\h) -- (h2\hh);
                }
            }
            
            \foreach \h in {1,2,3,4} {
                \foreach \y in {1,2,3} {
                    \draw[-latex, yellow!70!black] (h2\h) -- (y\y);
                }
            }
        \end{tikzpicture}
    \end{center}
    
    \begin{itemize}
        \item \highlight{Feedforward}: informação flui em uma direção (entrada → saída)
        \item Cada conexão tem um \highlight{peso} associado
        \item Cada neurônio tem um \highlight{bias}
    \end{itemize}
\end{frame}

\begin{frame}[c]{Camadas da Rede Neural}
    \begin{columns}[c]
        \begin{column}{0.33\textwidth}
            \centering
            \textbf{Camada de Entrada}
            \begin{tikzpicture}[scale=0.7]
                \node[circle, draw, fill=cyan!30] (x1) at (0, 1.5) {$x_1$};
                \node[circle, draw, fill=cyan!30] (x2) at (0, 0) {$x_2$};
                \node[circle, draw, fill=cyan!30] (x3) at (0, -1.5) {$x_3$};
                \draw[dashed, yellow!80!black, thick] (-0.8, -2.5) rectangle (0.8, 2.5);
            \end{tikzpicture}
            
            \begin{itemize}
                \small
                \item Recebe os dados
                \item Não processa
                \item Apenas repassa
            \end{itemize}
        \end{column}
        \begin{column}{0.33\textwidth}
            \centering
            \textbf{Camadas Ocultas}
            \begin{tikzpicture}[scale=0.7]
                \foreach \i in {1,2,3,4} {
                    \node[circle, draw, fill=orange!40] (h1\i) at (0, 2.5-\i) {$\sigma$};
                }
                \foreach \i in {1,2,3,4} {
                    \node[circle, draw, fill=orange!40] (h2\i) at (2, 2.5-\i) {$\sigma$};
                }
                \draw[dashed, yellow!80!black, thick] (-0.8, -2.5) rectangle (2.8, 2.5);
            \end{tikzpicture}
            
            \begin{itemize}
                \small
                \item Processamento
                \item Extração de features
                \item Não-linearidade
            \end{itemize}
        \end{column}
        \begin{column}{0.33\textwidth}
            \centering
            \textbf{Camada de Saída}
            
            \begin{tikzpicture}[scale=0.7]
                \node[circle, draw, fill=gray!50] (y1) at (0, 1.5) {$\hat{y}_1$};
                \node[circle, draw, fill=gray!50] (y2) at (0, 0) {$\hat{y}_2$};
                \node[circle, draw, fill=gray!50] (y3) at (0, -1.5) {$\hat{y}_3$};
                \draw[dashed, yellow!80!black, thick] (-0.8, -2.5) rectangle (0.8, 2.5);
            \end{tikzpicture}
            
            \begin{itemize}
                \small
                \item Produz resultado
                \item Formato depende do problema
                \item Classificação ou regressão
            \end{itemize}
        \end{column}
    \end{columns}
\end{frame}

\begin{frame}[c]{Pesos (Representados por Matrizes)}
    \begin{center}
        \begin{tikzpicture}[scale=0.8]
            % Camadas
            \node[circle, draw, fill=cyan!30] (x1) at (0, 1.5) {$x_1$};
            \node[circle, draw, fill=cyan!30] (x2) at (0, 0) {$x_2$};
            \node[circle, draw, fill=cyan!30] (x3) at (0, -1.5) {$x_3$};
            
            \foreach \i in {1,2,3,4} {
                \node[circle, draw, fill=orange!40] (h1\i) at (3, 2.5-\i) {$\sigma$};
            }
            
            \foreach \i in {1,2,3,4} {
                \node[circle, draw, fill=orange!40] (h2\i) at (6, 2.5-\i) {$\sigma$};
            }
            
            \node[circle, draw, fill=gray!50] (y1) at (9, 1.5) {$\hat{y}_1$};
            \node[circle, draw, fill=gray!50] (y2) at (9, 0) {$\hat{y}_2$};
            \node[circle, draw, fill=gray!50] (y3) at (9, -1.5) {$\hat{y}_3$};
            
            % Matrizes de pesos
            \node at (1.5, 3) {$W^{(1)}$};
            \node at (4.5, 3) {$W^{(2)}$};
            \node at (7.5, 3) {$W^{(3)}$};
            
            % Algumas conexões para ilustrar
            \foreach \x in {1,2,3} {
                \foreach \h in {1,2,3,4} {
                    \draw[-latex, yellow!60!black] (x\x) -- (h1\h);
                }
            }
            
            \foreach \h in {1,2,3,4} {
                \foreach \hh in {1,2,3,4} {
                    \draw[-latex, yellow!60!black] (h1\h) -- (h2\hh);
                }
            }
            
            \foreach \h in {1,2,3,4} {
                \foreach \y in {1,2,3} {
                    \draw[-latex, yellow!60!black] (h2\h) -- (y\y);
                }
            }
        \end{tikzpicture}
    \end{center}
    
    \begin{itemize}
        \item $W^{(1)}$: matriz de pesos da entrada para primeira camada oculta (3$\times$4)
        \item $W^{(2)}$: matriz de pesos entre camadas ocultas (4$\times$4)
        \item $W^{(3)}$: matriz de pesos da última camada oculta para saída (4$\times$3)
    \end{itemize}
\end{frame}

\section{Representação Matricial}
\begin{frame}[c]{Representação Matricial da Computação}
    Para a primeira camada oculta:
    
    \begin{columns}[c]
        \begin{column}{0.5\textwidth}
            \begin{tikzpicture}[scale=0.8]
                \node[circle, draw, fill=cyan!30] (x1) at (0, 1.5) {$x_1$};
                \node[circle, draw, fill=cyan!30] (x2) at (0, 0) {$x_2$};
                \node[circle, draw, fill=cyan!30] (x3) at (0, -1.5) {$x_3$};
                
                \foreach \i in {1,2,3,4} {
                    \node[circle, draw, fill=orange!40] (h\i) at (3, 2.5-\i) {$\sigma$};
                }
                
                \node at (1.5, 2.5) {$W^{(1)}$};
                \node at (2.5, 2.25) {$z^{(2)}$};
                \node at (3.5, -2.25) {$a^{(2)}$};
                
                \foreach \x in {1,2,3} {
                    \foreach \h in {1,2,3,4} {
                        \draw[-latex, yellow!60!black] (x\x) -- (h\h);
                    }
                }
            \end{tikzpicture}
        \end{column}
        \begin{column}{0.5\textwidth}
            \textbf{Cálculo:}
            \begin{align}
                \mathbf{x} &= [x_1, x_2, x_3] \quad (1\times 3)\\
                W^{(1)} &\text{ é uma matriz } 3\times 4 \\
                z^{(2)} &= \mathbf{x} W^{(1)} \quad (1\times 4)\\
                a^{(2)} &= \sigma(z^{(2)}) \quad (1\times 4)
            \end{align}
            
            onde $\mathbf{x} = a^{(1)}$ (entrada é a ativação da "camada 0")
        \end{column}
    \end{columns}
\end{frame}

\begin{frame}[c]{Continuando a Computação}
    \begin{display}[Para uma única instância de treinamento]
        \textbf{Entrada:} vetor $\mathbf{x}$ (vetor linha de tamanho 3)\\
        \textbf{Saída:} vetor $\hat{\mathbf{y}}$ (vetor linha de tamanho 3)
    \end{display}

    
    \begin{align}
        z^{(2)} &= \mathbf{x}W^{(1)} & a^{(2)} &= \sigma(z^{(2)}) \\
        z^{(3)} &= a^{(2)}W^{(2)} & a^{(3)} &= \sigma(z^{(3)}) \\
        z^{(4)} &= a^{(3)}W^{(3)} & \hat{\mathbf{y}} &= \text{softmax}(z^{(4)})
    \end{align}
\end{frame}

\begin{frame}[c]{Múltiplos Pontos de Dados}
    \begin{display}[Processamento em Lote]
        Na prática, fazemos essas computações para muitos pontos de dados ao mesmo tempo, "empilhando" as linhas em uma matriz. Mas as equações permanecem as mesmas!
    \end{display}
    
    \vspace{0.5cm}
    
    \textbf{Entrada:} matriz $X$ (uma matriz $n \times 3$) - cada linha é uma instância\\
    \textbf{Saída:} matriz $\hat{Y}$ (uma matriz $n \times 3$) - cada linha é uma predição
    
    \vspace{0.5cm}
    
    \begin{align}
        Z^{(2)} &= XW^{(1)} & A^{(2)} &= \sigma(Z^{(2)}) \\
        Z^{(3)} &= A^{(2)}W^{(2)} & A^{(3)} &= \sigma(Z^{(3)}) \\
        Z^{(4)} &= A^{(3)}W^{(3)} & \hat{Y} &= \text{softmax}(Z^{(4)})
    \end{align}
    
    \begin{alertblock}{Vantagem do Processamento em Lote}
        Operações matriciais são altamente otimizadas em GPUs e CPUs modernas!
    \end{alertblock}
\end{frame}

\begin{frame}[c]{Próximos Passos}
    \begin{center}
        \Large
        \textbf{Agora sabemos como redes neurais feedforward fazem computações.}
        
        \vspace{.25cm}
        
        \textbf{Próximo passo: aprender como ajustar os pesos para aprender dos dados.}
    \end{center}
    
    \vspace{1cm}
    
    \begin{columns}[c]
        \begin{column}{0.5\textwidth}
            \textbf{O que já sabemos:}
            \begin{itemize}
                \item Forward pass (propagação direta)
                \item Estrutura da rede
                \item Representação matricial
                \item Funções de ativação
            \end{itemize}
        \end{column}
        \begin{column}{0.5\textwidth}
            \textbf{Próxima aula:}
            \begin{itemize}
                \item \highlight{Backpropagation}
                \item Cálculo de gradientes
                \item Atualização de pesos
                \item Treinamento da rede
            \end{itemize}
        \end{column}
    \end{columns}

\end{frame}


\begin{frame}[c, noframenumbering, plain]
    \frametitle{~}
        \vfill
        \begin{center}
            {\Huge Obrigado!}\vspace{1.5em}\\
            {\Large \highlight{Alguma dúvida?}}\\
        \end{center}
        \vfill
        \begin{center}
            {\small Agora vamos para os exercícios!}
        \end{center}
\end{frame}

\end{document}