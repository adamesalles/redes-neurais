\documentclass[xcolor=dvipsnames,t,aspectratio=169]{beamer}

\usecolortheme{rose}
\usecolortheme{dolphin}
\usetheme{Boadilla}

% Encoding packages MUST come first
\usepackage[utf8]{inputenc}
\usepackage[brazil]{babel}
\usepackage[T1]{fontenc}

\usepackage{amsfonts}
\usepackage{amsmath}
\usepackage{amssymb}
% \usepackage{latexsym}
% \usepackage{graphicx}
% \usepackage{listings} 
\usepackage{xcolor}
\usepackage{amsthm}
\usepackage{url}
\usepackage{textpos}
% \usepackage{amssymb}
% \usepackage{caption}
% \usepackage{subcaption}
% \usepackage{multicol}
% \usepackage{mathrsfs}
\usepackage{float}
\setbeamersize{text margin left=1cm,text margin right=1cm,} 
\usepackage{subcaption}
\usepackage{dsfont}
\usepackage{physics}
\usepackage{mathpazo}
% \usepackage{tikz}
\usepackage{lipsum}
\usepackage{minted}
\usepackage[minted, most]{tcolorbox}
\tcbuselibrary{theorems}
\setbeamertemplate{enumerate items}[default]
\usefonttheme{professionalfonts}

\setlength{\parindent}{0pt} % zero indent

\setbeamertemplate{itemize items}[circle]
\setbeamertemplate{itemize subitem}{$\blacktriangleright$}
\definecolor{nes_dark_purple}{HTML}{4E006B}
\definecolor{nes_dark_orange}{HTML}{EE4F09}
\setbeamercolor{normal text}{fg=nes_dark_purple}\usebeamercolor*{normal text}
\setbeamercolor{math text}{fg=nes_dark_purple}\usebeamercolor[fg]{math text}


\setbeamercolor{title}{fg=nes_dark_purple}
\setbeamercolor{frametitle}{fg=nes_dark_purple}
\setbeamercolor{structure}{fg=nes_dark_purple}
\setbeamercolor{author}{fg=nes_dark_orange}
\setbeamercolor{footline}{fg=nes_dark_purple} 

\setbeamertemplate{footline}[frame number]
\setbeamertemplate{navigation symbols}{}

\definecolor{codegreen}{rgb}{0,0.6,0.2}
\definecolor{codegray}{rgb}{0.5,0.5,0.5}
\definecolor{codepurple}{rgb}{0.58,0,0.82}
\definecolor{backcolour}{rgb}{0.95,0.95,0.92}

\lstdefinestyle{mystyle}{
    backgroundcolor=\color{backcolour},   
    commentstyle=\color{codegreen},
    keywordstyle=\color{magenta},
    numberstyle=\tiny\color{codegray},
    stringstyle=\color{codepurple},
    basicstyle=\ttfamily\footnotesize,
    breakatwhitespace=false,  
    breaklines=true,                 
    captionpos=b,                    
    keepspaces=true,                 
    numbers=left,                    
    numbersep=5pt,                  
    showspaces=false,                
    showstringspaces=false,
    showtabs=false,                  
    tabsize=2
}
\input{../../templates/slides/commands}

\usepackage{tikz}
\usetikzlibrary{arrows.meta, shapes, positioning, calc}
\usepackage{algorithmic}
\usepackage{graphicx}
\usepackage{amsmath}
\usepackage{listings}
\usepackage{multicol}
\usepackage{svg}

% Configuração para código Python
\lstset{
    language=Python,
    basicstyle=\tiny\ttfamily,
    keywordstyle=\color{blue},
    commentstyle=\color{green!60!black},
    stringstyle=\color{orange},
    showstringspaces=false,
    breaklines=true,
    frame=single,
    numbers=left,
    numberstyle=\tiny\color{gray}
}

\titlegraphic{
    \includegraphics[scale = 0.5]{../../templates/slides/logo}
}

\logo{
\begin{tikzpicture}[overlay,remember picture]
\node[below left = 0.2cm] at (current page.30) {
    \includegraphics[width=0.1\textwidth]{../../templates/slides/logo}};
\end{tikzpicture}
}

\newcommand{\highlight}[1]{{\color{nes_dark_orange} #1}}

\title{LSTM + Extras} 

\author{
    Eduardo Adame
}

\date{{\color{nes_dark_purple}  \textbf{Redes Neurais}\\[0.5em] 18 de novembro de 2025 }}


\begin{document}


\frame[plain]{\titlepage}
\setcounter{framenumber}{0}


% ============================================================
% SLIDE 2 - AGENDA
% ============================================================
\begin{frame}[c]{Agenda da Aula}
\begin{itemize}
\item Revisão: Limitações das RNNs Padrão
\item Long Short-Term Memory (LSTM)
\begin{itemize}
\item Motivação e Arquitetura
\item Gates: Forget, Input, Output
\item Fluxo de Informação
\end{itemize}
\item Variantes: GRU e outras
\item \textbf{Extras: Tendências Modernas}
\begin{itemize}
\item Vision Transformers (ViT)
\item Diffusion Models
\end{itemize}
\item Encerramento do Curso
\end{itemize}
\end{frame}

\begin{frame}[c]{O Problema das RNNs Padrão}
\begin{display}[Limitação Fundamental]
As RNNs tradicionais sofrem do problema de \textbf{memória de curto prazo}
\end{display}

\begin{itemize}
\item A matriz de transição necessariamente \textbf{enfraquece o sinal} ao longo do tempo
\item \textbf{Vanishing gradients}: informação de passos distantes "desaparece"
\item \textbf{Exploding gradients}: instabilidade numérica
\item Dificuldade em capturar dependências de longo prazo
\end{itemize}

\vspace{0.3cm}
\begin{center}
\textit{Precisamos de uma estrutura que possa manter algumas dimensões inalteradas por muitos passos!}
\end{center}
\end{frame}

\begin{frame}[c]{LSTM: Tornando "Lembrar" Mais Fácil}
\begin{display}[Ideia Central]
Criar um mecanismo mais sofisticado de atualização do estado interno
\end{display}

\textbf{Princípio de Design:}
\begin{itemize}
\item Por padrão, LSTMs \textbf{lembram} a informação do passo anterior
\item Informações são \textbf{esquecidas} ou \textbf{adicionadas} como uma escolha ativa
\item Usa "gates" (portões) para controlar o fluxo de informação
\end{itemize}

\vspace{0.3cm}
\begin{center}
\textit{Publicado por Hochreiter \& Schmidhuber (1997)}\\
\textit{Revolucionou o processamento de sequências}
\end{center}
\end{frame}

\begin{frame}[c]{Arquitetura LSTM: Visão Geral}
\begin{figure}
    \centering
    \includegraphics[height=.4\textheight]{LSTM3-var-GRU.png}
\end{figure}

\textbf{Componentes principais:}
\begin{itemize}
\item $h_t$: hidden state (estado oculto) - saída da célula
\item $C_t$: cell state (estado da célula) - "memória" de longo prazo
\item $x_t$: entrada no tempo $t$
\item Três gates (portões) controlam o fluxo
\end{itemize}

\vspace{0.2cm}
Referência: \url{http://colah.github.io/posts/2015-08-Understanding-LSTMs/}
\end{frame}

\begin{frame}[c]{Gate 1: Forget Gate (Portão de Esquecimento)}
\begin{display}[Função]
Decide quais informações do estado anterior devem ser descartadas
\end{display}

\begin{equation*}
f_t = \sigma(W_f \cdot [h_{t-1}, x_t] + b_f)
\end{equation*}

\begin{itemize}
\item $\sigma$: função sigmoid (saída entre 0 e 1)
\item $f_t = 0$: "esqueça tudo"
\item $f_t = 1$: "mantenha tudo"
\item Aprende baseado em $h_{t-1}$ (contexto anterior) e $x_t$ (entrada atual)
\end{itemize}

\textbf{Exemplo:} Em processamento de texto, ao encontrar um novo sujeito, pode esquecer o gênero do sujeito anterior.
\end{frame}

\begin{frame}[c]{Gate 2: Input Gate (Portão de Entrada)}
\begin{display}[Função]

Decide quais novas informações serão armazenadas no estado da célula
\end{display}

\textbf{Duas operações:}
\begin{align*}
i_t &= \sigma(W_i \cdot [h_{t-1}, x_t] + b_i) \quad \text{(quanto adicionar)}\\
\tilde{C}_t &= \tanh(W_C \cdot [h_{t-1}, x_t] + b_C) \quad \text{(o que adicionar)}
\end{align*}

\begin{itemize}
\item $i_t$: sigmoid determina quais valores atualizar (0 a 1)
\item $\tilde{C}_t$: tanh cria vetor de candidatos (-1 a 1)
\item Multiplicação elemento-a-elemento combina ambos
\end{itemize}
\end{frame}

\begin{frame}[c]{Atualização do Cell State}
\begin{display}[Combinação: Esquecer + Adicionar]
O novo estado da célula combina informação antiga filtrada e nova informação
\end{display}

\begin{equation*}
C_t = f_t \odot C_{t-1} + i_t \odot \tilde{C}_t
\end{equation*}

\textbf{Notação:} $\odot$ representa multiplicação elemento-a-elemento (Hadamard)

\vspace{0.3cm}
\begin{itemize}
\item $f_t \odot C_{t-1}$: mantém partes do estado antigo
\item $i_t \odot \tilde{C}_t$: adiciona novas informações
\item Operação linear! Facilita propagação de gradientes
\end{itemize}

\textit{Esta é a "rodovia de informação" que resolve o problema de vanishing gradients}
\end{frame}

\begin{frame}[c]{Gate 3: Output Gate (Portão de Saída)}
\begin{display}[Função]
Decide qual parte do estado da célula será exposta como saída
\end{display}

\begin{align*}
o_t &= \sigma(W_o \cdot [h_{t-1}, x_t] + b_o)\\
h_t &= o_t \odot \tanh(C_t)
\end{align*}

\begin{itemize}
\item $o_t$: controla quanto do cell state mostrar
\item $\tanh(C_t)$: normaliza valores entre -1 e 1
\item $h_t$: hidden state, usado como saída e feed para próximo passo
\end{itemize}

\textbf{Nota:} Não há pesos na aplicação do $\tanh(C_t)$, apenas no gate $o_t$
\end{frame}

\begin{frame}[c]{LSTM: Todas as Equações}
\begin{display}[Sistema Completo]
\begin{align*}
f_t &= \sigma(W_f \cdot [h_{t-1}, x_t] + b_f) \quad \text{\small (forget gate)}\\
i_t &= \sigma(W_i \cdot [h_{t-1}, x_t] + b_i) \quad \text{\small (input gate)}\\
\tilde{C}_t &= \tanh(W_C \cdot [h_{t-1}, x_t] + b_C) \quad \text{\small (candidate values)}\\
C_t &= f_t \odot C_{t-1} + i_t \odot \tilde{C}_t \quad \text{\small (cell state)}\\
o_t &= \sigma(W_o \cdot [h_{t-1}, x_t] + b_o) \quad \text{\small (output gate)}\\
h_t &= o_t \odot \tanh(C_t) \quad \text{\small (hidden state)}
\end{align*}
\end{display}

\textbf{Parâmetros treináveis:} $W_f, W_i, W_C, W_o, b_f, b_i, b_C, b_o$
\end{frame}

\begin{frame}[c]{LSTM Desenrolado no Tempo}
\begin{figure}
    \centering
    \includegraphics[height=.4\textheight]{lstm_overtime.png}
\end{figure}

\begin{itemize}
\item Mesmos pesos compartilhados em todos os passos de tempo
\item Cell state $C_t$ flui horizontalmente com modificações mínimas
\item Hidden state $h_t$ conecta verticalmente às saídas
\item Backpropagation through time (BPTT) para treinamento
\end{itemize}
\end{frame}

\begin{frame}[c]{Variantes de LSTM}
\textbf{LSTM não é único! Várias arquiteturas foram propostas:}

\begin{itemize}
\item \textbf{GRU (Gated Recurrent Unit)} - Cho et al., 2014
\begin{itemize}
\item Simplificação do LSTM: combina forget e input gates
\item Menos parâmetros, mais rápido
\item Performance comparável em muitas tarefas
\end{itemize}

\item \textbf{Peephole Connections} - Gers \& Schmidhuber, 2000
\begin{itemize}
\item Gates podem "espiar" o cell state
\end{itemize}

\item \textbf{Coupled Gates}
\begin{itemize}
\item Forget e input gates acoplados: $i_t = 1 - f_t$
\end{itemize}
\end{itemize}

\vspace{0.2cm}
\textit{GRU tornou-se muito popular pela eficiência!}
\end{frame}

\begin{frame}[c]{GRU: Gated Recurrent Unit}
\textbf{Arquitetura simplificada com apenas 2 gates:}

\begin{align*}
z_t &= \sigma(W_z \cdot [h_{t-1}, x_t]) \quad \text{\small (update gate)}\\
r_t &= \sigma(W_r \cdot [h_{t-1}, x_t]) \quad \text{\small (reset gate)}\\
\tilde{h}_t &= \tanh(W \cdot [r_t \odot h_{t-1}, x_t])\\
h_t &= (1 - z_t) \odot h_{t-1} + z_t \odot \tilde{h}_t
\end{align*}

\textbf{Diferenças do LSTM:}
\begin{itemize}
\item Não há cell state separado
\item Menos parâmetros ($\sim 25\%$ menos)
\item Computação mais rápida
\item Performance similar em muitas aplicações
\end{itemize}
\end{frame}

\begin{frame}[c]{LSTM vs GRU: Quando Usar?}
\begin{center}
\begin{tabular}{|l|c|c|}
\hline
\textbf{Aspecto} & \textbf{LSTM} & \textbf{GRU} \\
\hline
Parâmetros & Mais & Menos \\
Velocidade & Mais lento & Mais rápido \\
Memória & Maior & Menor \\
Performance & Ligeiramente melhor* & Comparável \\
Sequências longas & Melhor & Bom \\
\hline
\end{tabular}
\end{center}

\vspace{0.3cm}
\textbf{Recomendação prática:}
\begin{itemize}
\item Comece com GRU (mais rápido para experimentar)
\item Use LSTM se precisar de controle mais fino
\item Teste ambos no seu problema específico
\item *Depende muito da tarefa!
\end{itemize}
\end{frame}

\begin{frame}[c]{Aplicações de LSTMs (2015-2020)}
\textbf{LSTMs dominaram processamento sequencial até recentemente:}

\begin{itemize}
\item \textbf{Processamento de Linguagem Natural}
\begin{itemize}
\item Tradução automática (seq2seq)
\item Análise de sentimento
\item Reconhecimento de entidades nomeadas
\end{itemize}

\item \textbf{Séries Temporais}
\begin{itemize}
\item Previsão de demanda
\item Análise financeira
\item Previsão climática
\end{itemize}

\item \textbf{Outros Domínios}
\begin{itemize}
\item Reconhecimento de fala
\item Geração de texto
\item Análise de vídeo
\item Geração de música
\end{itemize}
\end{itemize}
\end{frame}

\begin{frame}[c]{Extra 1: Vision Transformers (ViT)}
\begin{display}[A Revolução dos Transformers]
\textit{"Attention is All You Need"} - Vaswani et al., 2017
\end{display}

\textbf{Transformers vs RNN/LSTM:}
\begin{itemize}
\item Não processam sequencialmente - \textbf{paralelizável}!
\item Usam mecanismo de \textbf{self-attention}
\item Eliminaram domínio das RNNs em NLP (2017-2019)
\end{itemize}

\vspace{0.3cm}
\textbf{Vision Transformers (Dosovitskiy et al., 2020):}
\begin{itemize}
\item Aplicam Transformers a imagens
\item Dividem imagem em patches (pedaços)
\item Tratam patches como "tokens" de uma sequência
\item Superaram CNNs em muitas tarefas (com dados suficientes)
\end{itemize}
\end{frame}

\begin{frame}[c]{Vision Transformer: Arquitetura Básica}
\begin{figure}
    \centering
    \includegraphics[height=.4\textheight]{vit.png}
\end{figure}

\textbf{Pipeline:}
\begin{enumerate}
\item Divide imagem em patches fixos (ex: 16$\times$16 pixels)
\item Lineariza cada patch em vetor
\item Adiciona position embeddings
\item Passa pela arquitetura Transformer
\item Classification head na saída
\end{enumerate}

\textbf{Vantagem chave:} Atenção global desde a primeira camada (vs. campo receptivo local de CNNs)
\end{frame}

\begin{frame}[c]{Vision Transformers em 2025}
\textbf{Estado atual (tendências):}

\begin{itemize}
\item \textbf{Modelos Híbridos:} Combinam CNNs e Transformers
\begin{itemize}
\item ConvNeXt, CoAtNet, etc.
\end{itemize}

\item \textbf{Eficiência:} ViTs leves para dispositivos móveis
\begin{itemize}
\item MobileViT, EfficientViT
\end{itemize}

\item \textbf{Foundation Models:}
\begin{itemize}
\item CLIP (OpenAI): visão + linguagem
\item DINOv2 (Meta): self-supervised learning
\item SAM (Segment Anything Model)
\end{itemize}

\item \textbf{Aplicações:} Detecção de objetos, segmentação, geração de imagens
\end{itemize}

\textit{ViTs são componente fundamental dos modelos multimodais modernos!}
\end{frame}

\begin{frame}[c]{Extra 2: Diffusion Models}
\begin{display}[Geração de Imagens de Alta Qualidade]
Modelos de difusão revolucionaram geração de imagens (2020-2025)
\end{display}

\textbf{Exemplos famosos:}
\begin{itemize}
\item DALL-E 2, DALL-E 3 (OpenAI)
\item Stable Diffusion (Stability AI)
\item Midjourney
\item Imagen (Google)
\end{itemize}

\vspace{0.3cm}
\textbf{Por que são importantes?}
\begin{itemize}
\item Qualidade superior a GANs em muitos casos
\item Mais estáveis de treinar
\item Permitem controle detalhado (text-to-image)
\end{itemize}
\end{frame}

\begin{frame}[c]{Diffusion Models: Conceito Básico}
\textbf{Processo em duas fases:}

\begin{enumerate}
\item \textbf{Forward process (difusão):}
\begin{itemize}
\item Adiciona ruído gaussiano gradualmente à imagem
\item Em T passos: imagem $\rightarrow$ ruído puro
\item Processo simples e fixo (não aprendido)
\end{itemize}

\item \textbf{Reverse process (denoising):}
\begin{itemize}
\item Aprende a remover ruído passo a passo
\item Ruído puro $\rightarrow$ imagem
\item Usa rede neural (geralmente U-Net)
\end{itemize}
\end{enumerate}

\vspace{0.3cm}
\begin{center}
\textit{Treinamento: aprende a "prever o ruído" em cada passo}
\end{center}
\end{frame}

\begin{frame}[c]{Diffusion: Matemática Simplificada}
\textbf{Forward process (adicionando ruído):}
\begin{equation*}
q(x_t | x_{t-1}) = \mathcal{N}(x_t; \sqrt{1-\beta_t} x_{t-1}, \beta_t I)
\end{equation*}

\textbf{Reverse process (removendo ruído):}
\begin{equation*}
p_\theta(x_{t-1} | x_t) = \mathcal{N}(x_{t-1}; \mu_\theta(x_t, t), \Sigma_\theta(x_t, t))
\end{equation*}

\textbf{Objetivo de treinamento (simplificado):}
\begin{equation*}
\mathcal{L} = \mathbb{E}_{t, x_0, \epsilon} \left[ \| \epsilon - \epsilon_\theta(x_t, t) \|^2 \right]
\end{equation*}

\textit{Aprende a prever o ruído $\epsilon$ que foi adicionado!}
\end{frame}

\begin{frame}[c]{Diffusion Models: Arquitetura}
\textbf{Componentes principais:}

\begin{itemize}
\item \textbf{U-Net com attention:}
\begin{itemize}
\item Encoder-decoder com skip connections
\item Self-attention em múltiplas escalas
\item Recebe timestep $t$ como input adicional
\end{itemize}

\item \textbf{Conditioning (para text-to-image):}
\begin{itemize}
\item Text embeddings via CLIP ou T5
\item Cross-attention entre imagem e texto
\item Permite controle semântico da geração
\end{itemize}

\item \textbf{Latent Diffusion (Stable Diffusion):}
\begin{itemize}
\item Trabalha em espaço latente (comprimido)
\item Muito mais eficiente computacionalmente
\item VAE para encoder/decoder
\end{itemize}
\end{itemize}
\end{frame}

\begin{frame}[c]{Diffusion Models em 2025}
\textbf{Avanços recentes:}

\begin{itemize}
\item \textbf{Velocidade:} Modelos de poucos passos (vs. 50-1000 originais)
\begin{itemize}
\item LCM (Latent Consistency Models)
\item SDXL Turbo
\end{itemize}

\item \textbf{Controle:} ControlNet, IP-Adapter
\begin{itemize}
\item Controle fino de pose, edges, depth
\end{itemize}

\item \textbf{Vídeo:} Runway Gen-2, Pika, Stable Video Diffusion

\item \textbf{3D:} Geração de objetos 3D

\item \textbf{Outras modalidades:} Áudio, música, moléculas
\end{itemize}

\textit{Diffusion tornou-se framework dominante para geração!}
\end{frame}

\begin{frame}[c]{Evolução: RNNs → Transformers → Diffusion}
\begin{center}
\begin{tabular}{|l|c|c|c|}
\hline
\textbf{Aspecto} & \textbf{RNN/LSTM} & \textbf{Transformers} & \textbf{Diffusion} \\
\hline
Ano & 1997-2015 & 2017+ & 2020+ \\
Paralelização & Não & Sim & Parcial \\
Domínio & Sequências & Múltiplo & Geração \\
Uso hoje & Menor & Dominante & Crescente \\
\hline
\end{tabular}
\end{center}

\vspace{0.3cm}
\textbf{Mensagem principal:}
\begin{itemize}
\item LSTMs foram fundamentais, mas Transformers dominam hoje
\item Cada arquitetura resolveu problemas específicos
\item Evolução continua: novos modelos surgem constantemente
\item Fundamentos (gradientes, otimização) permanecem
\end{itemize}
\end{frame}

\begin{frame}[c]{Exemplo Interativo: LSTM Playground}
\textbf{Recomendação de demonstração online:}

\begin{display}[TensorFlow Playground - RNN Extension]
\url{https://distill.pub/2019/memorization-in-rnns/}
\end{display}

\textbf{Outras opções:}
\begin{itemize}
\item \textbf{LSTMVis:} \url{http://lstm.seas.harvard.edu/}
\begin{itemize}
\item Visualização de estados internos de LSTM
\item Análise de texto
\end{itemize}

\item \textbf{Hugging Face Spaces:}
\begin{itemize}
\item Text generation com GPT (Transformer)
\item Stable Diffusion demos
\end{itemize}
\end{itemize}

\textit{Experimente você mesmo para entender o comportamento!}
\end{frame}

\begin{frame}[c, fragile]{Implementação: LSTM em PyTorch}
\begin{code}[Exemplo básico]{python}
import torch.nn as nn
class LSTMModel(nn.Module):
    def __init__(self, input_size, hidden_size, 
                 num_layers, output_size):
        super().__init__()
        self.lstm = nn.LSTM(input_size, hidden_size, 
                           num_layers, batch_first=True)
        self.fc = nn.Linear(hidden_size, output_size)
    def forward(self, x):
        # x: (batch, seq_len, input_size)
        lstm_out, (h_n, c_n) = self.lstm(x)
        # Última saída
        out = self.fc(lstm_out[:, -1, :])
        return out
\end{code}
\end{frame}

\begin{frame}[c, fragile]{Implementação: GRU em PyTorch}
\begin{code}[Ainda mais simples]{python}
class GRUModel(nn.Module):
    def __init__(self, input_size, hidden_size, 
                 num_layers, output_size):
        super().__init__()
        self.gru = nn.GRU(input_size, hidden_size, 
                         num_layers, batch_first=True)
        self.fc = nn.Linear(hidden_size, output_size)
    def forward(self, x):
        gru_out, h_n = self.gru(x)
        out = self.fc(gru_out[:, -1, :])
        return out
\end{code}

\textit{PyTorch e TensorFlow implementam todas as otimizações!}
\end{frame}

\begin{frame}[c]{Dicas Práticas para Usar LSTMs/GRUs}
\textbf{Ao trabalhar com RNNs:}

\begin{enumerate}
\item \textbf{Normalização:} Use BatchNorm ou LayerNorm
\item \textbf{Regularização:} Dropout entre camadas
\item \textbf{Gradient clipping:} Essencial! (ex: clip à norm 5.0)
\item \textbf{Inicialização:} Xavier/He, bias do forget gate = 1
\item \textbf{Bidirectional:} Quando não há restrição temporal
\item \textbf{Tamanho:} 128-512 hidden units comum
\item \textbf{Camadas:} 1-4 layers (mais nem sempre é melhor)
\end{enumerate}

\vspace{0.3cm}
\textbf{Para produção hoje (2025):}
\begin{itemize}
\item Considere Transformers primeiro (mais suporte)
\item Use LSTMs quando sequências são muito longas
\item GRUs quando recursos computacionais são limitados
\end{itemize}
\end{frame}

\begin{frame}[c]{Panorama de Deep Learning em 2025}
\textbf{Arquiteturas dominantes por domínio:}

\begin{itemize}
\item \textbf{Linguagem:} Transformers (GPT, BERT, Claude, etc.)
\item \textbf{Visão:} ViT, híbridos CNN-Transformer
\item \textbf{Geração de Imagens:} Diffusion Models
\item \textbf{Séries Temporais:} Ainda mix (LSTM, Transformers, N-BEATS)
\item \textbf{Recomendação:} Deep Learning + métodos clássicos
\item \textbf{Multimodal:} Transformers com múltiplos encoders
\end{itemize}

\vspace{0.3cm}
\textbf{Tendências:}
\begin{itemize}
\item Foundation models cada vez maiores
\item Fine-tuning e transfer learning
\item Eficiência: quantização, pruning, distillation
\end{itemize}
\end{frame}

\begin{frame}[c]{Reflexões sobre o Curso}
\textbf{O que vimos ao longo do semestre:}

\begin{itemize}
\item \textbf{Fundamentos:} Perceptrons, MLPs, backpropagation
\item \textbf{Otimização:} SGD, Adam, learning rate, regularização
\item \textbf{CNNs:} Convolução, pooling, arquiteturas clássicas
\item \textbf{Transfer Learning:} Fine-tuning, feature extraction
\item \textbf{RNNs e LSTMs:} Processamento sequencial
\item \textbf{Tópicos avançados:} Attention, Transformers, Diffusion
\end{itemize}

\vspace{0.3cm}
\begin{center}
\textit{"A jornada de mil milhas começa com um único passo"}\\
\vspace{0.2cm}
Vocês agora têm a base para explorar qualquer arquitetura moderna!
\end{center}
\end{frame}

\begin{frame}[c]{Recursos para Continuar Aprendendo}
\textbf{Cursos online:}
\begin{itemize}
\item Fast.ai - Practical Deep Learning
\item CS231n (Stanford) - Computer Vision
\item CS224n (Stanford) - NLP
\item Deep Learning Specialization (Coursera)
\end{itemize}

\textbf{Leitura:}
\begin{itemize}
\item Deep Learning (Goodfellow, Bengio, Courville)
\item Papers with Code: \url{https://paperswithcode.com/}
\item Distill.pub - Visualizações interativas
\end{itemize}

\textbf{Prática:}
\begin{itemize}
\item Kaggle competitions
\item Projetos pessoais
\item Contribuir para open source
\end{itemize}
\end{frame}

\begin{frame}[c]{Próximos Passos Sugeridos}
\textbf{Após este curso, você pode:}

\begin{enumerate}
\item \textbf{Especializar-se:}
\begin{itemize}
\item Computer Vision (detecção, segmentação, tracking)
\item NLP (LLMs, RAG, agents)
\item Generative AI (GANs, Diffusion, VAEs)
\item Reinforcement Learning
\end{itemize}

\item \textbf{Aplicar:}
\begin{itemize}
\item Projetos de pesquisa
\item Competições (Kaggle, etc.)
\item Startup/empresa
\end{itemize}

\item \textbf{Aprofundar teoria:}
\begin{itemize}
\item Otimização matemática
\item Teoria da informação
\item Probabilidade e estatística bayesiana
\end{itemize}
\end{enumerate}
\end{frame}

\begin{frame}[c, plain]{}
\begin{center}
{\Large \textbf{O campo de Deep Learning está}}\\[0.3cm]
{\Large \textbf{em constante evolução}}\\[0.4cm]

\vspace{0.4cm}
\textit{O que vocês aprenderam aqui são os fundamentos}\\
\textit{que permanecem, mesmo quando surgem novas arquiteturas}\\[0.4cm]

\vspace{0.4cm}
{\large Backpropagation, gradientes, otimização,}\\
{\large regularização, avaliação...}\\[0.3cm]

\vspace{0.4cm}
{\Large \textbf{Esses conceitos são eternos!}}\\[0.4cm]

\vspace{0.4cm}
\textit{Continue aprendendo, experimentando, e questionando.}\\
\textit{O melhor ainda está por vir!}
\end{center}
\end{frame}

\begin{frame}[c, noframenumbering, plain]
    \frametitle{~}
    \vfill
    \begin{center}
        {\Huge Obrigado!}\vspace{1.5em}\\
        {\Large \highlight{Dúvidas?}}\\
    \end{center}
    \vfill
    \begin{center}
        {\small Espero que tenham gostado do curso, foi um prazer!}\\
        {\small Contem comigo \highlight{sempre} que precisarem, estarei na torcida.}
    \end{center}

    \begin{flushright}
        {\small Eduardo Adame, 2025}
    \end{flushright}
\end{frame}

% ============================================================
% REFERÊNCIAS
% ============================================================
\begin{frame}[c]{Referências Principais}
\footnotesize
\textbf{Papers Fundamentais:}
\begin{itemize}
\item Hochreiter \& Schmidhuber (1997). Long Short-Term Memory. Neural Computation.
\item Cho et al. (2014). Learning Phrase Representations using RNN Encoder-Decoder.
\item Vaswani et al. (2017). Attention is All You Need. NeurIPS.
\item Dosovitskiy et al. (2020). An Image is Worth 16x16 Words: Transformers for Image Recognition.
\item Ho et al. (2020). Denoising Diffusion Probabilistic Models. NeurIPS.
\item Rombach et al. (2022). High-Resolution Image Synthesis with Latent Diffusion Models. CVPR.
\end{itemize}

\end{frame}

\end{document}