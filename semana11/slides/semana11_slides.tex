\documentclass[xcolor=dvipsnames,t,aspectratio=169]{beamer}

\usecolortheme{rose}
\usecolortheme{dolphin}
\usetheme{Boadilla}

% Encoding packages MUST come first
\usepackage[utf8]{inputenc}
\usepackage[brazil]{babel}
\usepackage[T1]{fontenc}

\usepackage{amsfonts}
\usepackage{amsmath}
\usepackage{amssymb}
% \usepackage{latexsym}
% \usepackage{graphicx}
% \usepackage{listings} 
\usepackage{xcolor}
\usepackage{amsthm}
\usepackage{url}
\usepackage{textpos}
% \usepackage{amssymb}
% \usepackage{caption}
% \usepackage{subcaption}
% \usepackage{multicol}
% \usepackage{mathrsfs}
\usepackage{float}
\setbeamersize{text margin left=1cm,text margin right=1cm,} 
\usepackage{subcaption}
\usepackage{dsfont}
\usepackage{physics}
\usepackage{mathpazo}
% \usepackage{tikz}
\usepackage{lipsum}
\usepackage{minted}
\usepackage[minted, most]{tcolorbox}
\tcbuselibrary{theorems}
\setbeamertemplate{enumerate items}[default]
\usefonttheme{professionalfonts}

\setlength{\parindent}{0pt} % zero indent

\setbeamertemplate{itemize items}[circle]
\setbeamertemplate{itemize subitem}{$\blacktriangleright$}
\definecolor{nes_dark_purple}{HTML}{4E006B}
\definecolor{nes_dark_orange}{HTML}{EE4F09}
\setbeamercolor{normal text}{fg=nes_dark_purple}\usebeamercolor*{normal text}
\setbeamercolor{math text}{fg=nes_dark_purple}\usebeamercolor[fg]{math text}


\setbeamercolor{title}{fg=nes_dark_purple}
\setbeamercolor{frametitle}{fg=nes_dark_purple}
\setbeamercolor{structure}{fg=nes_dark_purple}
\setbeamercolor{author}{fg=nes_dark_orange}
\setbeamercolor{footline}{fg=nes_dark_purple} 

\setbeamertemplate{footline}[frame number]
\setbeamertemplate{navigation symbols}{}

\definecolor{codegreen}{rgb}{0,0.6,0.2}
\definecolor{codegray}{rgb}{0.5,0.5,0.5}
\definecolor{codepurple}{rgb}{0.58,0,0.82}
\definecolor{backcolour}{rgb}{0.95,0.95,0.92}

\lstdefinestyle{mystyle}{
    backgroundcolor=\color{backcolour},   
    commentstyle=\color{codegreen},
    keywordstyle=\color{magenta},
    numberstyle=\tiny\color{codegray},
    stringstyle=\color{codepurple},
    basicstyle=\ttfamily\footnotesize,
    breakatwhitespace=false,  
    breaklines=true,                 
    captionpos=b,                    
    keepspaces=true,                 
    numbers=left,                    
    numbersep=5pt,                  
    showspaces=false,                
    showstringspaces=false,
    showtabs=false,                  
    tabsize=2
}
\input{../../templates/slides/commands}

\usepackage{tikz}
\usetikzlibrary{arrows.meta, shapes, positioning, calc}
\usepackage{algorithmic}
\usepackage{graphicx}
\usepackage{amsmath}
\usepackage{listings}
\usepackage{multicol}
\usepackage{svg}

% Configuração para código Python
\lstset{
    language=Python,
    basicstyle=\tiny\ttfamily,
    keywordstyle=\color{blue},
    commentstyle=\color{green!60!black},
    stringstyle=\color{orange},
    showstringspaces=false,
    breaklines=true,
    frame=single,
    numbers=left,
    numberstyle=\tiny\color{gray}
}

\titlegraphic{
    \includegraphics[scale = 0.5]{../../templates/slides/logo}
}

\logo{
\begin{tikzpicture}[overlay,remember picture]
\node[below left = 0.2cm] at (current page.30) {
    \includegraphics[width=0.1\textwidth]{../../templates/slides/logo}};
\end{tikzpicture}
}

\newcommand{\highlight}[1]{{\color{nes_dark_orange} #1}}

\title{Redes Neurais Recorrentes} 

\author{
    Eduardo Adame
}

\date{{\color{nes_dark_purple}  \textbf{Redes Neurais}\\[0.5em] 12 de novembro de 2025 }}


\begin{document}

\frame[plain]{\titlepage}
\setcounter{framenumber}{0}



\begin{frame}[c]{O Desafio: Sequências de Comprimento Variável}

\begin{itemize}
    \item Com imagens, forçamos uma dimensão de entrada específica
    \vspace{0.3cm}
    \item Não é óbvio como fazer isso com texto ou outras sequências
    \vspace{0.3cm}
    \item \textbf{Exemplo:} Classificar tweets como positivos, negativos ou neutros
    \vspace{0.3cm}
    \item Tweets podem ter um número variável de palavras
    \vspace{0.3cm}
    \item \textbf{Outros exemplos:}
    \begin{itemize}
        \item Análise de séries temporais (preços de ações)
        \item Reconhecimento de fala
        \item Tradução automática
        \item Análise de sequências de DNA
    \end{itemize}
\end{itemize}

\end{frame}

\begin{frame}[c]{O Problema da Ordem das Palavras}

\begin{itemize}
    \item Queremos fazer melhor que implementações de "bag of words" (saco de palavras)
    \item Idealmente, cada palavra é processada ou compreendida no contexto apropriado
    \item Precisamos ter alguma noção de \textbf{"contexto"}
    \item As palavras devem ser tratadas de forma diferente dependendo do contexto
    \item Além disso, cada palavra deve atualizar o contexto
\end{itemize}

\vspace{0.25cm}
\begin{attention}[Exemplo]
"O banco estava cheio" vs "O banco do parque estava vazio"\\
A palavra "banco" tem significados diferentes dependendo do contexto!
\end{attention}

\end{frame}


\begin{frame}[c]{Ideia: Usar o Conceito de "Recorrência"}

\begin{itemize}
    \item Processar palavras uma por uma
    \vspace{0.3cm}
    \item A rede produz duas saídas:
    \begin{enumerate}
        \item \textbf{Predição:} Qual seria a predição se a sequência terminasse naquela palavra
        \item \textbf{Estado:} Resumo de tudo que aconteceu no passado
    \end{enumerate}
    \vspace{0.3cm}
    \item Dessa forma, podemos lidar com comprimentos variáveis de texto
    \vspace{0.3cm}
    \item A resposta a uma palavra depende das palavras que a precederam
\end{itemize}

\end{frame}

\begin{frame}[c]{Visualização da Recorrência}

% \begin{center}
% \textit{[PEDIR PARA O USUÁRIO: Por favor, busque uma imagem mostrando o diagrama circular de RNN com:]
% \begin{itemize}
%     \item Um círculo representando o ESTADO
%     \item Seta de entrada (de baixo)
%     \item Seta de saída (para cima)
%     \item Seta de recorrência (lateral)
% \end{itemize}
% \textit{Similar ao slide 5 do PDF original]}
% \end{center}

\begin{figure}[h]
\centering
\includegraphics[width=.75\textwidth]{recurrent_neural_network.png}
\end{figure}


O estado é \textbf{recorrente} - ele se alimenta de volta para a próxima etapa temporal!

\end{frame}


\begin{frame}[c]{``Desenrolando'' a RNN}

\begin{figure}[h]
\centering
\includesvg[width=.9\textwidth]{Recurrent_neural_network_unfold.svg}
\end{figure}

\vspace{0.3cm}
Visualização "desenrolada" no tempo ajuda a entender o fluxo de informação

\end{frame}

\begin{frame}[c]{Componentes da RNN Desenrolada}

\begin{columns}
\column{0.5\textwidth}
\textbf{Elementos:}
\begin{itemize}
    \item $w_i$: palavra na posição $i$
    \item $s_i$: estado na posição $i$
    \item $o_i$: saída na posição $i$
    \item $U$: pesos de entrada (kernel)
    \item $W$: pesos recorrentes
    \item $V$: pesos de saída
\end{itemize}

\column{0.5\textwidth}
\textbf{Importante:}
\begin{attention}[Compartilhamento]
As matrizes $U$, $W$, $V$ são as \textbf{mesmas} em todas as posições temporais!
\end{attention}
\end{columns}

\vspace{0.5cm}
Isso permite generalização através do tempo e reduz drasticamente o número de parâmetros.

\end{frame}

\begin{frame}[c]{Exemplo Prático}

\begin{center}
Considere a frase: \textbf{"Isso é ótimo!"}
\end{center}

\begin{table}[]
\centering
\begin{tabular}{c|c|c|c}
\textbf{Posição} & \textbf{Palavra} & \textbf{Estado} & \textbf{Predição} \\ \hline
1 & "Isso" & $s_1$ & ? \\
2 & "é" & $s_2$ & ? \\
3 & "ótimo" & $s_3$ & + \\
4 & "!" & $s_4$ & ++ \\
\end{tabular}
\end{table}

\vspace{0.3cm}
\begin{itemize}
    \item A predição fica mais confiante à medida que mais palavras são processadas
    \item O estado $s_i$ carrega informação de todas as palavras anteriores
\end{itemize}

\end{frame}


\begin{frame}[c]{Fórmulas das RNNs}

\begin{display}[Equações Fundamentais]
\begin{align*}
s_i &= f(Uw_i + Ws_{i-1}) \quad \text{(RNN principal)} \\
o_i &= \text{softmax}(Vs_i) \quad \text{(camada densa subsequente)}
\end{align*}
\end{display}

% \vspace{0.3cm}
\textbf{Em outras palavras:}
\begin{itemize}
    \item estado atual = função(estado anterior, entrada atual)
    \item saída atual = função(estado atual)
    \item Aprendemos as funções através do treinamento da rede!
\end{itemize}

% \vspace{0.3cm}
$f$ é tipicamente uma função de ativação não-linear (tanh ou ReLU)

\end{frame}

\begin{frame}[c]{Dimensões das Matrizes}

\begin{itemize}
    \item $r$ = dimensão do vetor de entrada
    \item $s$ = dimensão do estado oculto
    \item $t$ = dimensão do vetor de saída (após camada densa)
\end{itemize}

\vspace{0.5cm}

\begin{display}[Dimensões das Matrizes de Peso]
\begin{itemize}
    \item $U$ é uma matriz $s \times r$
    \item $W$ é uma matriz $s \times s$
    \item $V$ é uma matriz $t \times s$
\end{itemize}
\end{display}

\vspace{0.3cm}
\textbf{Nota importante:} As matrizes $U$, $V$, $W$ são as mesmas em todas as posições temporais.

\end{frame}


\begin{frame}[c, fragile]{Componentes no Keras/TensorFlow}

\begin{itemize}
    \item \textbf{Matriz $U$ (kernel):} Controlada por \texttt{kernel\_initializer}
    \vspace{0.3cm}
    \item \textbf{Matriz $W$ (recorrente):} Controlada por \texttt{recurrent\_initializer}
    \vspace{0.3cm}
    \item \textbf{Matriz $V$ (saída):} Implementada como uma camada \texttt{Dense} subsequente
\end{itemize}

\vspace{0.5cm}


\begin{code}[Exemplo de Código]{python}
model.add(SimpleRNN(units=128,
                    kernel_initializer='glorot_uniform',
                    recurrent_initializer='orthogonal'))
model.add(Dense(num_classes, activation='softmax'))
\end{code}

\end{frame}


\begin{frame}[c, fragile]{Detalhes Práticos de Implementação}

\begin{itemize}
    \item Frequentemente, treinamos apenas na saída "final" e ignoramos as saídas intermediárias
    \vspace{0.3cm}
    \item Uma variação chamada \textbf{Backpropagation Through Time (BPTT)} é usada para treinar RNNs
    \vspace{0.3cm}
    \item Sensível ao comprimento da sequência (devido ao problema de "gradiente desaparecendo/explodindo")
    \vspace{0.3cm}
    \item Na prática, ainda definimos um comprimento máximo para nossas sequências:
    \begin{itemize}
        \item Se a entrada é mais curta, fazemos \textbf{padding} (preenchimento)
        \item Se a entrada é mais longa, truncamos
    \end{itemize}
\end{itemize}

\end{frame}

\begin{frame}[c, fragile]{Exemplo de Código: Preparação dos Dados}

\begin{code}[Pré-processamento de Sequências]{python}
from tensorflow.keras.preprocessing import sequence

# Definir comprimento máximo
max_length = 100

# Padding das sequências
X_train = sequence.pad_sequences(X_train, 
                                 maxlen=max_length,
                                 padding='post',
                                 truncating='post')
\end{code}

\texttt{padding='post'}: Adiciona zeros no final\\
\texttt{truncating='post'}: Corta no final se exceder max\_length

\end{frame}

% ============================================
% SLIDE 15
% ============================================
\begin{frame}[c, fragile]{Exemplo de Código: Construindo o Modelo}


\begin{code}[Modelo RNN Simples para Classificação de Sentimento]{python}
from tensorflow.keras.models import Sequential
from tensorflow.keras.layers import Embedding, SimpleRNN, Dense

model = Sequential([
    Embedding(vocab_size, embedding_dim, 
              input_length=max_length),
    SimpleRNN(64, return_sequences=False),
    Dense(1, activation='sigmoid')
])

model.compile(optimizer='adam',
              loss='binary_crossentropy',
              metrics=['accuracy'])
\end{code}

\end{frame}

% ============================================
% SLIDE 16
% ============================================
\begin{frame}[c]{Embedding Layer}

\begin{itemize}
    \item Antes de processar palavras, precisamos convertê-las em vetores numéricos
    \vspace{0.3cm}
    \item \textbf{Embedding} aprende uma representação densa de palavras
    \vspace{0.3cm}
    \item Cada palavra é mapeada para um vetor de tamanho fixo
    \vspace{0.3cm}
    \item Palavras semanticamente similares terão embeddings próximos
\end{itemize}

\vspace{0.5cm}

\begin{display}[Exemplo]
\begin{tabular}{l|l}
\textbf{Palavra} & \textbf{Embedding (simplificado)} \\ \hline
"bom" & [0.8, 0.3, -0.1, ...] \\
"ótimo" & [0.85, 0.35, -0.05, ...] \\
"ruim" & [-0.7, -0.4, 0.2, ...] \\
\end{tabular}
\end{display}

\end{frame}


\begin{frame}[c]{Aplicações de RNNs (2025)}

\textbf{Processamento de Linguagem Natural:}
\begin{itemize}
    \item Análise de sentimento
    \item Classificação de textos
    \item Geração de texto (chatbots básicos)
    \item Modelagem de linguagem
\end{itemize}

\vspace{0.3cm}

\textbf{Séries Temporais:}
\begin{itemize}
    \item Previsão de preços de ações
    \item Previsão de demanda de energia
    \item Análise de dados climáticos
    \item Monitoramento de sensores IoT
\end{itemize}

\end{frame}

\begin{frame}[c]{Mais Aplicações de RNNs}

\textbf{Reconhecimento de Fala:}
\begin{itemize}
    \item Transcrição de áudio para texto
    \item Comandos de voz
\end{itemize}

\vspace{0.3cm}

\textbf{Bioinformática:}
\begin{itemize}
    \item Análise de sequências de DNA
    \item Predição de estrutura de proteínas
    \item Classificação de sequências genômicas
\end{itemize}

\vspace{0.3cm}

\textbf{Outras Aplicações:}
\begin{itemize}
    \item Reconhecimento de escrita manuscrita
    \item Análise de vídeo (frame por frame)
    \item Detecção de anomalias em séries temporais
\end{itemize}

\end{frame}


\begin{frame}[c]{Exemplo Interativo Online}

\begin{display}[TensorFlow Playground - RNN Visualizer]
\textbf{URL:} \url{https://distill.pub/2019/memorization-in-rnns/}
\end{display}

\vspace{0.3cm}

\textbf{Este site interativo permite:}
\begin{itemize}
    \item Visualizar como RNNs processam sequências
    \item Ver como o estado oculto evolui ao longo do tempo
    \item Experimentar com diferentes arquiteturas
    \item Entender o problema de memória de longo prazo
\end{itemize}


\end{frame}



\begin{frame}[c]{Problema do Gradiente Desaparecendo/Explodindo}

\begin{columns}
\column{0.5\textwidth}
\textbf{Gradiente Desaparecendo:}
\begin{itemize}
    \item Gradientes ficam muito pequenos
    \item Informação do passado distante é "esquecida"
    \item Dificuldade em aprender dependências de longo prazo
\end{itemize}

\column{0.5\textwidth}
\textbf{Gradiente Explodindo:}
\begin{itemize}
    \item Gradientes ficam muito grandes
    \item Instabilidade no treinamento
    \item Pode ser mitigado com gradient clipping
\end{itemize}
\end{columns}

\vspace{0.5cm}

\begin{display}[Exemplo]
Tentando lembrar o sujeito de uma frase:
\textit{"O homem que \textcolor{gray}{... muitas palavras ...} estava cansado"}\\
RNNs simples têm dificuldade em manter "homem" na memória.
\end{display}

\end{frame}

\begin{frame}[c]{Limitações das RNNs Simples}

\begin{itemize}
    \item \textbf{Memória de curto prazo:} Dificuldade em manter informação do passado distante sem reforço
    \vspace{0.3cm}
    \item \textbf{Processamento sequencial:} Não pode ser facilmente paralelizado
    \vspace{0.3cm}
    \item \textbf{Sensibilidade ao comprimento:} Performance degrada com sequências muito longas
    \vspace{0.3cm}
    \item \textbf{Treinamento lento:} Backpropagation through time é computacionalmente custoso
\end{itemize}

% \vspace{0.5cm}

\begin{attention}[Solução]
Na próxima aula, veremos \textbf{LSTMs (Long Short-Term Memory)}, que foram projetadas especificamente para resolver esses problemas!
\end{attention}

\end{frame}


\begin{frame}[c]{Comparação: RNN vs Outras Arquiteturas (2025)}

\begin{table}[]
\centering
\small
\begin{tabular}{l|c|c|c}
\textbf{Arquitetura} & \textbf{Paralelização} & \textbf{Memória} & \textbf{Uso em 2025} \\ \hline
RNN Simples & $\times$ & Curto prazo & Limitado \\
LSTM/GRU & $\times$ & Longo prazo & Moderado \\
Transformers & $\checkmark$ & Atenção total & Dominante \\
CNN 1D & $\checkmark$ & Local & Específico \\
\end{tabular}
\end{table}

\vspace{0.5cm}

\textbf{Nota:} Embora Transformers dominem em 2025, RNNs ainda são:
\begin{itemize}
    \item Fundamentais para entender arquiteturas modernas
    \item Úteis em cenários com recursos limitados
    \item Base para variantes mais eficientes
\end{itemize}

\end{frame}


\begin{frame}[c]{Prévia da Próxima Aula: LSTMs}

\begin{center}
\Large \textbf{Long Short-Term Memory (LSTM)}
\end{center}

\vspace{0.5cm}

\textbf{O que veremos:}
\begin{itemize}
    \item Arquitetura LSTM: gates de entrada, saída e esquecimento
    \item Como LSTMs resolvem o problema do gradiente desaparecendo
    \item Variante GRU (Gated Recurrent Unit)
    \item Aplicações práticas: tradução automática, chatbots, geração de texto
    \item Implementação em TensorFlow/Keras
    \item Comparação RNN vs LSTM vs GRU
\end{itemize}

\end{frame}

\begin{frame}[c]{Referências e Recursos}

\textbf{Leitura Recomendada:}
\begin{itemize}
    \item Goodfellow et al. - Deep Learning, Capítulo 10 (RNNs)
    \item \url{https://colah.github.io/posts/2015-08-Understanding-LSTMs/}
\end{itemize}

\vspace{0.3cm}

\textbf{Tutoriais Interativos:}
\begin{itemize}
    \item TensorFlow RNN Tutorial: \url{https://www.tensorflow.org/text/tutorials/text_classification_rnn}
    \item Distill.pub - Visualizações de RNNs
\end{itemize}

\vspace{0.3cm}

\textbf{Vídeos Complementares:}
\begin{itemize}
    \item Stanford CS231n - Lecture 10 (RNNs)
    \item 3Blue1Brown - Neural Networks series
\end{itemize}

\end{frame}



\begin{frame}[c, noframenumbering, plain]
    \frametitle{~}
    \vfill
    \begin{center}
        {\Huge Obrigado!}\vspace{1.5em}\\
        {\Large \highlight{Dúvidas?}}\\
    \end{center}
    \vfill
    % \begin{center}
    %     {\small Próxima aula: 22/10/2025}\\
    %     {\small Tema: Arquiteturas Modernas de CNNs}
    % \end{center}
\end{frame}

\end{document}