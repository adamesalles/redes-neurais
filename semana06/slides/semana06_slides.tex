\documentclass[xcolor=dvipsnames,t,aspectratio=169]{beamer}

\usecolortheme{rose}
\usecolortheme{dolphin}
\usetheme{Boadilla}

% Encoding packages MUST come first
\usepackage[utf8]{inputenc}
\usepackage[brazil]{babel}
\usepackage[T1]{fontenc}

\usepackage{amsfonts}
\usepackage{amsmath}
\usepackage{amssymb}
% \usepackage{latexsym}
% \usepackage{graphicx}
% \usepackage{listings} 
\usepackage{xcolor}
\usepackage{amsthm}
\usepackage{url}
\usepackage{textpos}
% \usepackage{amssymb}
% \usepackage{caption}
% \usepackage{subcaption}
% \usepackage{multicol}
% \usepackage{mathrsfs}
\usepackage{float}
\setbeamersize{text margin left=1cm,text margin right=1cm,} 
\usepackage{subcaption}
\usepackage{dsfont}
\usepackage{physics}
\usepackage{mathpazo}
% \usepackage{tikz}
\usepackage{lipsum}
\usepackage{minted}
\usepackage[minted, most]{tcolorbox}
\tcbuselibrary{theorems}
\setbeamertemplate{enumerate items}[default]
\usefonttheme{professionalfonts}

\setlength{\parindent}{0pt} % zero indent

\setbeamertemplate{itemize items}[circle]
\setbeamertemplate{itemize subitem}{$\blacktriangleright$}
\definecolor{nes_dark_purple}{HTML}{4E006B}
\definecolor{nes_dark_orange}{HTML}{EE4F09}
\setbeamercolor{normal text}{fg=nes_dark_purple}\usebeamercolor*{normal text}
\setbeamercolor{math text}{fg=nes_dark_purple}\usebeamercolor[fg]{math text}


\setbeamercolor{title}{fg=nes_dark_purple}
\setbeamercolor{frametitle}{fg=nes_dark_purple}
\setbeamercolor{structure}{fg=nes_dark_purple}
\setbeamercolor{author}{fg=nes_dark_orange}
\setbeamercolor{footline}{fg=nes_dark_purple} 

\setbeamertemplate{footline}[frame number]
\setbeamertemplate{navigation symbols}{}

\definecolor{codegreen}{rgb}{0,0.6,0.2}
\definecolor{codegray}{rgb}{0.5,0.5,0.5}
\definecolor{codepurple}{rgb}{0.58,0,0.82}
\definecolor{backcolour}{rgb}{0.95,0.95,0.92}

\lstdefinestyle{mystyle}{
    backgroundcolor=\color{backcolour},   
    commentstyle=\color{codegreen},
    keywordstyle=\color{magenta},
    numberstyle=\tiny\color{codegray},
    stringstyle=\color{codepurple},
    basicstyle=\ttfamily\footnotesize,
    breakatwhitespace=false,  
    breaklines=true,                 
    captionpos=b,                    
    keepspaces=true,                 
    numbers=left,                    
    numbersep=5pt,                  
    showspaces=false,                
    showstringspaces=false,
    showtabs=false,                  
    tabsize=2
}
\input{../../templates/slides/commands}

\usepackage{tikz}
\usetikzlibrary{arrows.meta, shapes, positioning, calc}
\usepackage{algorithmic}
\usepackage{graphicx}

\titlegraphic{
    \includegraphics[scale = 0.5]{../../templates/slides/logo}
}

\logo{
\begin{tikzpicture}[overlay,remember picture]
\node[below left = 0.2cm] at (current page.30) {
    \includegraphics[width=0.1\textwidth]{../../templates/slides/logo}};
\end{tikzpicture}
}

\newcommand{\highlight}[1]{{\color{nes_dark_orange} #1}}

\title{Introdução às Redes Neurais Convolucionais (CNNs)} 

\author{
    Eduardo Adame
}

\date{{\color{nes_dark_purple}  \textbf{Redes Neurais}\\[0.5em] 24 de setembro de 2025 }}

\begin{document}

\frame[plain]{\titlepage}
\setcounter{framenumber}{0}

\section{Motivação e Contexto}

\begin{frame}[c]{De MLPs para CNNs: Por que Mudar?}
    \begin{columns}[c]
        \begin{column}{0.5\textwidth}
            \textbf{Limitações das MLPs para Imagens:}
            \begin{itemize}
                \item \highlight{Explosão de parâmetros}
                \begin{itemize}
                    \item Imagem 200x200 RGB = 120.000 entradas
                    \item Uma camada: 14.4 bilhões de pesos!
                \end{itemize}
                \item \highlight{Perda de estrutura espacial}
                \begin{itemize}
                    \item MLP trata pixels como lista
                    \item Ignora proximidade dos pixels
                \end{itemize}
                \item \highlight{Sem invariância à translação}
                \begin{itemize}
                    \item Objeto em posições diferentes
                    \item Requer reaprender padrões
                \end{itemize}
            \end{itemize}
        \end{column}
        \begin{column}{0.5\textwidth}
            \begin{tikzpicture}[scale=0.8]
                % MLP representation
                \node at (0,3.5) {\textbf{MLP: Visão "Achatada"}};
                % Input pixels flattened
                \foreach \i in {0,...,4} {
                    \draw[fill=blue!20] (0,2.5-\i*0.3) rectangle (0.3,2.2-\i*0.3);
                }
                \node at (0.15,0.5) {$\vdots$};
                
                % Fully connected to hidden layer
                \foreach \i in {0,...,2} {
                    \node[circle, draw, fill=orange!30] (h\i) at (2,2-\i*0.8) {};
                    \foreach \j in {0,...,4} {
                        \draw[gray, opacity=0.3] (0.3,2.35-\j*0.3) -- (h\i);
                    }
                }
            \end{tikzpicture}
        \end{column}
    \end{columns}
\end{frame}

\begin{frame}[c]{A Natureza Hierárquica da Visão}
    \begin{columns}[c]
        \begin{column}{0.5\textwidth}
            \textbf{Como construímos representações visuais:}
            \begin{itemize}
                \item \highlight{Nível 1}: Detectores de bordas
                \item \highlight{Nível 2}: Formas simples
                \item \highlight{Nível 3}: Partes de objetos
                \item \highlight{Nível 4}: Objetos completos
            \end{itemize}
            
            \vspace{0.5cm}
            \textbf{Exemplo - Reconhecendo um Gato:}
            \begin{itemize}
                \item Bordas $\to$ Círculos
                \item Círculos $\to$ Olhos
                \item Olhos + Textura $\to$ Face de gato
            \end{itemize}
        \end{column}
        \begin{column}{0.5\textwidth}
            \begin{tikzpicture}[scale=0.9]
                % Hierarchical feature visualization
                % Level 1: Edges
                \node at (0,3.5) {\small \textbf{Bordas}};
                \draw[thick] (-0.5,3) -- (0.5,3);
                \draw[thick] (0,2.5) -- (0,3);
                \draw[thick] (-0.3,2.7) -- (0.3,3.3);
                
                % Arrow
                \draw[->, ultra thick, gray] (1,2.8) -- (2,2.8);
                
                % Level 2: Shapes
                \node at (3,3.5) {\small \textbf{Formas}};
                \draw[thick] (3,3) circle (0.3);
                \draw[thick] (2.5,2.5) rectangle (3.5,3);
                
                % Arrow
                \draw[->, ultra thick, gray] (4,2.8) -- (5,2.8);
                
                % Level 3: Parts
                \node at (6,3.5) {\small \textbf{Partes}};
                % Simple eye representation
                \draw[thick] (6,2.75) circle (0.4);
                \draw[fill=black] (6,2.75) circle (0.15);
                
                % Arrow down
                \draw[->, ultra thick, gray] (6,2.2) -- (6,1.2);
                
                % Level 4: Complete object
                \node at (6,0.8) {\small \textbf{Objeto Completo}};
                % Simple cat face
                \draw[thick] (6,0) circle (0.5);
                \draw[fill=black] (5.8,0.1) circle (0.08);
                \draw[fill=black] (6.2,0.1) circle (0.08);
                \draw (5.8,-0.2) .. controls (6,-0.1) .. (6.2,-0.2);
            \end{tikzpicture}
        \end{column}
    \end{columns}
\end{frame}

\begin{frame}[c]{Conceito de Convolução: Filtros Locais}
    \begin{columns}[c]
        \begin{column}{0.55\textwidth}
            \textbf{O que é uma Convolução?}
            \begin{itemize}
                \item \highlight{Operação matemática} entre imagem e filtro
                \item Filtro (kernel) "desliza" sobre a imagem
                \item Detecta padrões locais específicos
            \end{itemize}
            
            \vspace{0.3cm}
            \textbf{Vantagens sobre MLP:}
            \begin{itemize}
                \item \highlight{Compartilhamento de pesos}: mesmo filtro em toda imagem
                \item \highlight{Conectividade local}: neurônio vê apenas região
                \item \highlight{Invariância à translação}: detecta padrão em qualquer posição
            \end{itemize}
        \end{column}
        \begin{column}{0.45\textwidth}
            \textbf{Exemplo: Detector de Bordas Verticais}
            \begin{tikzpicture}[scale=0.7]
                % Kernel
                \node at (0,2.5) {\small \textbf{Filtro}};
                \draw[fill=blue!20] (-0.5,1.5) grid (1,0);
                \node at (-0.25,1.25) {-1};
                \node at (0.25,1.25) {1};
                \node at (0.75,1.25) {-1};
                \node at (-0.25,0.75) {-1};
                \node at (0.25,0.75) {1};
                \node at (0.75,0.75) {-1};
                \node at (-0.25,0.25) {-1};
                \node at (0.25,0.25) {1};
                \node at (0.75,0.25) {-1};
                
                % Arrow
                \draw[->, thick] (1.5,0.75) -- (2.5,0.75);
                
                % Result visualization
                \node at (4,2.5) {\small \textbf{Resposta}};
                \draw[fill=white] (3,-0.5) rectangle (5,2);
                \draw[line width=3pt, gray] (4,-0.5) -- (4,2);
                \node at (4,-1) {\small Borda detectada!};
            \end{tikzpicture}
        \end{column}
    \end{columns}
\end{frame}

\section{Componentes Fundamentais}

\begin{frame}[c]{Camadas Convolucionais: O Coração das CNNs}
    \begin{columns}[c]
        \begin{column}{0.5\textwidth}
            \textbf{Operação de Convolução:}
            
            Para cada posição $(i,j)$:
            $$y_{i,j} = \sum_{m,n} x_{i+m,j+n} \cdot w_{m,n} + b$$
            
            \vspace{0.3cm}
            \textbf{Hiperparâmetros Importantes:}
            \begin{itemize}
                \item \highlight{Tamanho do filtro}: 3$\times$3, 5$\times$5, 7$\times$7
                \item \highlight{Número de filtros}: 32, 64, 128...
                \item \highlight{Stride}: passo do filtro (1, 2...)
                \item \highlight{Padding}: zeros nas bordas
            \end{itemize}
        \end{column}
        \begin{column}{0.5\textwidth}
            \begin{tikzpicture}[scale=0.6]
                % Input
                \node at (0,4) {\textbf{Entrada 5$\times$5}};
                \draw[step=0.5] (0,0) grid (2.5,2.5);
                
                % Highlight kernel area
                \draw[ultra thick, red] (0,1.5) rectangle (1.5,3);
                
                % Kernel
                \node at (4,4) {\textbf{Filtro 3$\times$3}};
                \draw[step=0.5, fill=orange!20] (3.49, 1.49) grid (5,3);
                
                % Multiplication symbol
                \node at (5.75,2.25) {$\times$};
                
                % Output
                \node at (7.5,4) {\textbf{Saída}};
                \draw[step=0.5] (6.49,0.99) grid (8,2.5);
                \draw[fill=green!40] (6.5,2) rectangle (7,2.5);
                
                % Arrow showing sliding
                \draw[->, ultra thick, blue, dashed] (0.75,0.75) -- (1.25,0.75);
                \node[below] at (1,0.3) {\small desliza};
            \end{tikzpicture}
            
            \vspace{0.3cm}
            \textbf{Múltiplos Filtros = Múltiplos Mapas de Características}
        \end{column}
    \end{columns}
\end{frame}

\begin{frame}[c]{Pooling: Reduzindo Dimensionalidade}
    \begin{columns}[c]
        \begin{column}{0.5\textwidth}
            \textbf{Por que usar Pooling?}
            \begin{itemize}
                \item Reduz tamanho espacial
                \item Diminui número de parâmetros
                \item Adiciona invariância a pequenas translações
                \item Controla overfitting
            \end{itemize}
            
            \vspace{0.5cm}
            \textbf{Tipos Comuns:}
            \begin{itemize}
                \item \highlight{Max Pooling}: valor máximo da região
                \item \highlight{Average Pooling}: média dos valores
                \item Geralmente 2$\times$2 com stride 2
            \end{itemize}
        \end{column}
        \begin{column}{0.5\textwidth}
            \textbf{Exemplo: Max Pooling 2$\times$2}
            \vspace{1cm}

            \begin{tikzpicture}[scale=0.8]
                % Input
                \node at (0.2,2.5) {\small \textbf{Entrada 4$\times$4}};
                \draw[step=0.4] (0,0) grid (1.6,1.6);
                % Fill with sample values
                \node[scale=0.6] at (0.2,1.4) {1};
                \node[scale=0.6] at (0.6,1.4) {3};
                \node[scale=0.6] at (1.0,1.4) {2};
                \node[scale=0.6] at (1.4,1.4) {4};
                
                \node[scale=0.6] at (0.2,1.0) {5};
                \node[scale=0.6] at (0.6,1.0) {6};
                \node[scale=0.6] at (1.0,1.0) {1};
                \node[scale=0.6] at (1.4,1.0) {2};
                
                \node[scale=0.6] at (0.2,0.6) {3};
                \node[scale=0.6] at (0.6,0.6) {2};
                \node[scale=0.6] at (1.0,0.6) {3};
                \node[scale=0.6] at (1.4,0.6) {4};
                
                \node[scale=0.6] at (0.2,0.2) {1};
                \node[scale=0.6] at (0.6,0.2) {0};
                \node[scale=0.6] at (1.0,0.2) {1};
                \node[scale=0.6] at (1.4,0.2) {2};
                
                % Highlight pools
                \draw[ultra thick, red] (0,0.8) rectangle (0.8,1.6);
                \draw[ultra thick, blue] (0.8,0.8) rectangle (1.6,1.6);
                \draw[ultra thick, green] (0,0) rectangle (0.8,0.8);
                \draw[ultra thick, orange] (0.8,0) rectangle (1.6,0.8);
                
                % Arrow
                \draw[->, thick] (2,0.8) -- (2.8,0.8);
                
                % Output
                \node at (3.7,2.5) {\small \textbf{Saída 2$\times$2}};
                \draw[step=0.5] (3.49,0.49) grid (4.5,1.5);
                \node at (3.75,1.25) {6};
                \node at (4.25,1.25) {4};
                \node at (3.75,0.75) {3};
                \node at (4.25,0.75) {4};
            \end{tikzpicture}
            
            \vspace{0.3cm}
            \centering
            \highlight{Redução de 75\% no tamanho!}
        \end{column}
    \end{columns}
\end{frame}

\begin{frame}[c]{Arquitetura Típica de uma CNN}
    \begin{center}
        \begin{tikzpicture}[scale=0.85]
            % Input image
            \node at (-1,0) {\textbf{Entrada}};
            \draw[fill=blue!20] (0,0) rectangle (1,2);
            \draw[fill=blue!20] (0.1,-0.1) rectangle (1.1,1.9);
            \draw[fill=blue!20] (0.2,-0.2) rectangle (1.2,1.8);
            \node at (0.5,-0.6) {\small RGB};
            
            % Conv1
            \draw[->, thick] (1.4,1) -- (1.8,1);
            \node at (2.5,2.5) {\tiny Conv+ReLU};
            \foreach \i in {0,...,3} {
                \draw[fill=green!30] (2+\i*0.15,0.5-\i*0.05) rectangle (2.8+\i*0.15,1.8-\i*0.05);
            }
            \node at (2.5,-0.6) {\small 32 filtros};
            
            % Pool1
            \draw[->, thick] (3.5,1) -- (3.9,1);
            \node at (4.3,2.5) {\tiny MaxPool};
            \foreach \i in {0,...,3} {
                \draw[fill=green!40] (4+\i*0.1,0.7-\i*0.03) rectangle (4.5+\i*0.1,1.5-\i*0.03);
            }
            
            % Conv2
            \draw[->, thick] (5,1) -- (5.4,1);
            \node at (6,2.5) {\tiny Conv+ReLU};
            \foreach \i in {0,...,5} {
                \draw[fill=orange!30] (5.7+\i*0.1,0.8-\i*0.02) rectangle (6.2+\i*0.1,1.4-\i*0.02);
            }
            \node at (6,-0.6) {\small 64 filtros};
            
            % Pool2
            \draw[->, thick] (6.8,1) -- (7.2,1);
            \node at (7.5,2.5) {\tiny MaxPool};
            \foreach \i in {0,...,5} {
                \draw[fill=orange!40] (7.3+\i*0.08,0.9-\i*0.01) rectangle (7.6+\i*0.08,1.3-\i*0.01);
            }
            
            % Flatten
            \draw[->, thick] (8.2,1) -- (8.6,1);
            \draw[fill=red!30] (8.8,0.5) rectangle (9,1.5);
            \node at (8.9,2.5) {\tiny Flatten};
            
            % FC layers
            \draw[->, thick] (9.2,1) -- (9.6,1);
            \foreach \i in {0,...,4} {
                \node[circle, draw, fill=yellow!30] at (10,1.5-\i*0.3) {};
            }
            \node at (10,2.5) {\tiny FC};
            
            % Output
            \draw[->, thick] (10.5,1) -- (10.9,1);
            \foreach \i in {0,...,2} {
                \node[circle, draw, fill=purple!30] at (11.5,1.2-\i*0.3) {};
            }
            \node at (11.5,2.5) {\small Saída};
            \node at (11.5,-0.6) {\small Classes};
        \end{tikzpicture}
    \end{center}
    
    \vspace{0.3cm}
    \begin{center}
        \highlight{Padrão}: CONV $\to$ ReLU $\to$ POOL $\to$ CONV $\to$ ReLU $\to$ POOL $\to$ FC $\to$ Softmax
    \end{center}
\end{frame}

\begin{frame}[c]{Comparação: CNN vs MLP para Imagens}
    \begin{center}
        \begin{tabular}{|l|c|c|}
            \hline
            \textbf{Aspecto} & \textbf{MLP} & \textbf{CNN} \\
            \hline
            \hline
            Preserva estrutura espacial & X & \checkmark \\
            \hline
            Número de parâmetros & \highlight{Muito alto} & \highlight{Reduzido} \\
            \hline
            Compartilhamento de pesos & X & \checkmark \\
            \hline
            Invariância à translação & X & \checkmark \\
            \hline
            Detecta padrões locais & Difícil & Natural \\
            \hline
            Hierarquia de características & X & \checkmark \\
            \hline
            Eficiência computacional & Baixa & Alta \\
            \hline
            Generalização em imagens & Limitada & Excelente \\
            \hline
        \end{tabular}
    \end{center}
    
    \vspace{0.5cm}
    \begin{columns}[c]
        \begin{column}{0.5\textwidth}
            \centering
            \textbf{MLP: Força Bruta}
            \begin{itemize}
                \item Conecta tudo com tudo
                \item Ignora localidade
                \item Precisa de muitos dados
            \end{itemize}
        \end{column}
        \begin{column}{0.5\textwidth}
            \centering
            \textbf{CNN: Inteligência Estrutural}
            \begin{itemize}
                \item Explora estrutura da imagem
                \item Aprende filtros úteis
                \item Generaliza melhor
            \end{itemize}
        \end{column}
    \end{columns}
\end{frame}

\begin{frame}[c]{Exemplo Prático: Reconhecimento de Dígitos}
    \begin{columns}[c]
        \begin{column}{0.5\textwidth}
            \textbf{Dataset MNIST:}
            \begin{itemize}
                \item 60.000 imagens de treino
                \item 10.000 imagens de teste
                \item Imagens 28$\times$28 pixels
                \item 10 classes (dígitos 0-9)
            \end{itemize}
            
            \vspace{0.5cm}
            \textbf{Arquitetura Simples:}
            \begin{itemize}
                \item \highlight{Conv1}: 32 filtros 3$\times$3
                \item \highlight{MaxPool}: 2$\times$2
                \item \highlight{Conv2}: 64 filtros 3$\times$3
                \item \highlight{MaxPool}: 2$\times$2
                \item \highlight{FC}: 128 neurônios
                \item \highlight{Saída}: 10 classes
            \end{itemize}
        \end{column}
        \begin{column}{0.5\textwidth}
            \textbf{Visualização dos Filtros Aprendidos:}
            \begin{tikzpicture}[scale=0.7]
                % First layer filters (edge detectors)
                \node at (0,3.5) {\small \textbf{Camada 1: Detectores de Bordas}};
                \foreach \i in {0,...,3} {
                    \draw[fill=gray!20] (\i*0.8,2.5) rectangle (\i*0.8+0.6,3.1);
                }
                % Simulate edge patterns
                \draw[ultra thick] (0.3,2.8) -- (0.3,2.8);
                \draw[ultra thick] (1.1,2.6) -- (1.1,3);
                \draw[ultra thick] (1.7,2.8) -- (2.1,2.8);
                \draw[ultra thick] (2.5,2.6) -- (2.8,2.9);
                
                % Second layer filters (shape detectors)
                \node at (0,1.5) {\small \textbf{Camada 2: Detectores de Formas}};
                \foreach \i in {0,...,3} {
                    \draw[fill=gray!20] (\i*0.8,0.5) rectangle (\i*0.8+0.6,1.1);
                }
                % Simulate shape patterns
                \draw[thick] (0.3,0.8) circle (0.15);
                \draw[thick] (1,0.8) -- (1.3,0.8) -- (1.15,0.6) -- cycle;
                \draw[thick] (1.8,0.6) rectangle (2.1,0.9);
                \draw[thick] (2.5,0.8) .. controls (2.6,0.9) and (2.7,0.7) .. (2.8,0.8);
                
                \node at (0,-0.5) {\small \textbf{Resultado: 99\% de Acurácia!}};
            \end{tikzpicture}
        \end{column}
    \end{columns}
\end{frame}

\begin{frame}[c, fragile]{Implementação em Keras/TensorFlow}
    \begin{code}[CNN Simples para MNIST]{python}
from tensorflow.keras import Sequential
from tensorflow.keras.layers import Conv2D, MaxPooling2D
from tensorflow.keras.layers import Flatten, Dense, Dropout

# Construindo o modelo CNN
model = Sequential([
    # Primeira camada convolucional
    Conv2D(32, (3,3), activation='relu', 
           input_shape=(28, 28, 1)),
    MaxPooling2D((2,2)),

    # Segunda camada convolucional
    Conv2D(64, (3,3), activation='relu'),
    MaxPooling2D((2,2)),
        \end{code}
\end{frame}


\begin{frame}[c, fragile]{Implementação em Keras/TensorFlow}
    \begin{code}[CNN Simples para MNIST]{python}

    # Camadas densas
    Flatten(),
    Dense(128, activation='relu'),
    Dropout(0.5),
    Dense(10, activation='softmax')
])

model.compile(optimizer='adam',
              loss='categorical_crossentropy',
              metrics=['accuracy'])
    \end{code}
\end{frame}

\begin{frame}[c]{Aplicações Modernas das CNNs}
    \begin{columns}[c]
        \begin{column}{0.5\textwidth}
            \textbf{Visão Computacional:}
            \begin{itemize}
                \item \highlight{Classificação}: ImageNet, CIFAR
                \item \highlight{Detecção}: YOLO, R-CNN
                \item \highlight{Segmentação}: U-Net, Mask R-CNN
                \item \highlight{Reconhecimento facial}: FaceNet
            \end{itemize}
            
            \vspace{0.3cm}
            \textbf{Medicina:}
            \begin{itemize}
                \item Diagnóstico por imagem
                \item Detecção de tumores
                \item Análise de retina
            \end{itemize}
        \end{column}
        \begin{column}{0.5\textwidth}
            \textbf{Outras Aplicações:}
            \begin{itemize}
                \item \highlight{Carros autônomos}: detecção de objetos
                \item \highlight{Arte}: transferência de estilo
                \item \highlight{Agricultura}: análise de culturas
                \item \highlight{Segurança}: vigilância inteligente
            \end{itemize}
            
            \vspace{0.3cm}
            \textbf{Arquiteturas Famosas:}
            \begin{itemize}
                \item LeNet-5 (1998)
                \item AlexNet (2012)
                \item VGG (2014)
                \item ResNet (2015)
                \item EfficientNet (2019)
            \end{itemize}
        \end{column}
    \end{columns}
\end{frame}

\begin{frame}[c]{Resumo: CNNs vs MLPs}
    \begin{center}
        \begin{tikzpicture}[scale=0.9]
            % MLP side
            \node[rectangle, draw, fill=blue!20, minimum width=3cm, minimum height=1cm] at (-4,3.25) {\textbf{MLP}};
            \node[text width=5cm] at (-3,1.5) {
                • Conexão total\\
                • Muitos parâmetros\\
                • Sem estrutura espacial\\
                • Bom para dados tabulares
            };
            
            % CNN side
            \node[rectangle, draw, fill=green!20, minimum width=3cm, minimum height=1cm] at (3,3.25) {\textbf{CNN}};
            \node[text width=5cm] at (4,1.5) {
                • Conexão local\\
                • Compartilha pesos\\
                • Preserva topologia\\
                • Ideal para imagens
            };
            
            % Comparison arrow
            \draw[<->, ultra thick, orange] (-1,2) -- (1,2);
            \node[above] at (0,2) {\highlight{Escolha}};
            
            % Bottom message
            \node[text width=8cm, align=center] at (0,-1.25) {
                \textbf{Lição Principal:}\\
                CNNs são MLPs especializadas que exploram a estrutura espacial dos dados visuais através de\\
                \highlight{convoluções} e \highlight{pooling} para aprender hierarquias de características
            };
        \end{tikzpicture}
    \end{center}
\end{frame}


\begin{frame}[c]{Explorando as CNNs}
    \begin{figure}
        \centering
        \includegraphics[height=.75\textheight]{image.png}
        \caption*{Recomendação: assistir o vídeo do \highlight{3Blue1Brown} (legendas em português).\\ Link: \href{https://www.youtube.com/watch?v=KuXjwB4LzSA}{https://www.youtube.com/watch?v=KuXjwB4LzSA}}
    \end{figure}
\end{frame}

\begin{frame}[c, noframenumbering, plain]
    \frametitle{~}
    \vfill
    \begin{center}
        {\Huge Obrigado!}\vspace{1.5em}\\
        {\Large \highlight{Dúvidas?}}\\
    \end{center}
    \vfill
    \begin{center}
        {\small Próximo: Arquiteturas de CNNs}
    \end{center}
\end{frame}


\end{document}