\documentclass[xcolor=dvipsnames,t,aspectratio=169]{beamer}

\usecolortheme{rose}
\usecolortheme{dolphin}
\usetheme{Boadilla}

% Encoding packages MUST come first
\usepackage[utf8]{inputenc}
\usepackage[brazil]{babel}
\usepackage[T1]{fontenc}

\usepackage{amsfonts}
\usepackage{amsmath}
\usepackage{amssymb}
% \usepackage{latexsym}
% \usepackage{graphicx}
% \usepackage{listings} 
\usepackage{xcolor}
\usepackage{amsthm}
\usepackage{url}
\usepackage{textpos}
% \usepackage{amssymb}
% \usepackage{caption}
% \usepackage{subcaption}
% \usepackage{multicol}
% \usepackage{mathrsfs}
\usepackage{float}
\setbeamersize{text margin left=1cm,text margin right=1cm,} 
\usepackage{subcaption}
\usepackage{dsfont}
\usepackage{physics}
\usepackage{mathpazo}
% \usepackage{tikz}
\usepackage{lipsum}
\usepackage{minted}
\usepackage[minted, most]{tcolorbox}
\tcbuselibrary{theorems}
\setbeamertemplate{enumerate items}[default]
\usefonttheme{professionalfonts}

\setlength{\parindent}{0pt} % zero indent

\setbeamertemplate{itemize items}[circle]
\setbeamertemplate{itemize subitem}{$\blacktriangleright$}
\definecolor{nes_dark_purple}{HTML}{4E006B}
\definecolor{nes_dark_orange}{HTML}{EE4F09}
\setbeamercolor{normal text}{fg=nes_dark_purple}\usebeamercolor*{normal text}
\setbeamercolor{math text}{fg=nes_dark_purple}\usebeamercolor[fg]{math text}


\setbeamercolor{title}{fg=nes_dark_purple}
\setbeamercolor{frametitle}{fg=nes_dark_purple}
\setbeamercolor{structure}{fg=nes_dark_purple}
\setbeamercolor{author}{fg=nes_dark_orange}
\setbeamercolor{footline}{fg=nes_dark_purple} 

\setbeamertemplate{footline}[frame number]
\setbeamertemplate{navigation symbols}{}

\definecolor{codegreen}{rgb}{0,0.6,0.2}
\definecolor{codegray}{rgb}{0.5,0.5,0.5}
\definecolor{codepurple}{rgb}{0.58,0,0.82}
\definecolor{backcolour}{rgb}{0.95,0.95,0.92}

\lstdefinestyle{mystyle}{
    backgroundcolor=\color{backcolour},   
    commentstyle=\color{codegreen},
    keywordstyle=\color{magenta},
    numberstyle=\tiny\color{codegray},
    stringstyle=\color{codepurple},
    basicstyle=\ttfamily\footnotesize,
    breakatwhitespace=false,  
    breaklines=true,                 
    captionpos=b,                    
    keepspaces=true,                 
    numbers=left,                    
    numbersep=5pt,                  
    showspaces=false,                
    showstringspaces=false,
    showtabs=false,                  
    tabsize=2
}
\input{../../templates/slides/commands}

\usepackage{tikz}
\usetikzlibrary{arrows.meta, shapes, positioning, calc}
\usepackage{algorithmic}
\usepackage{graphicx}
\usepackage{amsmath}
\usepackage{listings}

% Configuração para código Python
\lstset{
    language=Python,
    basicstyle=\tiny\ttfamily,
    keywordstyle=\color{blue},
    commentstyle=\color{green!60!black},
    stringstyle=\color{orange},
    showstringspaces=false,
    breaklines=true,
    frame=single,
    numbers=left,
    numberstyle=\tiny\color{gray}
}

\titlegraphic{
    \includegraphics[scale = 0.5]{../../templates/slides/logo}
}

\logo{
\begin{tikzpicture}[overlay,remember picture]
\node[below left = 0.2cm] at (current page.30) {
    \includegraphics[width=0.1\textwidth]{../../templates/slides/logo}};
\end{tikzpicture}
}

\newcommand{\highlight}[1]{{\color{nes_dark_orange} #1}}

\title{Redes Neurais Convolucionais (CNNs) e Transfer Learning} 

\author{
    Eduardo Adame
}

\date{{\color{nes_dark_purple}  \textbf{Redes Neurais}\\[0.5em] 15 de outubro de 2025 }}

\begin{document}

\frame[plain]{\titlepage}
\setcounter{framenumber}{0}

\section{Revisão e Motivação}

\begin{frame}[c]{Agenda da Aula}
    \begin{columns}[c]
        \begin{column}{0.5\textwidth}
            \textbf{Parte 1: CNNs}
            \begin{itemize}
                \item Revisão: Limitações das MLPs
                \item Operação de Convolução
                \item Arquitetura LeNet-5
                \item Cálculo de dimensões
                \item Pooling e seus tipos
            \end{itemize}
        \end{column}
        \begin{column}{0.5\textwidth}
            \textbf{Parte 2: Transfer Learning}
            \begin{itemize}
                \item Motivação e conceitos
                \item Fine-tuning estratégico
                \item Quando usar cada abordagem
                \item Implementação prática
                \item Aplicações modernas
            \end{itemize}
        \end{column}
    \end{columns}
\end{frame}

\begin{frame}[c]{Revisão: Por que CNNs?}
    \begin{columns}[c]
        \begin{column}{0.5\textwidth}
            \textbf{Problemas das MLPs com Imagens:}
            \begin{itemize}
                \item \highlight{Explosão paramétrica}
                \begin{itemize}
                    \item Imagem 224$\times$224 RGB = 150.528 entradas
                    \item Uma camada oculta (1000 neurônios): 
                    \item \alert{150 milhões de parâmetros!}
                \end{itemize}
                \item \highlight{Perda de localidade espacial}
                \item \highlight{Sem invariância translacional}
            \end{itemize}
            
            \vspace{0.3cm}
            \textbf{Solução: Convolução}
            \begin{itemize}
                \item Compartilhamento de pesos
                \item Conectividade local
                \item Hierarquia de características
            \end{itemize}
        \end{column}
        \begin{column}{0.5\textwidth}
            \begin{tikzpicture}[scale=0.7]
                % MLP: todos conectados
                \node at (0,4.5) {\textbf{MLP: Conexão Total}};
                \foreach \i in {0,...,3} {
                    \foreach \j in {0,...,3} {
                        \draw[fill=blue!20] (\i*0.3,2.5+\j*0.3) rectangle (\i*0.3+0.25,2.75+\j*0.3);
                    }
                }
                \foreach \k in {0,...,2} {
                    \node[circle, draw, fill=orange!30] (h\k) at (2.5,2.8+\k*0.5) {};
                    \foreach \i in {0,...,3} {
                        \foreach \j in {0,...,3} {
                            \draw[gray, opacity=0.2] (\i*0.3+0.125,2.625+\j*0.3) -- (h\k);
                        }
                    }
                }
                
                % CNN: conexão local
                \node at (0,1) {\textbf{CNN: Conexão Local}};
                \foreach \i in {0,...,3} {
                    \foreach \j in {0,...,3} {
                        \draw[fill=blue!20] (\i*0.3,-1+\j*0.3) rectangle (\i*0.3+0.25,-0.75+\j*0.3);
                    }
                }
                % Destacar região 3x3
                \draw[ultra thick, red] (0,-1) rectangle (0.85,-0.15);
                \node[circle, draw, fill=green!30] (c1) at (2.5,-0.5) {};
                \draw[thick, ->] (0.425,-0.575) -- (c1);
            \end{tikzpicture}
        \end{column}
    \end{columns}
\end{frame}

\section{Arquitetura LeNet-5}

\begin{frame}[c]{LeNet-5: Pioneira das CNNs}
    \begin{columns}[c]
        \begin{column}{0.5\textwidth}
            \textbf{Criada por Yann LeCun (1998)}
            \begin{itemize}
                \item Dataset MNIST: dígitos manuscritos
                \item 60.000 imagens treino / 10.000 teste
                \item Imagens 28$\times$28 em escala de cinza
                \item Padding para 32$\times$32
            \end{itemize}
            
            \vspace{0.5cm}
            \textbf{Inovação Principal:}
            \begin{itemize}
                \item Usar \highlight{convoluções} para aprender características
                \item Redução progressiva de dimensões
                \item Extração hierárquica de features
            \end{itemize}
        \end{column}
        \begin{column}{0.5\textwidth}
            \begin{center}
                \includegraphics[width=\textwidth]{lenet_architecture.png}
                
                \vspace{0.3cm}
                {\small Arquitetura: Conv $\to$ Pool $\to$ Conv $\to$ Pool $\to$ FC $\to$ FC $\to$ Saída}
            \end{center}
        \end{column}
    \end{columns}
\end{frame}

\begin{frame}[c]{Operação de Convolução: Matemática}
    \begin{columns}[]
        \begin{column}{0.55\textwidth}
            \textbf{Definição Matemática:}
            
            Para entrada $X$ e kernel $W$:
            \begin{equation}
                Y_{i,j} = \sum_{m=0}^{k-1} \sum_{n=0}^{k-1} X_{i+m,j+n} \cdot W_{m,n} + b
            \end{equation}
            
            onde:
            \begin{itemize}
                \item $Y_{i,j}$: pixel da saída na posição $(i,j)$
                \item $k$: tamanho do kernel (ex: 3$\times$3, 5$\times$5)
                \item $b$: bias
            \end{itemize}
            
            % \vspace{0.3cm}
            
        \end{column}
        \begin{column}{0.45\textwidth}
            \textbf{Cálculo de Dimensão de Saída:}
            \begin{equation}
                \text{Saída} = \left\lfloor \frac{N - K + 2P}{S} \right\rfloor + 1
            \end{equation}
            \begin{itemize}
                \item $N$: dimensão entrada
                \item $K$: dimensão kernel
                \item $P$: padding
                \item $S$: stride
            \end{itemize}
            
        \end{column}
    \end{columns}
\end{frame}

\begin{frame}[c]{Operação de Convolução: Matemática}
    \textbf{Exemplo Numérico:}
            % \begin{tikzpicture}[scale=0.5]
            %     % Entrada 5x5
            %     \node at (0,5.5) {\small Entrada 5$\times$5};
            %     \foreach \i in {0,...,4} {
            %         \foreach \j in {0,...,4} {
            %             \node[scale=0.6] at (\i*0.8,4-\j*0.8) {1};
            %         }
            %     }
            %     \draw[step=0.8] (-0.4,-0.4) grid (3.6,4.4);
                
            %     % Kernel 3x3
            %     \node at (6,5.5) {\small Kernel 3$\times$3};
            %     \draw[fill=orange!20] (5.2,3.2) grid (7.2,4.4);
            %     \node[scale=0.6] at (5.6,4) {1}; \node[scale=0.6] at (6.4,4) {0}; \node[scale=0.6] at (7.2,4) {-1};
            %     \node[scale=0.6] at (5.6,3.2) {1}; \node[scale=0.6] at (6.4,3.2) {0}; \node[scale=0.6] at (7.2,3.2) {-1};
            %     \node[scale=0.6] at (5.6,2.4) {1}; \node[scale=0.6] at (6.4,2.4) {0}; \node[scale=0.6] at (7.2,2.4) {-1};
                
            %     % Highlight região
            %     \draw[ultra thick, red] (-0.4,2.8) rectangle (2,4.4);
                
            %     % Seta
            %     \draw[->, ultra thick] (3,1.5) -- (5,1.5);
                
            %     % Saída
            %     \node at (7,1) {\small Saída 3$\times$3};
            %     \draw[fill=green!20] (5.6,-0.4) grid (7.6,1.6);
            % \end{tikzpicture}
            
            \vspace{0.5cm}
            Entrada 5x5, kernel 3x3, com stride=1 e padding=0:\pause{}
            $$\text{Saída} = \frac{5-3+0}{1} + 1 = 3$$
\end{frame}

\begin{frame}[c]{LeNet-5: Análise Camada por Camada}
    \begin{center}
        \begin{tikzpicture}[scale=0.9]
            % C1
            \node[draw, fill=blue!20] (c1) at (-1,0) {
                \begin{tabular}{c}
                    \textbf{C1: Conv} \\
                    6@28$\times$28 \\
                    \tiny 156 pesos
                \end{tabular}
            };
            
            % S2
            \node[draw, fill=green!20] (s2) at (2,0) {
                \begin{tabular}{c}
                    \textbf{S2: Pool} \\
                    6@14$\times$14 \\
                    \tiny 0 pesos
                \end{tabular}
            };
            
            % C3
            \node[draw, fill=blue!20] (c3) at (5,0) {
                \begin{tabular}{c}
                    \textbf{C3: Conv} \\
                    16@10$\times$10 \\
                    \tiny 2.416 pesos
                \end{tabular}
            };
            
            % S4
            \node[draw, fill=green!20] (s4) at (8,0) {
                \begin{tabular}{c}
                    \textbf{S4: Pool} \\
                    16@5$\times$5 \\
                    \tiny 0 pesos
                \end{tabular}
            };
            
            % C5
            \node[draw, fill=orange!20] (c5) at (10.75,0) {
                \begin{tabular}{c}
                    \textbf{C5: FC} \\
                    120 \\
                    \tiny 48.120 pesos
                \end{tabular}
            };
            
            % Arrows
            \draw[->, thick] (-2.5,0) node[left] {\small 32$\times$32$\times$1} -- (c1);
            \draw[->, thick] (c1) -- (s2);
            \draw[->, thick] (s2) -- (c3);
            \draw[->, thick] (c3) -- (s4);
            \draw[->, thick] (s4) -- (c5);
            
            % Annotations
            \node[above=.5cm] at (0.75,0.35) {\tiny 5$\times$5, stride=1};
            \node[above=.5cm] at (3.75,0.35) {\tiny 2$\times$2, stride=2};
            \node[above=.5cm] at (6.75,0.35) {\tiny 5$\times$5, stride=1};
            \node[above=.5cm] at (9.75,0.35) {\tiny 2$\times$2, stride=2};
        \end{tikzpicture}
    \end{center}
    
    \vspace{0.5cm}
    
    \begin{columns}[c]
        \begin{column}{0.5\textwidth}
            \textbf{Cálculo de Pesos - C1:}
            \begin{itemize}
                \item Kernel: 5$\times$5$\times$1 = 25 pesos
                \item Bias: 1 peso
                \item Total por filtro: 26
                \item \highlight{6 filtros: 6$\times$26 = 156}
            \end{itemize}
        \end{column}
        \begin{column}{0.5\textwidth}
            \textbf{Cálculo de Pesos - C3:}
            \begin{itemize}
                \item Kernel: 5$\times$5$\times$\alert{6} = 150 pesos
                \item Bias: 1 peso
                \item Total por filtro: 151
                \item \highlight{16 filtros: 16$\times$151 = 2.416}
            \end{itemize}
        \end{column}
    \end{columns}
\end{frame}

\begin{frame}[c]{Pooling: Redução de Dimensionalidade}
    \begin{columns}[c]
        \begin{column}{0.5\textwidth}
            \textbf{Objetivos do Pooling:}
            \begin{itemize}
                \item Reduzir dimensões espaciais
                \item Diminuir parâmetros/computação
                \item Adicionar invariância local
                \item Controlar overfitting
            \end{itemize}
            
            \vspace{0.5cm}
            \textbf{Tipos Principais:}
            \begin{itemize}
                \item \highlight{Max Pooling}: $\max(\text{região})$
                \item \highlight{Average Pooling}: $\text{média}(\text{região})$
                \item Global Average Pooling (GAP)
                \item Stochastic Pooling
            \end{itemize}
            
            \textbf{Observação:} Pooling não tem pesos treináveis!
        \end{column}
        \begin{column}{0.5\textwidth}
            \textbf{Exemplo: Max Pooling 2$\times$2, stride=2}
            
            \begin{tikzpicture}[scale=0.8]
                \node at (0,3) {\small Entrada 4$\times$4};
                \draw[step=0.5] (0,0) grid (2,2);
                % Values
                \node[scale=0.5] at (0.25,1.75) {12};
                \node[scale=0.5] at (0.75,1.75) {20};
                \node[scale=0.5] at (1.25,1.75) {30};
                \node[scale=0.5] at (1.75,1.75) {0};
                
                \node[scale=0.5] at (0.25,1.25) {8};
                \node[scale=0.5] at (0.75,1.25) {12};
                \node[scale=0.5] at (1.25,1.25) {2};
                \node[scale=0.5] at (1.75,1.25) {0};
                
                \node[scale=0.5] at (0.25,0.75) {34};
                \node[scale=0.5] at (0.75,0.75) {70};
                \node[scale=0.5] at (1.25,0.75) {37};
                \node[scale=0.5] at (1.75,0.75) {4};
                
                \node[scale=0.5] at (0.25,0.25) {112};
                \node[scale=0.5] at (0.75,0.25) {100};
                \node[scale=0.5] at (1.25,0.25) {25};
                \node[scale=0.5] at (1.75,0.25) {12};
                
                % Highlight regions
                \draw[ultra thick, red] (0,1) rectangle (1,2);
                \draw[ultra thick, blue] (1,1) rectangle (2,2);
                \draw[ultra thick, green] (0,0) rectangle (1,1);
                \draw[ultra thick, orange] (1,0) rectangle (2,1);
                
                % Arrow
                \draw[->, ultra thick] (2.5,1) -- (3.5,1);
                
                % Output
                \node at (4.5,3) {\small Saída 2$\times$2};
                \draw[step=0.5] (4,0.5) grid (5,1.5);
                \node at (4.25,1.25) {\color{red}20};
                \node at (4.75,1.25) {\color{blue}30};
                \node at (4.25,0.75) {\color{green}112};
                \node at (4.75,0.75) {\color{orange}37};
            \end{tikzpicture}
            
            \vspace{0.3cm}
            \highlight{Redução: 75\% dos dados!}
            
        \end{column}
    \end{columns}
\end{frame}

\begin{frame}[c]{Contagem Total de Parâmetros}
    \begin{center}
        \begin{tabular}{|l|l|c|c|}
            \hline
            \textbf{Camada} & \textbf{Cálculo} & \textbf{Pesos} & \textbf{Acumulado} \\
            \hline
            \hline
            Conv1 (C1) & $1 \times 6 \times 5 \times 5 + 6$ & 156 & 156 \\
            \hline
            Pool1 (S2) & Sem parâmetros & 0 & 156 \\
            \hline
            Conv2 (C3) & $6 \times 16 \times 5 \times 5 + 16$ & 2.416 & 2.572 \\
            \hline
            Pool2 (S4) & Sem parâmetros & 0 & 2.572 \\
            \hline
            FC1 (C5) & $400 \times 120 + 120$ & 48.120 & 50.692 \\
            \hline
            FC2 (F6) & $120 \times 84 + 84$ & 10.164 & 60.856 \\
            \hline
            Output & $84 \times 10 + 10$ & 850 & \highlight{61.706} \\
            \hline
        \end{tabular}
    \end{center}
    
    
    \begin{attention}[Comparação com MLP]
        \begin{itemize}
            \item MLP equivalente (32$\times$32 $\to$ 120 $\to$ 84 $\to$ 10): \alert{$\approx$ 130.000 parâmetros}
            \item LeNet-5: \highlight{61.706 parâmetros}
            \item \textbf{Redução de 53\%} com melhor performance!
        \end{itemize}
    \end{attention}
\end{frame}

\begin{frame}[c, fragile]{Implementação em PyTorch}
\begin{code}[]{python}
import torch
import torch.nn as nn
import torch.nn.functional as F

class LeNet5(nn.Module):
    def __init__(self):
        super(LeNet5, self).__init__()
        # Camadas convolucionais
        self.conv1 = nn.Conv2d(1, 6, kernel_size=5, stride=1, padding=2)
        self.pool = nn.MaxPool2d(kernel_size=2, stride=2)
        self.conv2 = nn.Conv2d(6, 16, kernel_size=5, stride=1)
        
        # Camadas totalmente conectadas
        self.fc1 = nn.Linear(16 * 5 * 5, 120)
        self.fc2 = nn.Linear(120, 84)
        self.fc3 = nn.Linear(84, 10)
\end{code}
\end{frame}

\begin{frame}[c, fragile]{Implementação em PyTorch}
\begin{code}[]{python}
    def forward(self, x):
        # C1: 32x32x1 -> 28x28x6
        x = self.pool(F.relu(self.conv1(x)))  # -> 14x14x6
        # C3: 14x14x6 -> 10x10x16
        x = self.pool(F.relu(self.conv2(x)))  # -> 5x5x16
        # Flatten
        x = x.view(-1, 16 * 5 * 5)
        # Fully connected
        x = F.relu(self.fc1(x))
        x = F.relu(self.fc2(x))
        x = self.fc3(x)
        return x
\end{code}
\end{frame}

\begin{frame}[c, fragile]{Treinamento no MNIST}
\begin{code}[]{python}
import torchvision
import torchvision.transforms as transforms
from torch.optim import Adam

# Preparar dados
transform = transforms.Compose([
    transforms.Pad(2),  # 28x28 -> 32x32
    transforms.ToTensor(),
    transforms.Normalize((0.1307,), (0.3081,))
])

trainset = torchvision.datasets.MNIST(root='./data', 
                                      train=True,
                                      download=True, 
                                      transform=transform)
trainloader = torch.utils.data.DataLoader(trainset, 
                                          batch_size=64,
                                          shuffle=True)
\end{code}
\end{frame}


\begin{frame}[c, fragile]{Treinamento no MNIST}
\begin{code}[]{python}
# Modelo, loss e otimizador
model = LeNet5()
criterion = nn.CrossEntropyLoss()
optimizer = Adam(model.parameters(), lr=0.001)

# Loop de treinamento
for epoch in range(10):
    for i, (images, labels) in enumerate(trainloader):
        optimizer.zero_grad()
        outputs = model(images)
        loss = criterion(outputs, labels)
        loss.backward()
        optimizer.step()
\end{code}
\end{frame}
\section{Transfer Learning}

\begin{frame}[c]{Motivação para Transfer Learning}
    \begin{columns}[c]
        \begin{column}{0.5\textwidth}
            \textbf{Por que Transfer Learning?}
            \begin{itemize}
                \item \highlight{Camadas iniciais} aprendem features genéricas
                \begin{itemize}
                    \item Bordas, texturas, formas
                    \item Úteis para várias tarefas
                \end{itemize}
                \item \highlight{Camadas finais} são específicas da tarefa
                \item Treinar do zero é caro:
                \begin{itemize}
                    \item ImageNet: 1.2M imagens
                    \item Semanas de GPU
                    \item Ajuste de hiperparâmetros
                \end{itemize}
            \end{itemize}
            
        \end{column}
        \begin{column}{0.5\textwidth}
            \textbf{Vanishing Gradient:}
            $$\frac{\partial L}{\partial w_1} = \frac{\partial L}{\partial a_n} \prod_{i=2}^{n} \frac{\partial a_i}{\partial a_{i-1}}$$
            
            Gradientes diminuem exponencialmente!
            \begin{tikzpicture}[scale=0.8]
                % Feature hierarchy
                \node at (2,5) {\textbf{Hierarquia de Features}};
                
                % Layer 1
                \draw[fill=nes_dark_orange!20] (0,4) rectangle (4,3.5);
                \node at (2,3.75) {Camada 1: Bordas};
                
                % Layer 2
                \draw[fill=nes_dark_orange!30] (0,3) rectangle (4,2.5);
                \node at (2,2.75) {Camada 2: Texturas};
                
                % Layer 3
                \draw[fill=nes_dark_orange!40] (0,2) rectangle (4,1.5);
                \node at (2,1.75) {Camada 3: Partes};
                
                % Layer 4
                \draw[fill=nes_dark_purple!30] (0,1) rectangle (4,0.5);
                \node at (2,0.75) {Camada 4: Objetos};
                
                % Arrow generic
                \draw[<->, ultra thick, nes_dark_orange] (4.5,4) -- (4.5,2);
                \node[right] at (4.5,3) {\small \color{nes_dark_orange}Genérico};
                
                % Arrow specific
                \draw[<->, ultra thick, nes_dark_purple] (4.5,2) -- (4.5,0.5);
                \node[right] at (4.5,1.25) {\small \color{nes_dark_purple}Específico};
            \end{tikzpicture}
        \end{column}
    \end{columns}
\end{frame}

\begin{frame}[c]{Estratégias de Transfer Learning}
    \begin{center}
        \begin{tikzpicture}[scale=0.9]
            % Pre-trained network
            \node at (0,4) {\textbf{Rede Pré-treinada (ex: ResNet-50)}};
            
            % Convolutional layers
            \draw[fill=blue!20] (-3,2.5) rectangle (2,3);
            \node at (-.5,2.75) {Camadas Convolucionais};
            
            % FC layers
            \draw[fill=orange!20] (2.5,2.5) rectangle (4.5,3);
            \node at (3.5,2.75) {FC};
            
            % Original output
            \draw[fill=red!20] (5,2.5) rectangle (6,3);
            \node at (5.5,2.75) {1000};
            
            % Strategies
            \node at (-3,1.5) {\textbf{Estratégia 1:}};
            \node at (-3,1) {\small Feature Extractor};
            \draw[fill=blue!20, opacity=0.5] (-2,0.5) rectangle (2,0);
            \draw[dashed] (-2,0.5) rectangle (2,0);
            \node at (0,0.25) {\small \color{gray}Congelado};
            \draw[fill=green!30] (2.5,0.5) rectangle (4.5,0);
            \node at (3.5,0.25) {Nova FC};
            
            \node at (-3,-0.5) {\textbf{Estratégia 2:}};
            \node at (-3,-1) {\small Fine-tuning};
            \draw[fill=blue!30] (-2,-1.5) rectangle (2,-2);
            \node at (0,-1.75) {\small Ajustável};
            \draw[fill=green!30] (2.5,-1.5) rectangle (4.5,-2);
            \node at (3.5,-1.75) {Nova FC};
            
            \node at (-3,-2.5) {\textbf{Estratégia 3:}};
            \node at (-3,-3) {\small Full training};
            \draw[fill=blue!40] (-2,-3.5) rectangle (2,-4);
            \node at (0,-3.75) {\small Treinando tudo};
            \draw[fill=green!30] (2.5,-3.5) rectangle (4.5,-4);
            \node at (3.5,-3.75) {Nova FC};
        \end{tikzpicture}
    \end{center}
    
    \vspace{0.3cm}
    \textbf{Exemplos Práticos:}
    \begin{itemize}
        \item \highlight{Q1}: ImageNet $\rightarrow$ Cães vs Gatos (100 imagens)
        \item \highlight{Q3}: ImageNet $\rightarrow$ CIFAR-100 (50k imagens)
        \item \highlight{Q4}: ImageNet $\rightarrow$ Imagens médicas (100k raios-X)
    \end{itemize}
\end{frame}

\begin{frame}[c]{Princípios do Fine-tuning}
    \begin{columns}[c]
        \begin{column}{0.55\textwidth}
            \textbf{1. Taxa de Aprendizado Diferenciada}
            \begin{itemize}
                \item Camadas iniciais: $lr = 10^{-5}$
                \item Camadas finais: $lr = 10^{-3}$
                \item Evita ``catastrophic forgetting''
            \end{itemize}
            
            \vspace{0.3cm}
            \textbf{2. Descongelamento Gradual}
            \begin{itemize}
                \item Época 1-5: só treina FC
                \item Época 6-10: descongela últimas convs
                \item Época 11+: treina tudo
            \end{itemize}
            
            \vspace{0.3cm}
            \textbf{3. Regularização}
            \begin{itemize}
                \item Dropout mais agressivo (0.5 $\to$ 0.7)
                \item Data augmentation pesada
                \item Early stopping
            \end{itemize}
        \end{column}
        \begin{column}{0.45\textwidth}
            \textbf{Fluxo de Fine-tuning:}
            \begin{tikzpicture}[scale=0.7]
                % Step 1
                \node[draw, fill=yellow!20] (s1) at (0,4) {1. Carregar modelo};
                
                % Step 2
                \node[draw, fill=yellow!20] (s2) at (0,2) {2. Substituir última camada};
                
                % Step 3
                \node[draw, fill=yellow!20] (s3) at (0,0) {3. Congelar conv layers};
                
                % Step 4
                \node[draw, fill=green!20] (s4) at (0,-2) {4. Treinar FC};
                
                % Step 5
                \node[draw, fill=green!20] (s5) at (0,-4) {5. Descongelar gradual};
                
                % Step 6
                \node[draw, fill=blue!20] (s6) at (0,-6) {6. Fine-tune completo};
                
                % Arrows
                \foreach \i in {1,...,5} {
                    \pgfmathtruncatemacro{\j}{\i+1}
                    \draw[->, thick] (s\i) -- (s\j);
                }
            \end{tikzpicture}
        \end{column}
    \end{columns}
\end{frame}

\begin{frame}[c, fragile]{Implementação: Feature Extractor}
\begin{code}[CNN Simples para MNIST]{python}
import torch
import torchvision.models as models

# Carregar ResNet-18 pré-treinada
resnet = models.resnet18(pretrained=True)

# Congelar todos os parâmetros
for param in resnet.parameters():
    param.requires_grad = False

# Substituir última camada (1000 -> 10 classes)
num_features = resnet.fc.in_features
resnet.fc = nn.Linear(num_features, 10)

# Apenas a última camada será treinada
trainable_params = sum(p.numel() for p in resnet.parameters() 
                      if p.requires_grad)
print(f"Parâmetros treináveis: {trainable_params}")  # 5,130
\end{code}
\end{frame}

\begin{frame}[c, fragile]{Implementação: Feature Extractor}
\begin{code}[CNN Simples para MNIST]{python}
# Otimizador só para parâmetros treináveis
optimizer = torch.optim.Adam(resnet.fc.parameters(), lr=0.001)

# Treinar normalmente
for epoch in range(10):
    for inputs, labels in dataloader:
        outputs = resnet(inputs)
        loss = criterion(outputs, labels)
        loss.backward()
        optimizer.step()
\end{code}
\end{frame}

\begin{frame}[c, fragile]{Implementação: Fine-tuning Gradual}
\begin{code}[]{python}
# Fine-tuning com descongelamento gradual
def unfreeze_layers(model, layer_name):
    """Descongela camadas a partir de layer_name"""
    found = False
    for name, param in model.named_parameters():
        if layer_name in name:
            found = True
        if found:
            param.requires_grad = True

# Estratégia de descongelamento
resnet = models.resnet18(pretrained=True)

# Fase 1: Apenas FC (épocas 1-5)
for param in resnet.parameters():
    param.requires_grad = False
resnet.fc = nn.Linear(resnet.fc.in_features, 10)
optimizer = Adam(resnet.fc.parameters(), lr=1e-3)
train_epochs(5)
\end{code}
\end{frame}

\begin{frame}[c, fragile]{Implementação: Fine-tuning Gradual}
\begin{code}[]{python}
# Fase 2: Descongelar layer4 (épocas 6-10)
unfreeze_layers(resnet, 'layer4')
optimizer = Adam(filter(lambda p: p.requires_grad, 
                       resnet.parameters()), lr=1e-4)
train_epochs(5)

# Fase 3: Treinar tudo (épocas 11-15)
for param in resnet.parameters():
    param.requires_grad = True
optimizer = Adam(resnet.parameters(), lr=1e-5)
train_epochs(5)
\end{code}
\end{frame}

\begin{frame}[c, fragile]{Taxa de Aprendizado Diferenciada}
\begin{code}[]{python}
# Diferentes learning rates para diferentes camadas
def get_parameter_groups(model):
    """Agrupa parâmetros com diferentes LRs"""
    params = []
    
    # Camadas iniciais - LR muito baixo
    params.append({
        'params': model.layer1.parameters(),
        'lr': 1e-5
    })
    
    # Camadas intermediárias - LR baixo
    params.append({
        'params': model.layer2.parameters(),
        'lr': 5e-5
    })
    params.append({
        'params': model.layer3.parameters(),
        'lr': 1e-4
    })
    \end{code}
\end{frame}

\begin{frame}[c, fragile]{Taxa de Aprendizado Diferenciada}
\begin{code}[]{python}  
    # Camadas finais - LR normal
    params.append({
        'params': model.layer4.parameters(),
        'lr': 5e-4
    })
    params.append({
        'params': model.fc.parameters(),
        'lr': 1e-3
    })
    
    return params

# Criar otimizador com grupos
param_groups = get_parameter_groups(resnet)
optimizer = Adam(param_groups)
\end{code}
\end{frame}

\begin{frame}[c]{Modelos Pré-treinados Populares (2025)}
    \begin{center}
        \small
        \begin{tabular}{|l|c|c|c|c|}
            \hline
            \textbf{Modelo} & \textbf{Ano} & \textbf{Parâmetros} & \textbf{Top-1 Acc} & \textbf{Uso} \\
            \hline
            \hline
            AlexNet & 2012 & 61M & 56.1\% & Histórico \\
            \hline
            VGG-16 & 2014 & 138M & 71.5\% & Segmentação \\
            \hline
            ResNet-50 & 2015 & 25M & 76.1\% & \highlight{Padrão} \\
            \hline
            MobileNet V3 & 2019 & 5.4M & 75.2\% & Mobile \\
            \hline
            EfficientNet-B7 & 2019 & 66M & 84.3\% & SOTA CNN \\
            \hline
            Vision Transformer & 2020 & 86M & 85.8\% & Trending \\
            \hline
            ConvNeXt & 2022 & 89M & 87.8\% & \highlight{Moderna} \\
            \hline
            DINOv2 & 2023 & 1B & 88.3\% & Self-supervised \\
            \hline
            EVA-02 & 2024 & 1B & 90.1\% & SOTA atual \\
            \hline
        \end{tabular}
    \end{center}
    
    \vspace{0.3cm}
    \textbf{Observações:}
    \begin{itemize}
        \item ResNet-50 ainda é excelente para transfer learning
        \item EfficientNet oferece melhor trade-off tamanho/performance
        \item Vision Transformers dominam benchmarks mas precisam mais dados
    \end{itemize}
\end{frame}

\begin{frame}[c]{Aplicações Práticas de Transfer Learning}
    \vspace{.5cm}
    \begin{columns}[c]
        \begin{column}{0.5\textwidth}
            \textbf{Medicina:}
            \begin{itemize}
                \item Detecção de câncer em mamografias
                \item COVID-19 em raios-X
                \item Retinopatia diabética
                \item Base: ImageNet $\to$ Fine-tune médico
            \end{itemize}
            
            \textbf{Agricultura:}
            \begin{itemize}
                \item Identificação de pragas
                \item Estimativa de colheita
                \item Análise de solo via drone
                \item Base: ImageNet $\to$ Dados agrícolas
            \end{itemize}
        \end{column}
        \begin{column}{0.5\textwidth}
            \textbf{Indústria:}
            \begin{itemize}
                \item Detecção de defeitos
                \item Controle de qualidade
                \item Manutenção preditiva
                \item Base: ImageNet $\to$ Imagens industriais
            \end{itemize}
            
            \textbf{Varejo/E-commerce:}
            \begin{itemize}
                \item Busca visual de produtos
                \item Classificação automática
                \item Detecção de falsificações
                \item Base: Fashion-MNIST $\to$ Catálogo
            \end{itemize}
        \end{column}
    \end{columns}
    \begin{attention}[Sucesso do Transfer Learning]
        \centering
        \highlight{90\% das aplicações} de visão computacional em produção usam transfer learning!
    \end{attention}
\end{frame}

\begin{frame}[c]{Comparação: Treinar do Zero vs Transfer Learning}
    \begin{center}
        \begin{tikzpicture}[scale=0.9]
            % Axes
            \draw[->] (0,0) -- (8,0) node[right] {Épocas};
            \draw[->] (0,0) -- (0,4) node[above] {Acurácia (\%)};
            
            % Grid
            \foreach \x in {1,...,7} {
                \draw[gray, very thin] (\x,0) -- (\x,3.5);
            }
            \foreach \y in {1,...,3} {
                \draw[gray, very thin] (0,\y) -- (7,\y);
            }
            
            % Labels
            \foreach \x in {1,...,7} {
                \node[below] at (\x,0) {\x 0};
            }
            \node[left] at (0,1) {70};
            \node[left] at (0,2) {80};
            \node[left] at (0,3) {90};
            
            % From scratch curve
            \draw[ultra thick, red] (0,0.2) .. controls (2,0.8) and (4,1.5) .. (7,2.3);
            \node[red, right] at (7,2.3) {Do zero: 83\%};
            
            % Transfer learning curve
            \draw[ultra thick, blue] (0,1.8) .. controls (1,2.3) and (3,2.7) .. (7,3.2);
            \node[blue, right] at (7,3.2) {Transfer: 92\%};
            
            % Legend
            \draw[ultra thick, red] (1,3.5) -- (2,3.5);
            \node[right] at (2,3.5) {\small Treinar do zero};
            \draw[ultra thick, blue] (4,3.5) -- (5,3.5);
            \node[right] at (5,3.5) {\small Transfer Learning};
        \end{tikzpicture}
    \end{center}
    
    \vspace{0.3cm}
    \begin{columns}[t]
        \begin{column}{0.5\textwidth}
            \textbf{Treinar do Zero:}
            \begin{itemize}
                \item Precisa $\sim$100k imagens
                \item 100+ épocas
                \item Convergência lenta
                \item GPU por dias/semanas
            \end{itemize}
        \end{column}
        \begin{column}{0.5\textwidth}
            \textbf{Transfer Learning:}
            \begin{itemize}
                \item Funciona com $\sim$1k imagens
                \item 10-30 épocas
                \item \highlight{Começa em 70-80\%}
                \item Horas de GPU
            \end{itemize}
        \end{column}
    \end{columns}
\end{frame}

\begin{frame}[c]{Exemplo Completo: Cães vs Gatos}
    \begin{columns}[c]
        \begin{column}{0.5\textwidth}
            \textbf{Dataset:}
            \begin{itemize}
                \item 25.000 imagens (Kaggle)
                \item 12.500 cães / 12.500 gatos
                \item Split: 80/10/10
            \end{itemize}
            
            \vspace{0.3cm}
            \textbf{Resultados:}
            \begin{itemize}
                \item \highlight{Do zero}: 78\% (200 épocas)
                \item \highlight{VGG-16 frozen}: 92\% (10 épocas)
                \item \highlight{ResNet-50 fine-tune}: 97\% (25 épocas)
                \item \highlight{EfficientNet-B0}: 98.5\% (20 épocas)
            \end{itemize}
            
            \vspace{0.3cm}
            \textbf{Tempo de treino (RTX 4090):}
            \begin{itemize}
                \item Do zero: 8 horas
                \item Transfer: 30 minutos
            \end{itemize}
        \end{column}
        \begin{column}{0.5\textwidth}
            \textbf{Pipeline Completo:}
            \begin{tikzpicture}[scale=0.6]
                % Data Aug
                \node[draw, fill=yellow!20] (aug) at (0,4) {Data Augmentation};
                
                % Pretrained
                \node[draw, fill=blue!20] (pre) at (0,2.5) {ResNet-50 (ImageNet)};
                
                % Replace FC
                \node[draw, fill=green!20] (fc) at (0,1) {FC: 2048 $\to$ 2};
                
                % Train
                \node[draw, fill=orange!20] (train) at (0,-0.5) {Adam, lr=1e-4};
                
                % Result
                \node[draw, fill=red!20] (res) at (0,-2) {97\% accuracy};
                
                % Arrows
                \draw[->, thick] (aug) -- (pre);
                \draw[->, thick] (pre) -- (fc);
                \draw[->, thick] (fc) -- (train);
                \draw[->, thick] (train) -- (res);
            \end{tikzpicture}
            
            \vspace{0.3cm}
            \textbf{Matriz de Confusão:}
            \begin{center}
                \begin{tabular}{|c|c|c|}
                    \hline
                    & Cão & Gato \\
                    \hline
                    Cão & \highlight{97.2\%} & 2.8\% \\
                    \hline
                    Gato & 3.1\% & \highlight{96.9\%} \\
                    \hline
                \end{tabular}
            \end{center}
        \end{column}
    \end{columns}
\end{frame}

\begin{frame}[c]{Tendências e Futuro}
    \vspace{0.2cm}
    \begin{columns}[c]
        \begin{column}{0.5\textwidth}
            \textbf{Foundation Models:}
            \begin{itemize}
                \item CLIP (texto + imagem)
                \item DINO (self-supervised)
                \item SAM (Segment Anything)
                \item Transfer learning multimodal
            \end{itemize}
            
            % \vspace{0.3cm}
            \textbf{Efficient Transfer:}
            \begin{itemize}
                \item LoRA (Low-Rank Adaptation)
                \item Adapter layers
                \item Prompt tuning visual
                \item Parameter-efficient fine-tuning
            \end{itemize}
        \end{column}
        \begin{column}{0.5\textwidth}
            \textbf{Novos Paradigmas:}
            \begin{itemize}
                \item \highlight{Zero-shot}: sem fine-tuning!
                \item \highlight{Few-shot}: 5-10 exemplos
                \item \highlight{Cross-domain}: médico $\to$ industrial
                \item \highlight{Continual learning}: sem esquecer
            \end{itemize}
            
            % \vspace{0.3cm}
            \textbf{Aplicações Emergentes:}
            \begin{itemize}
                \item Visão 3D (NeRFs)
                \item Vídeo understanding
                \item Robótica embodied AI
                \item AR/VR em tempo real
            \end{itemize}
        \end{column}
    \end{columns}
    
    \begin{attention}[Mensagem Final]
        \centering
        \textbf{Transfer Learning democratizou Deep Learning:} qualquer um pode criar modelos state-of-the-art com recursos limitados!
    \end{attention}
\end{frame}

\begin{frame}[c]{Resumo da Aula}
    \vspace{0.5cm}
    \begin{columns}[c]
        \begin{column}{0.5\textwidth}
            \textbf{CNNs:}
            \begin{itemize}
                \item Convoluções preservam localidade
                \item Compartilhamento de pesos
                \item Pooling reduz dimensões
                \item Hierarquia de features
                \item LeNet-5: 61k parâmetros
            \end{itemize}
            
            % \vspace{0.3cm}
            \textbf{Conceitos-Chave:}
            \begin{itemize}
                \item Stride e padding
                \item Cálculo de dimensões
                \item Receptive field
                \item Feature maps
            \end{itemize}
        \end{column}
        \begin{column}{0.5\textwidth}
            \textbf{Transfer Learning:}
            \begin{itemize}
                \item Features genéricas vs específicas
                \item Feature extraction vs fine-tuning
                \item Descongelamento gradual
                \item Learning rate diferenciada
                \item 10-100x mais rápido!
            \end{itemize}
            
            % \vspace{0.3cm}
            \textbf{Na Prática:}
            \begin{itemize}
                \item Use modelos pré-treinados
                \item Comece com feature extraction
                \item Fine-tune se tiver dados
                \item PyTorch/TensorFlow facilitam
            \end{itemize}
        \end{column}
    \end{columns}
    
    % \vspace{0.5cm}
    \begin{center}
        \large
        \highlight{Próxima aula: Arquiteturas Modernas de CNNs (ResNet, Transformers)}
    \end{center}
\end{frame}

\begin{frame}[c]{Material Complementar e Exercícios}
    \textbf{Leitura Recomendada:}
    \begin{itemize}
        \item Paper original LeNet-5 (LeCun et al., 1998)
        \item ``Deep Learning'' - Goodfellow, Cap. 9
        \item CS231n Stanford - Lecture Notes
        \item Papers With Code - Transfer Learning
    \end{itemize}
\end{frame}

\begin{frame}[c, noframenumbering, plain]
    \frametitle{~}
    \vfill
    \begin{center}
        {\Huge Obrigado!}\vspace{1.5em}\\
        {\Large \highlight{Dúvidas?}}\\
    \end{center}
    \vfill
    \begin{center}
        {\small Próxima aula: 22/10/2025}\\
        {\small Tema: Arquiteturas Modernas de CNNs}
    \end{center}
\end{frame}

\end{document}