\documentclass[xcolor=dvipsnames,t,aspectratio=169]{beamer}

\usecolortheme{rose}
\usecolortheme{dolphin}
\usetheme{Boadilla}

% Encoding packages MUST come first
\usepackage[utf8]{inputenc}
\usepackage[brazil]{babel}
\usepackage[T1]{fontenc}

\usepackage{amsfonts}
\usepackage{amsmath}
\usepackage{amssymb}
% \usepackage{latexsym}
% \usepackage{graphicx}
% \usepackage{listings} 
\usepackage{xcolor}
\usepackage{amsthm}
\usepackage{url}
\usepackage{textpos}
% \usepackage{amssymb}
% \usepackage{caption}
% \usepackage{subcaption}
% \usepackage{multicol}
% \usepackage{mathrsfs}
\usepackage{float}
\setbeamersize{text margin left=1cm,text margin right=1cm,} 
\usepackage{subcaption}
\usepackage{dsfont}
\usepackage{physics}
\usepackage{mathpazo}
% \usepackage{tikz}
\usepackage{lipsum}
\usepackage{minted}
\usepackage[minted, most]{tcolorbox}
\tcbuselibrary{theorems}
\setbeamertemplate{enumerate items}[default]
\usefonttheme{professionalfonts}

\setlength{\parindent}{0pt} % zero indent

\setbeamertemplate{itemize items}[circle]
\setbeamertemplate{itemize subitem}{$\blacktriangleright$}
\definecolor{nes_dark_purple}{HTML}{4E006B}
\definecolor{nes_dark_orange}{HTML}{EE4F09}
\setbeamercolor{normal text}{fg=nes_dark_purple}\usebeamercolor*{normal text}
\setbeamercolor{math text}{fg=nes_dark_purple}\usebeamercolor[fg]{math text}


\setbeamercolor{title}{fg=nes_dark_purple}
\setbeamercolor{frametitle}{fg=nes_dark_purple}
\setbeamercolor{structure}{fg=nes_dark_purple}
\setbeamercolor{author}{fg=nes_dark_orange}
\setbeamercolor{footline}{fg=nes_dark_purple} 

\setbeamertemplate{footline}[frame number]
\setbeamertemplate{navigation symbols}{}

\definecolor{codegreen}{rgb}{0,0.6,0.2}
\definecolor{codegray}{rgb}{0.5,0.5,0.5}
\definecolor{codepurple}{rgb}{0.58,0,0.82}
\definecolor{backcolour}{rgb}{0.95,0.95,0.92}

\lstdefinestyle{mystyle}{
    backgroundcolor=\color{backcolour},   
    commentstyle=\color{codegreen},
    keywordstyle=\color{magenta},
    numberstyle=\tiny\color{codegray},
    stringstyle=\color{codepurple},
    basicstyle=\ttfamily\footnotesize,
    breakatwhitespace=false,  
    breaklines=true,                 
    captionpos=b,                    
    keepspaces=true,                 
    numbers=left,                    
    numbersep=5pt,                  
    showspaces=false,                
    showstringspaces=false,
    showtabs=false,                  
    tabsize=2
}
\input{../../templates/slides/commands}

\usepackage{tikz}
\usetikzlibrary{arrows.meta, shapes, positioning, calc}
\usepackage{algorithmic}
\usepackage{graphicx}
\usepackage{amsmath}
\usepackage{listings}
\usepackage{multicol}

% Configuração para código Python
\lstset{
    language=Python,
    basicstyle=\tiny\ttfamily,
    keywordstyle=\color{blue},
    commentstyle=\color{green!60!black},
    stringstyle=\color{orange},
    showstringspaces=false,
    breaklines=true,
    frame=single,
    numbers=left,
    numberstyle=\tiny\color{gray}
}

\titlegraphic{
    \includegraphics[scale = 0.5]{../../templates/slides/logo}
}

\logo{
\begin{tikzpicture}[overlay,remember picture]
\node[below left = 0.2cm] at (current page.30) {
    \includegraphics[width=0.1\textwidth]{../../templates/slides/logo}};
\end{tikzpicture}
}

\newcommand{\highlight}[1]{{\color{nes_dark_orange} #1}}

\title{Texto e Word Vectors} 

\author{
    Eduardo Adame
}

\date{{\color{nes_dark_purple}  \textbf{Redes Neurais}\\[0.5em] 5 de novembro de 2025 }}


\begin{document}

\frame[plain]{\titlepage}
\setcounter{framenumber}{0}

\begin{frame}[c]
\frametitle{Motivação}
\begin{itemize}
\item Demonstramos como usar Redes Neurais com dados numéricos estruturados
\item Imagens podem ser redimensionadas para um tamanho específico
\item Valores de imagem são números (escala de cinza, RGB)
\item Mas como trabalhamos com texto?
\item \textbf{Problema 1:} Como lidar com sequências de texto (sequências de palavras com comprimento variável)?
\item \textbf{Problema 2:} Como converter palavras em algo numérico?
\end{itemize}
\end{frame}

\begin{frame}[c]
\frametitle{Problema: Sequências de Comprimento Variável}
\begin{itemize}
\item Com imagens, forçamos dimensões de entrada específicas
\item Não é óbvio como fazer isso com texto
\item Usaremos uma nova estrutura de rede chamada \textit{"Rede Neural Recorrente" (RNN)} que será discutida na próxima aula
\item Hoje focamos em: \textbf{como representar palavras numericamente}
\end{itemize}
\end{frame}

\begin{frame}[c]
\frametitle{Tokenização}
\begin{itemize}
\item Necessário converter palavra em algo numérico
\item \textbf{Primeira abordagem:} Tokenização
\item Tratar como variável categórica com enorme número de categorias (codificação one-hot)
\item Lidar com detalhes de maiúsculas, pontuação, etc.
\end{itemize}

\vspace{0.5cm}
\begin{center}
\texttt{"O gato de botas."}

$\downarrow$

\texttt{['o', 'gato', 'de', 'botas', '.', '<EOS>']}
\end{center}
\end{frame}

\begin{frame}[c, fragile]
\frametitle{Tokenização - Construindo o Vocabulário}
\begin{columns}
\begin{column}{0.5\textwidth}
\begin{itemize}
\item Usar tokens para construir vocabulário
\item Vocabulário é um mapeamento um-para-um de índice para token
\item Geralmente representado por lista e dicionário
\end{itemize}
\end{column}

\begin{column}{0.5\textwidth}
\textbf{índice $\rightarrow$ palavra}
\begin{code}[]{python}
['<EOS>',
 'o',
 'gato',
 'de',
 'botas',
 '.']
\end{code}

\textbf{palavra $\rightarrow$ índice}
\begin{code}[]{python}
{'<EOS>': 0,
 'o': 1,
 'gato': 2,
 'de': 3,
 'botas': 4,
 '.': 5}
\end{code}
\end{column}
\end{columns}
\end{frame}

\begin{frame}[c]
\frametitle{Problemas com Tokenização}
\begin{itemize}
\item Tokenização perde muita informação sobre palavras:
\begin{itemize}
\item Classe gramatical (substantivo, verbo, etc.)
\item \textbf{Sinonímia} (palavras distintas com significado igual ou similar)
\item \textbf{Polissemia} (palavra única com múltiplos significados)
\item Contexto geral em que a palavra provavelmente aparece\\
(ex.: "desemprego" e "inflação" são ambas sobre economia)
\end{itemize}
\item Aumentar tamanho do vocabulário é difícil (requereria re-treinar o modelo)
\item Comprimento do vetor é enorme $\rightarrow$ grande número de pesos
\item Ainda assim, informação no vetor é muito \textbf{esparsa}
\end{itemize}
\end{frame}

\begin{frame}[c]
\frametitle{Vetores de Palavras (Word Embeddings)}
\begin{itemize}
\item \textbf{Objetivo:} representar palavra por vetor $m$-dimensional (para $m$ de tamanho médio, digamos, $m=300$)
\item Ter palavras "similares" representadas por vetores "próximos" neste espaço $m$-dimensional
\item Palavras em domínio particular (economia, ciência, esportes) poderiam estar mais próximas entre si
\item Poderia ajudar com sinonímia
\begin{itemize}
\item ex.: "grande" e "enorme" têm vetores próximos
\end{itemize}
\item Poderia ajudar com polissemia
\begin{itemize}
\item "manga" (fruta) e "laranja" próximas em algumas dimensões
\item "manga" (roupa) e "bolso" próximas em outras dimensões
\end{itemize}
\end{itemize}
\end{frame}

\begin{frame}[c]
\frametitle{Vetores de Palavras - Vantagens}
\begin{itemize}
\item Vetores seriam mais curtos e densos em informação, ao invés de muito longos e esparsos
\item Requeririam menos pesos e parâmetros
\item \textbf{Felizmente:} existem mapeamentos pré-treinados que podem ser baixados e utilizados
\item Foram treinados em grandes corpora por muito tempo
\item Vamos entender como foram desenvolvidos e treinados
\end{itemize}
\end{frame}

\begin{frame}[c]
\frametitle{O que Torna Duas Palavras Similares?}
\begin{itemize}
\item \textbf{Ideia:} palavras similares ocorrem em contextos similares
\item Para uma palavra dada, olhar palavras em uma "janela" ao redor dela
\item Considerar tentar prever uma palavra dado o contexto
\item Este é exatamente o modelo \textbf{CBOW} (Continuous Bag of Words)
\end{itemize}

\vspace{0.3cm}
\begin{center}
\textit{"Todos são iguais perante a \textcolor{red}{lei}, sem distinção de qualquer natureza"}

$\downarrow$

(['iguais', 'perante', 'a', 'sem', 'distinção', 'de'], '\textcolor{red}{lei}')

\vspace{0.2cm}
\small{contexto \hspace{3cm} palavra-alvo}
\end{center}
\end{frame}

\begin{frame}[c]
\frametitle{Modelo CBOW}
\textbf{Treinar rede neural em grande corpus de dados.}


\begin{itemize}
\item Palavras de contexto (one-hot encoded)
\item Camada oculta única (dimensão m)
\item Palavra alvo (one-hot encoded)
\end{itemize}
\end{frame}

\begin{frame}[c]
\frametitle{Modelo CBOW - Extraindo Embeddings}
\textbf{Uma vez que a rede é treinada, os pesos $\rightarrow$ vetores de palavras.}


\begin{figure}[h]
\centering
\includegraphics[width=.75\textwidth]{CBOW.png}
\end{figure}


\end{frame}
\begin{frame}[c]
\frametitle{Modelo CBOW - Extraindo Embeddings}
\begin{itemize}
\item A matriz de pesos entre entrada e camada oculta representa os word embeddings
\item Cada linha corresponde ao vetor de uma palavra
\item Dimensão: vocabulário $\times m$
\end{itemize}
\end{frame}

\begin{frame}[c]
\frametitle{Modelo Skip-gram}
\textbf{Mesma ideia, exceto que predizemos o contexto a partir do alvo.}

\begin{figure}[h]
\centering
\includegraphics[width=.75\textwidth]{skipgram.png}
\end{figure}

\end{frame}
\begin{frame}[c]
\frametitle{Modelo CBOW - Extraindo Embeddings}

\begin{itemize}
\item Palavra alvo (one-hot encoded)
\item Camada oculta de Rede Neural
\item Palavras de contexto (one-hot encoded)
\item Geralmente funciona melhor que CBOW
\end{itemize}
\end{frame}

\begin{frame}[c]
\frametitle{Word2Vec (2013)}
\begin{itemize}
\item \textit{Distributed Representations of Words and Phrases and Their Compositionality}, Mikolov et al.
\item Usa modelo Skip-gram para treinar em grande corpus
\item Muitos detalhes para fazê-lo funcionar melhor:
\begin{itemize}
\item Agregação de frases multi-palavras (ex.: "São Paulo")
\item Subamostragem (sobreamostrar palavras menos comuns)
\item Amostragem negativa (dar exemplos de palavras erradas à rede)
\end{itemize}
\item Implementações disponíveis em \texttt{gensim}, \texttt{spaCy}
\item Modelos pré-treinados disponíveis online
\end{itemize}
\end{frame}

\begin{frame}[c]
\frametitle{Propriedades Interessantes - Analogias}
\begin{itemize}
\item Word2Vec captura relações semânticas e sintáticas
\item Exemplo clássico de analogia:
\end{itemize}

\begin{center}
\Large
\textbf{rei} $-$ homem $+$ mulher $\approx$ \textbf{rainha}

\vspace{0.5cm}

\textbf{Paris} $-$ França $+$ Brasil $\approx$ \textbf{Brasília}
\end{center}

\begin{itemize}
\item Operações vetoriais capturam relações conceituais
\item Útil para tarefas de reasoning e NLP
\end{itemize}
\end{frame}

\begin{frame}[c]
\frametitle{GloVe (2014)}
\begin{itemize}
\item \textbf{Global Vectors for Word Representation} (GloVe)
\item Usa matriz de co-ocorrência com palavras vizinhas para determinar similaridade
\end{itemize}

\vspace{0.3cm}
$$J = \frac{1}{2}\sum_{i,j=1}^{W} f(P_{ij})(u_i^T v_j - \log(P_{ij}))^2$$

\vspace{0.3cm}
\begin{itemize}
\item $f$ $\rightarrow$ frequência de palavra, com limite máximo
\item $P_{ij}$ $\rightarrow$ probabilidade das palavras i e j ocorrerem juntas
\item Abordagem mais estatística que Word2Vec
\end{itemize}
\end{frame}

\begin{frame}[c]
\frametitle{GloVe - Disponibilidade}
\begin{itemize}
\item GloVe é publicamente disponível
\item Desenvolvido em Stanford: \url{https://nlp.stanford.edu/projects/glove/}
\item Treinado em enormes corpora:
\begin{itemize}
\item Wikipedia 2014 + Gigaword 5 (6B tokens, 400K vocab)
\item Common Crawl (42B tokens, 1.9M vocab)
\item Common Crawl (840B tokens, 2.2M vocab)
\end{itemize}
\item Diferentes dimensionalidades disponíveis (50, 100, 200, 300)
\item Amplamente usado em pesquisa e indústria
\end{itemize}
\end{frame}

\begin{frame}[c]
\frametitle{Comparação: Word2Vec vs GloVe}
\begin{table}
\small
\begin{tabular}{|l|l|l|}
\hline
\textbf{Aspecto} & \textbf{Word2Vec} & \textbf{GloVe} \\
\hline
Abordagem & Preditiva (NN) & Baseada em contagem \\
\hline
Contexto & Local (janela) & Global (corpus todo) \\
\hline
Treinamento & Mais rápido & Mais lento \\
\hline
Memória & Menor & Maior (matriz) \\
\hline
Performance & Similar & Similar \\
\hline
\end{tabular}
\end{table}

\vspace{0.3cm}
\begin{itemize}
\item Na prática, performance similar na maioria das tarefas
\item Escolha depende de recursos computacionais e corpus
\end{itemize}
\end{frame}

\begin{frame}[c]
\frametitle{Limitações dos Word Embeddings Estáticos}
\begin{itemize}
\item \textbf{Problema principal:} uma palavra = um vetor único
\item Não capturam contexto dinâmico:
\begin{itemize}
\item "banco" em "banco de dados" vs "banco do parque"
\item Polissemia não é totalmente resolvida
\end{itemize}
\item Não lidam bem com:
\begin{itemize}
\item Palavras fora do vocabulário (OOV)
\item Morfologia e palavras compostas
\item Mudanças de significado ao longo do tempo
\end{itemize}
\item Solução: \textbf{Embeddings Contextuais} (próximo tópico)
\end{itemize}
\end{frame}

\begin{frame}[c]
\frametitle{Evolução: Embeddings Contextuais (2018-2025)}
\begin{itemize}
\item \textbf{ELMo} (2018): Embeddings de modelos de linguagem bidirecionais
\item \textbf{BERT} (2018): Transformers bidirecionais
\item \textbf{GPT} (2018-2023): Modelos autoregressivos
\item \textbf{2025:} Modelos ainda mais sofisticados
\begin{itemize}
\item Embeddings adaptam-se ao contexto da frase
\item "banco" terá representações diferentes dependendo do uso
\item Base para LLMs modernos (ChatGPT, Claude, etc.)
\end{itemize}
\item \textbf{Diferença chave:} representação depende do contexto completo
\end{itemize}
\end{frame}

\begin{frame}[c]
\frametitle{Aplicações Práticas de Word Embeddings}
\begin{enumerate}
\item \textbf{Análise de Sentimento}
\begin{itemize}
\item Classificar opiniões como positivas/negativas
\end{itemize}
\item \textbf{Sistemas de Recomendação}
\begin{itemize}
\item Encontrar produtos/conteúdos similares
\end{itemize}
\item \textbf{Tradução Automática}
\begin{itemize}
\item Representar palavras em diferentes idiomas
\end{itemize}
\item \textbf{Busca Semântica}
\begin{itemize}
\item Encontrar documentos por significado, não apenas palavras-chave
\end{itemize}
\item \textbf{Chatbots e Assistentes Virtuais}
\begin{itemize}
\item Entender intenção do usuário
\end{itemize}
\end{enumerate}
\end{frame}

\begin{frame}[c]
\frametitle{Avaliação de Word Embeddings}
\begin{itemize}
\item \textbf{Avaliação Intrínseca:}
\begin{itemize}
\item Testes de analogia (rei - homem + mulher = ?)
\item Tarefas de similaridade de palavras
\item Correlação com julgamentos humanos
\end{itemize}
\item \textbf{Avaliação Extrínseca:}
\begin{itemize}
\item Performance em tarefas downstream
\item Classificação de texto
\item Named Entity Recognition (NER)
\item Part-of-Speech (POS) tagging
\end{itemize}
\item Benchmarks comuns: WordSim-353, SimLex-999, Google Analogy
\end{itemize}
\end{frame}

\begin{frame}[c, fragile]
\frametitle{Implementação Prática - Python}
\begin{itemize}
\item \textbf{Bibliotecas principais:}
\begin{itemize}
\item \texttt{gensim}: Word2Vec, FastText
\item \texttt{spaCy}: Embeddings pré-treinados integrados
\item \texttt{transformers} (HuggingFace): BERT, GPT
\item \texttt{tensorflow/keras}: Embedding layers
\end{itemize}
\item \textbf{Carregar embeddings pré-treinados:}
\end{itemize}

\begin{code}[]{python}
from gensim.models import KeyedVectors
# Carregar GloVe
model = KeyedVectors.load_word2vec_format(
    'glove.6B.300d.txt', binary=False)
# Similaridade
model.similarity('gato', 'cachorro')
\end{code}
\end{frame}

\begin{frame}[c]
\frametitle{Demonstração Interativa}
\begin{itemize}
\item \textbf{TensorFlow Embedding Projector}
\begin{itemize}
\item \url{https://projector.tensorflow.org}
\item Visualização 3D interativa de embeddings
\item Busca por vizinhos mais próximos
\item Visualização de relações analógicas
\end{itemize}
\item \textbf{Experimente:}
\begin{itemize}
\item Busque "king" e veja palavras similares
\item Use PCA ou t-SNE para redução dimensional
\item Explore diferentes datasets (Word2Vec, GloVe)
\end{itemize}
\end{itemize}

\end{frame}

\begin{frame}[c]
\frametitle{Recursos Adicionais}
\begin{itemize}
\item \textbf{Papers Fundamentais:}
\begin{itemize}
\item Mikolov et al. (2013) - Word2Vec
\item Pennington et al. (2014) - GloVe
\item Peters et al. (2018) - ELMo
\end{itemize}
\item \textbf{Tutoriais e Cursos:}
\begin{itemize}
\item Stanford CS224N: NLP with Deep Learning
\item Fast.ai: Practical Deep Learning for Coders
\end{itemize}
\item \textbf{Datasets e Modelos:}
\begin{itemize}
\item \url{https://nlp.stanford.edu/projects/glove/}
\item \url{https://code.google.com/archive/p/word2vec/}
\item HuggingFace Model Hub
\end{itemize}
\end{itemize}
\end{frame}

\begin{frame}[c]
\frametitle{Próxima Aula}
\begin{itemize}
\item \textbf{Redes Neurais Recorrentes (RNN)}
\begin{itemize}
\item Como processar sequências de comprimento variável
\item LSTM e GRU
\item Aplicações em processamento de texto
\end{itemize}
\item \textbf{Preparação:}
\begin{itemize}
\item Revisar conceitos de word embeddings
\item Familiarizar-se com notação de sequências
\item Instalar bibliotecas necessárias
\end{itemize}
\end{itemize}
\end{frame}

\begin{frame}[c, noframenumbering, plain]
    \frametitle{~}
    \vfill
    \begin{center}
        {\Huge Obrigado!}\vspace{1.5em}\\
        {\Large \highlight{Dúvidas?}}\\
    \end{center}
    \vfill
    % \begin{center}
    %     {\small Próxima aula: 22/10/2025}\\
    %     {\small Tema: Arquiteturas Modernas de CNNs}
    % \end{center}
\end{frame}

\end{document}