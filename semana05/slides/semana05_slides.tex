\documentclass[xcolor=dvipsnames,t,aspectratio=169]{beamer}

\usecolortheme{rose}
\usecolortheme{dolphin}
\usetheme{Boadilla}

% Encoding packages MUST come first
\usepackage[utf8]{inputenc}
\usepackage[brazil]{babel}
\usepackage[T1]{fontenc}

\usepackage{amsfonts}
\usepackage{amsmath}
\usepackage{amssymb}
% \usepackage{latexsym}
% \usepackage{graphicx}
% \usepackage{listings} 
\usepackage{xcolor}
\usepackage{amsthm}
\usepackage{url}
\usepackage{textpos}
% \usepackage{amssymb}
% \usepackage{caption}
% \usepackage{subcaption}
% \usepackage{multicol}
% \usepackage{mathrsfs}
\usepackage{float}
\setbeamersize{text margin left=1cm,text margin right=1cm,} 
\usepackage{subcaption}
\usepackage{dsfont}
\usepackage{physics}
\usepackage{mathpazo}
% \usepackage{tikz}
\usepackage{lipsum}
\usepackage{minted}
\usepackage[minted, most]{tcolorbox}
\input{../../templates/slides/settings}
\input{../../templates/slides/commands}

\usepackage{tikz}
\usetikzlibrary{arrows.meta, shapes, positioning}
\usepackage{algorithmic}

\titlegraphic{
    \includegraphics[scale = 0.5]{../../templates/slides/logo}
}

\logo{
\begin{tikzpicture}[overlay,remember picture]
\node[below left = 0.2cm] at (current page.30) {
    \includegraphics[width=0.1\textwidth]{../../templates/slides/logo}};
\end{tikzpicture}
}

\newcommand{\highlight}[1]{{\color{nes_dark_orange} #1}}

\title{Regularização e Dropout em Redes Neurais} 

\author{
    Eduardo Adame
}

\date{{\color{nes_dark_purple}  \textbf{Redes Neurais}\\[0.5em] 10 de setembro de 2025 }}

\begin{document}

\frame[plain]{\titlepage}
\setcounter{framenumber}{0}

\section{O Problema do Overfitting}

\begin{frame}[c]{O Desafio do Overfitting}
    \begin{columns}[c]
        \begin{column}{0.5\textwidth}
            \textbf{O que é Overfitting?}
            \begin{itemize}
                \item Modelo memoriza dados de treino
                \item Não generaliza para novos dados
                \item Performance ruim em validação/teste
            \end{itemize}
            
            \vspace{0.5cm}
            \textbf{Sinais de Overfitting:}
            \begin{itemize}
                \item[$\triangleright$] Erro de treino $\downarrow$ muito baixo
                \item[$\triangleright$] Erro de validação $\uparrow$ alto
                \item[$\triangleright$] Grande diferença entre os dois
            \end{itemize}
        \end{column}
        \begin{column}{0.5\textwidth}
            \begin{tikzpicture}[scale=0.8]
                % Gráfico de erro
                \draw[->] (0,0) -- (5,0) node[right] {Épocas};
                \draw[->] (0,0) -- (0,3.5) node[above] {Erro};
                
                % Curva de treino
                \draw[thick, blue] (0,3) .. controls (1,2) and (2,1.2) .. (4.5,0.5);
                \node[blue, right] at (4.5,0.5) {Treino};
                
                % Curva de validação
                \draw[thick, red] (0,3) .. controls (1,2.2) and (2,1.8) .. (3,1.7) .. controls (3.5,1.8) .. (4.5,2.5);
                \node[red, right] at (4.5,2.5) {Validação};
                
                % Região de overfitting
                \draw[dashed, gray] (2.5,0) -- (2.5,3.5);
                \node[gray] at (3.5,3.2) {Overfitting};
            \end{tikzpicture}
        \end{column}
    \end{columns}
\end{frame}

\begin{frame}[c]{Estratégias de Regularização}
    \begin{center}
        \textbf{Como Prevenir Overfitting?}
        
        \vspace{0.5cm}
        \begin{tikzpicture}[scale=0.9]
            % Nó central
            \node[circle, draw, fill=blue!20, minimum size=2cm] (center) {\textbf{Regularização}};
            
            % Técnicas ao redor
            \node[rectangle, draw, fill=green!20, rounded corners] (penalty) at (0,3) {Penalização L2/L1};
            \node[rectangle, draw, fill=yellow!20, rounded corners] (dropout) at (3,2) {\highlight{Dropout}};
            \node[rectangle, draw, fill=orange!20, rounded corners] (early) at (3,-2) {Early Stopping};
            \node[rectangle, draw, fill=red!20, rounded corners] (batch) at (0,-3) {Mini-batch SGD};
            \node[rectangle, draw, fill=purple!20, rounded corners] (data) at (-3,-2) {Mais Dados};
            \node[rectangle, draw, fill=cyan!20, rounded corners] (aug) at (-3,2) {Data Augmentation};
            
            % Conexões
            \draw[->] (center) -- (penalty);
            \draw[->] (center) -- (dropout);
            \draw[->] (center) -- (early);
            \draw[->] (center) -- (batch);
            \draw[->] (center) -- (data);
            \draw[->] (center) -- (aug);
        \end{tikzpicture}
    \end{center}
\end{frame}

\section{Regularização L2 e L1}

\begin{frame}[c]{Regularização com Penalização de Pesos}
    \begin{columns}[c]
        \begin{column}{0.55\textwidth}
            \textbf{Ideia Principal:}
            \begin{itemize}
                \item Adicionar termo de penalização à função de custo
                \item Penalizar pesos grandes
                \item Forçar modelo mais simples
            \end{itemize}
            
            \vspace{0.5cm}
            \textbf{Função de Custo Regularizada:}
            $$J_{reg} = J_{original} + \lambda \cdot R(W)$$
            
            Onde:
            \begin{itemize}
                \item $\lambda$: força da regularização
                \item $R(W)$: termo de regularização
            \end{itemize}
        \end{column}
        \begin{column}{0.45\textwidth}
            \textbf{Tipos de Regularização:}
            
            % \vspace{0.3cm}
            \highlight{L2 (Ridge):}
            $$R(W) = \sum_{i,j} W_{ij}^2$$
            
            \begin{itemize}
                \item[$+$] Pesos pequenos
                \item[$+$] Suave
            \end{itemize}
            
            % \vspace{0.3cm}
            \highlight{L1 (Lasso):}
            $$R(W) = \sum_{i,j} |W_{ij}|$$
            
            \begin{itemize}
                \item[$+$] Esparsidade
                \item[$+$] Seleção de features
            \end{itemize}
        \end{column}
    \end{columns}
\end{frame}

\begin{frame}[c]{Efeito da Regularização L2}
    \begin{columns}[c]
        \begin{column}{0.5\textwidth}
            \textbf{Atualização com L2:}
            $$W_{novo} = W_{antigo} - \alpha \left(\frac{\partial J}{\partial W} + 2\lambda W_{antigo}\right)$$
            
            \vspace{0.5cm}
            \textbf{Interpretação:}
            \begin{itemize}
                \item ``Weight decay'' (decaimento de peso)
                \item Pesos diminuem a cada iteração
                \item Previne valores extremos
            \end{itemize}
        \end{column}
        \begin{column}{0.5\textwidth}
            \begin{tikzpicture}[scale=0.8]
                % Espaço de pesos
                \draw[->] (-0.5,0) -- (4,0) node[right] {$W_1$};
                \draw[->] (0,-0.5) -- (0,4) node[above] {$W_2$};
                
                % Contornos sem regularização
                \foreach \r in {0.5, 1, 1.5, 2} {
                    \draw[gray!40] (1.5,1.5) ellipse ({1.3*\r} and {0.8*\r});
                }
                
                % Círculo de regularização L2
                \draw[thick, blue] (0,0) circle (1.8);
                \node[blue] at (1.5,2) {Restrição L2};
                
                % Ponto ótimo
                \node[circle, fill=red, inner sep=2pt] at (1.27,1.27) {};
                \node[red, right] at (1.27,1.27) {Ótimo};
                
                % Mínimo original
                \node[mark=x, mark size=4pt, thick] at (1.5,1.5) {};
                \node[right] at (2,2.5) {Sem reg.};
            \end{tikzpicture}
        \end{column}
    \end{columns}
\end{frame}

\section{Dropout}

\begin{frame}[c]{Dropout: Desligando Neurônios Aleatoriamente}
    \begin{columns}[c]
        \begin{column}{0.55\textwidth}
            \textbf{Conceito do Dropout:}
            \begin{itemize}
                \item Durante \highlight{treinamento}:
                \begin{itemize}
                    \item Desligar neurônios aleatoriamente
                    \item Probabilidade $p$ de manter ativo
                    \item Diferente a cada batch
                \end{itemize}
                \item Durante \highlight{teste}:
                \begin{itemize}
                    \item Todos neurônios ativos
                    \item Pesos multiplicados por $p$
                \end{itemize}
            \end{itemize}
            
            \vspace{0.5cm}
            \textbf{Por que funciona?}
            \begin{itemize}
                \item[$\checkmark$] Previne co-adaptação
                \item[$\checkmark$] Como treinar ensemble
                \item[$\checkmark$] Força redundância
            \end{itemize}
        \end{column}
        \begin{column}{0.45\textwidth}
            \begin{tikzpicture}[scale=0.7]
                % Título
                \node[above] at (2,4) {\textbf{Treinamento}};
                
                % Camada de entrada
                \foreach \i in {0,1,2} {
                    \node[circle, draw, fill=green!30] (i\i) at (0,\i) {};
                }
                
                % Camada oculta (com dropout)
                \node[circle, draw, fill=blue!30] (h0) at (2,0) {};
                \node[circle, draw, fill=gray!30, dashed] (h1) at (2,1) {$\times$};
                \node[circle, draw, fill=blue!30] (h2) at (2,2) {};
                \node[circle, draw, fill=gray!30, dashed] (h3) at (2,3) {$\times$};
                
                % Camada de saída
                \node[circle, draw, fill=orange!30] (o0) at (4,1.5) {};
                
                % Conexões (apenas neurônios ativos)
                \foreach \i in {0,1,2} {
                    \draw[->] (i\i) -- (h0);
                    \draw[->] (i\i) -- (h2);
                }
                \draw[->] (h0) -- (o0);
                \draw[->] (h2) -- (o0);
                
                % Label
                \node[red] at (2,-1) {p = 0.5};
            \end{tikzpicture}
        \end{column}
    \end{columns}
\end{frame}

\begin{frame}[c]{Implementação do Dropout}
    \begin{columns}[c]
        \begin{column}{0.5\textwidth}
            \textbf{Durante o Treinamento:}
            \begin{enumerate}
                \item Gerar máscara aleatória
                \item Aplicar à camada
                \item Forward e backward pass normal
            \end{enumerate}
            
            \vspace{0.3cm}
            \textbf{Pseudocódigo:}
            \begin{itemize}
                \item[] \texttt{mask = random(0,1) > dropout\_rate}
                \item[] \texttt{output = input * mask}
                \item[] \texttt{output = output / (1-dropout\_rate)}
            \end{itemize}
            
            \vspace{0.3cm}
            \textbf{Inverted Dropout:}
            \begin{itemize}
                \item Escala durante treino
                \item Teste fica inalterado
            \end{itemize}
        \end{column}
        \begin{column}{0.5\textwidth}
            \textbf{Valores Típicos de Dropout:}
            
            \vspace{0.3cm}
            \begin{tikzpicture}[scale=0.8]
                % Barras horizontais
                \draw[fill=blue!30] (0,3) rectangle (2,3.5);
                \node[left] at (0,3.25) {Entrada:};
                \node[right] at (2,3.25) {0.1-0.2};
                
                \draw[fill=blue!50] (0,2) rectangle (3,2.5);
                \node[left] at (0,2.25) {Ocultas:};
                \node[right] at (3,2.25) {0.5};
                
                \draw[fill=blue!70] (0,1) rectangle (4,1.5);
                \node[left] at (0,1.25) {Convolucionais:};
                \node[right] at (4,1.25) {0.2-0.3};
                
                \draw[fill=blue!90] (0,0) rectangle (3.5,0.5);
                \node[left] at (0,0.25) {Recorrentes:};
                \node[right] at (3.5,0.25) {0.1-0.5};
            \end{tikzpicture}
            
            \vspace{0.5cm}
            \highlight{Dica:} Começar sem dropout, adicionar se houver overfitting
        \end{column}
    \end{columns}
\end{frame}

\begin{frame}[c]{Dropout vs Regularização L2}
    \begin{center}
        \begin{tabular}{|l|c|c|}
            \hline
            \textbf{Aspecto} & \textbf{Dropout} & \textbf{L2} \\
            \hline
            \hline
            Tipo & Estocástico & Determinístico \\
            \hline
            Aplicação & Por camada & Global \\
            \hline
            Interpretação & Ensemble & Weight decay \\
            \hline
            Hiperparâmetros & Taxa por camada & $\lambda$ único \\
            \hline
            Tempo de treino & Mais lento & Normal \\
            \hline
            Teste & Precisa ajuste & Direto \\
            \hline
        \end{tabular}
    \end{center}
    
    \vspace{0.5cm}
    \begin{columns}[c]
        \begin{column}{0.5\textwidth}
            \textbf{Quando usar Dropout:}
            \begin{itemize}
                \item[$\bullet$] Redes profundas
                \item[$\bullet$] Muitos parâmetros
                \item[$\bullet$] Dados limitados
            \end{itemize}
        \end{column}
        \begin{column}{0.5\textwidth}
            \textbf{Quando usar L2:}
            \begin{itemize}
                \item[$\bullet$] Sempre (baseline)
                \item[$\bullet$] Redes menores
                \item[$\bullet$] Combinado com dropout
            \end{itemize}
        \end{column}
    \end{columns}
\end{frame}

\begin{frame}[c]{Early Stopping}
    \begin{columns}[c]
        \begin{column}{0.55\textwidth}
            \textbf{Conceito:}
            \begin{itemize}
                \item Parar treinamento antes da convergência
                \item Monitorar erro de validação
                \item Guardar melhor modelo
            \end{itemize}
            
            \vspace{0.5cm}
            \textbf{Estratégia Prática:}
            \begin{enumerate}
                \item Definir "patience" (paciência)
                \item Se validação não melhora por N épocas
                \item Restaurar melhor checkpoint
                \item Parar treinamento
            \end{enumerate}
            
            \vspace{0.3cm}
            \highlight{Vantagem:} Simples e efetivo!
        \end{column}
        \begin{column}{0.45\textwidth}
            \begin{tikzpicture}[scale=0.8]
                % Eixos
                \draw[->] (0,-0.2) -- (5,-0.2) node[right] {Épocas};
                \draw[->] (0,-0.2) -- (0,3.5) node[above] {Erro};
                
                % Curvas
                \draw[thick, blue] (0,3) .. controls (1,2) and (2,1.2) .. (4.5,0.3);
                \node[blue] at (4.5,0.1) {Treino};
                
                \draw[thick, red] (0,3) .. controls (1,2.2) and (2,1.5) .. (2.5,1.4) .. controls (3,1.5) .. (4.5,2.2);
                \node[red] at (4.5,2.4) {Val.};
                
                % Ponto de parada
                \node[circle, fill=green, inner sep=2pt] at (2.5,1.4) {};
                \draw[dashed, green, thick] (2.5,0) -- (2.5,3.5);
                \node[green, above] at (2.5,3.5) {Parar aqui!};
                
                % Patience
                \draw[<->, orange, thick] (2.5,0.5) -- (3.5,0.5);
                \node[orange, below] at (3,0.5) {patience};
            \end{tikzpicture}
        \end{column}
    \end{columns}
\end{frame}

\begin{frame}[c, fragile]{Implementação em Keras}
    \begin{code}[Exemplo Completo com Regularização]{python}
from tensorflow.keras import layers, regularizers

model = Sequential([
    # Camada com L2 e Dropout
    layers.Dense(128, 
                activation='relu',
                kernel_regularizer=regularizers.l2(0.01)),
    layers.Dropout(0.5),
    
    # Segunda camada oculta
    layers.Dense(64, 
                activation='relu',
                kernel_regularizer=regularizers.l2(0.01)),
    layers.Dropout(0.3),
    
    # Saída
    layers.Dense(10, activation='softmax')
])
    \end{code}

\end{frame}

\begin{frame}[c, fragile]{Implementação em Keras}
    \begin{code}[Exemplo Completo com Regularização]{python} 
    # Compilar
    model.compile(loss='categorical_crossentropy',
              optimizer=Adam(learning_rate=0.01),
              metrics=['accuracy'])

    # Early stopping
    early_stop = EarlyStopping(
        monitor='val_loss',
        patience=10,
        restore_best_weights=True
    )

    model.fit(X_train, y_train,
            validation_split=0.2,
            callbacks=[early_stop])
    \end{code}
\end{frame}

\begin{frame}[c]{Combinando Técnicas de Regularização}
    \begin{center}
        \textbf{Estratégia Recomendada:}
        
        \vspace{0.5cm}
        \begin{tikzpicture}[scale=0.9]
            % Fluxo de decisão
            \node[rectangle, draw, fill=yellow!20, rounded corners] (start) at (0,0) {Começar sem regularização};
            
            \node[rectangle, draw, fill=orange!20, rounded corners] (check) at (0,-1.5) {Overfitting?};
            
            \node[rectangle, draw, fill=green!20, rounded corners] (l2) at (-4,-3) {Adicionar L2};
            
            \node[rectangle, draw, fill=blue!20, rounded corners] (dropout) at (0,-3) {Adicionar Dropout};
            
            \node[rectangle, draw, fill=red!20, rounded corners] (early) at (4,-3) {Early Stopping};

            % Setas
            \draw[-latex] (start) -- (check);
            \draw[-latex] (check) -- (l2);
            \draw[-latex] (check) -- (dropout);
            \draw[-latex] (check) -- (early);
        \end{tikzpicture}
    \end{center}
    
    \highlight{Dicas Práticas:}
    \begin{itemize}
        \item Começar com early stopping (sempre!)
        \item L2 com $\lambda$ pequeno (0.001 a 0.01)
        \item Dropout gradual (0.2 $\to$ 0.5)
    \end{itemize}
\end{frame}

\begin{frame}[c]{Resumo: Regularização e Dropout}
    \begin{columns}[c]
        \begin{column}{0.5\textwidth}
            \textbf{Conceitos Principais:}
            \begin{itemize}
                \item \highlight{Overfitting}: memorização vs generalização
                \item \highlight{Regularização L2/L1}: penalizar complexidade
                \item \highlight{Dropout}: desativar neurônios
                \item \highlight{Early Stopping}: parar no momento certo
            \end{itemize}
            
            \vspace{0.5cm}
            \textbf{Boas Práticas:}
            \begin{itemize}
                \item[$\checkmark$] Sempre usar validação
                \item[$\checkmark$] Começar simples
                \item[$\checkmark$] Adicionar regularização gradualmente
                \item[$\checkmark$] Monitorar curvas de aprendizado
            \end{itemize}
        \end{column}
        \begin{column}{0.5\textwidth}
            \textbf{Checklist de Regularização:}
            \begin{enumerate}
                \item Dividir dados (treino/val/teste)
                \item Treinar modelo base
                \item Identificar overfitting
                \item Aplicar técnicas:
                \begin{itemize}
                    \item Early stopping
                    \item L2 regularization
                    \item Dropout (se necessário)
                \end{itemize}
                \item Ajustar hiperparâmetros
                \item Validar no conjunto de teste
            \end{enumerate}
            
            \vspace{0.3cm}
            \highlight{Lembre-se:} Regularização é essencial para modelos que generalizam bem!
        \end{column}
    \end{columns}
\end{frame}

\begin{frame}[c, noframenumbering, plain]
    \frametitle{~}
    \vfill
    \begin{center}
        {\Huge Obrigado!}\vspace{1.5em}\\
        {\Large \highlight{Dúvidas?}}\\
    \end{center}
    \vfill
    \begin{center}
        {\small Próximo: Introdução às CNNs}
    \end{center}
\end{frame}

\end{document}