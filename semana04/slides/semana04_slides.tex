\documentclass[xcolor=dvipsnames,t,aspectratio=169]{beamer}

\usecolortheme{rose}
\usecolortheme{dolphin}
\usetheme{Boadilla}

% Encoding packages MUST come first
\usepackage[utf8]{inputenc}
\usepackage[brazil]{babel}
\usepackage[T1]{fontenc}

\usepackage{amsfonts}
\usepackage{amsmath}
\usepackage{amssymb}
% \usepackage{latexsym}
% \usepackage{graphicx}
% \usepackage{listings} 
\usepackage{xcolor}
\usepackage{amsthm}
\usepackage{url}
\usepackage{textpos}
% \usepackage{amssymb}
% \usepackage{caption}
% \usepackage{subcaption}
% \usepackage{multicol}
% \usepackage{mathrsfs}
\usepackage{float}
\setbeamersize{text margin left=1cm,text margin right=1cm,} 
\usepackage{subcaption}
\usepackage{dsfont}
\usepackage{physics}
\usepackage{mathpazo}
% \usepackage{tikz}
\usepackage{lipsum}
\usepackage{minted}
\usepackage[minted, most]{tcolorbox}
\tcbuselibrary{theorems}
\setbeamertemplate{enumerate items}[default]
\usefonttheme{professionalfonts}

\setlength{\parindent}{0pt} % zero indent

\setbeamertemplate{itemize items}[circle]
\setbeamertemplate{itemize subitem}{$\blacktriangleright$}
\definecolor{nes_dark_purple}{HTML}{4E006B}
\definecolor{nes_dark_orange}{HTML}{EE4F09}
\setbeamercolor{normal text}{fg=nes_dark_purple}\usebeamercolor*{normal text}
\setbeamercolor{math text}{fg=nes_dark_purple}\usebeamercolor[fg]{math text}


\setbeamercolor{title}{fg=nes_dark_purple}
\setbeamercolor{frametitle}{fg=nes_dark_purple}
\setbeamercolor{structure}{fg=nes_dark_purple}
\setbeamercolor{author}{fg=nes_dark_orange}
\setbeamercolor{footline}{fg=nes_dark_purple} 

\setbeamertemplate{footline}[frame number]
\setbeamertemplate{navigation symbols}{}

\definecolor{codegreen}{rgb}{0,0.6,0.2}
\definecolor{codegray}{rgb}{0.5,0.5,0.5}
\definecolor{codepurple}{rgb}{0.58,0,0.82}
\definecolor{backcolour}{rgb}{0.95,0.95,0.92}

\lstdefinestyle{mystyle}{
    backgroundcolor=\color{backcolour},   
    commentstyle=\color{codegreen},
    keywordstyle=\color{magenta},
    numberstyle=\tiny\color{codegray},
    stringstyle=\color{codepurple},
    basicstyle=\ttfamily\footnotesize,
    breakatwhitespace=false,  
    breaklines=true,                 
    captionpos=b,                    
    keepspaces=true,                 
    numbers=left,                    
    numbersep=5pt,                  
    showspaces=false,                
    showstringspaces=false,
    showtabs=false,                  
    tabsize=2
}
\input{../../templates/slides/commands}

\usepackage{tikz}
\usetikzlibrary{arrows.meta, shapes, positioning}
\usepackage{algorithmic}

\titlegraphic{
    \includegraphics[scale = 0.5]{../../templates/slides/logo}
}

\logo{
\begin{tikzpicture}[overlay,remember picture]
\node[below left = 0.2cm] at (current page.30) {
    \includegraphics[width=0.1\textwidth]{../../templates/slides/logo}};
\end{tikzpicture}
}

\newcommand{\highlight}[1]{{\color{nes_dark_orange} #1}}

\title{Treinamento de Redes Neurais} 

\author{
    Eduardo Adame
}

\date{{\color{nes_dark_purple}  \textbf{Redes Neurais}\\[0.5em] 03 de setembro de 2025 }}

\begin{document}

\frame[plain]{\titlepage}
\setcounter{framenumber}{0}

\section{Estratégias de Atualização de Pesos}

\begin{frame}[c]{Como Atualizar os Pesos?}
    \begin{columns}[c]
        \begin{column}{0.5\textwidth}
            \textbf{Após calcular os gradientes:}
            \begin{itemize}
                \item Sabemos a derivada para cada peso
                \item Como exatamente atualizamos?
                \item Com que frequência?
                \begin{itemize}
                    \item Após cada exemplo?
                    \item Após todos os dados?
                    \item Algo intermediário?
                \end{itemize}
            \end{itemize}
            
            \vspace{0.5cm}
            \textbf{Regra de Atualização:}
            $$W_{novo} = W_{antigo} - \alpha \cdot \frac{\partial J}{\partial W}$$
        \end{column}
        \begin{column}{0.5\textwidth}
            \begin{tikzpicture}[scale=0.8]
                % Superfície de erro
                \draw[thick] (0,3) .. controls (1,1) and (3,1.5) .. (5,2);
                \draw[thick] (0,2.5) .. controls (1,0.5) and (3,1) .. (5,1.5);
                \draw[thick] (0,2) .. controls (1,0) and (3,0.5) .. (5,1);
                
                % Ponto atual
                \node[circle, fill=red, inner sep=2pt] (current) at (3,1.25) {};
                
                % Gradiente
                \draw[->, thick, blue] (current) -- (2,2) node[above] {$-\nabla J$};
                
                % Labels
                \node at (2.5, -0.5) {Espaço de Pesos};
                \node[rotate=90] at (-0.7, 2) {Erro};
            \end{tikzpicture}
        \end{column}
    \end{columns}
\end{frame}

\begin{frame}[c]{Descida de Gradiente Clássica (Batch)}
    \begin{columns}[c]
        \begin{column}{0.55\textwidth}
            \textbf{Abordagem Tradicional:}
            \begin{itemize}
                \item Calcular gradiente para \highlight{todo o conjunto}
                \item Dar um passo na direção oposta
                \item Repetir até convergência
            \end{itemize}
            
            \vspace{0.5cm}
            \textbf{Vantagens:}
            \begin{itemize}
                \item[$+$] Cada passo usa toda informação
                \item[$+$] Convergência mais estável
            \end{itemize}
            
            \textbf{Desvantagens:}
            \begin{itemize}
                \item[$-$] Muito lento para datasets grandes
                \item[$-$] Pode ficar preso em mínimos locais
            \end{itemize}
        \end{column}
        \begin{column}{0.45\textwidth}
            \begin{tikzpicture}[scale=0.9]
                % Círculos concêntricos
                \foreach \r in {0.5, 1, 1.5, 2, 2.5} {
                    \draw[gray!50] (0,0) circle (\r);
                }
                
                % Trajetória suave
                \draw[->, ultra thick, red] (2.5,0) -- (2,0) -- (1.5,0) -- (1,0) -- (0.5,0) -- (0.1,0);
                
                % Centro (mínimo)
                \node[circle, fill=green, inner sep=2pt] at (0,0) {};
                
                % Label
                \node[below] at (0,-3) {Convergência Suave};
            \end{tikzpicture}
        \end{column}
    \end{columns}
\end{frame}

\begin{frame}[c]{Descida de Gradiente Estocástica (SGD)}
    \begin{columns}[c]
        \begin{column}{0.55\textwidth}
            \textbf{Abordagem Online:}
            \begin{itemize}
                \item Calcular gradiente para \highlight{um exemplo}
                \item Atualizar imediatamente
                \item Mais passos, menos informados
            \end{itemize}
            
            \vspace{0.5cm}
            \textbf{Vantagens:}
            \begin{itemize}
                \item[$+$] Muito mais rápido
                \item[$+$] Pode escapar de mínimos locais
                \item[$+$] Funciona bem online
            \end{itemize}
            
            \textbf{Desvantagens:}
            \begin{itemize}
                \item[$-$] Convergência ruidosa
                \item[$-$] Precisa de taxa de aprendizado menor
            \end{itemize}
        \end{column}
        \begin{column}{0.45\textwidth}
            \begin{tikzpicture}[scale=0.9]
                % Círculos concêntricos
                \foreach \r in {0.5, 1, 1.5, 2, 2.5} {
                    \draw[gray!50] (0,0) circle (\r);
                }
                
                % Trajetória ziguezague
                \draw[->, ultra thick, red] (2.5,0) -- (1.8,0.7) -- (1.2,-0.5) -- (0.6,0.3) -- (0.2,-0.2) -- (0.1,0.1);
                
                % Centro (mínimo)
                \node[circle, fill=green, inner sep=2pt] at (0,0) {};
                
                % Label
                \node[below] at (0,-3) {Convergência Ruidosa};
            \end{tikzpicture}
        \end{column}
    \end{columns}
\end{frame}

\begin{frame}[c]{Mini-batch: O Melhor dos Dois Mundos}
    \begin{columns}[c]
        \begin{column}{0.55\textwidth}
            \textbf{Abordagem Híbrida:}
            \begin{itemize}
                \item Usar \highlight{pequenos lotes} (16, 32, 64...)
                \item Equilibrio entre velocidade e estabilidade
                \item Padrão na prática moderna
            \end{itemize}
            
            \vspace{0.5cm}
            \textbf{Benefícios:}
            \begin{itemize}
                \item[$\checkmark$] Aproveita paralelização (GPU)
                \item[$\checkmark$] Reduz variância do gradiente
                \item[$\checkmark$] Ainda permite escapar de mínimos
                \item[$\checkmark$] Boa relação custo-benefício
            \end{itemize}
        \end{column}
        \begin{column}{0.45\textwidth}
            \begin{tikzpicture}[scale=0.9]
                % Círculos concêntricos
                \foreach \r in {0.5, 1, 1.5, 2, 2.5} {
                    \draw[gray!50] (0,0) circle (\r);
                }
                
                % Trajetória intermediária
                \draw[->, ultra thick, red] (2.5,0) -- (2,0.2) -- (1.5,-0.1) -- (1,0.1) -- (0.5,-0.05) -- (0.1,0);
                
                % Centro (mínimo)
                \node[circle, fill=green, inner sep=2pt] at (0,0) {};
                
                % Label
                \node[below] at (0,-3) {Convergência Balanceada};
            \end{tikzpicture}
        \end{column}
    \end{columns}
\end{frame}

\begin{frame}[c]{Comparação das Estratégias de Batch}
    \begin{center}
        \begin{tikzpicture}[scale=1.1]
            % Três círculos alvo
            \begin{scope}[shift={(0,0)}]
                \foreach \r in {0.3, 0.6, 0.9, 1.2} {
                    \draw[gray!40] (0,0) circle (\r);
                }
                \draw[->, ultra thick, red, densely dashed] (1.2,0) 
                    .. controls (0.8,0.4) and (0.4,-0.3) .. (0.1,0.1) 
                    .. controls (0,-0.05) .. (0,0);
                \node[below] at (0,-1.6) {\textbf{SGD}};
                \node[below] at (0,-2) {\small (Batch = 1)};
                \node[circle, fill=green, inner sep=1.5pt] at (0,0) {};
            \end{scope}
            
            \begin{scope}[shift={(3.5,0)}]
                \foreach \r in {0.3, 0.6, 0.9, 1.2} {
                    \draw[gray!40] (0,0) circle (\r);
                }
                \draw[->, ultra thick, red] (1.2,0) -- (0.8,0.1) -- (0.4,-0.05) -- (0.1,0) -- (0,0);
                \node[below] at (0,-1.6) {\textbf{Mini-batch}};
                \node[below] at (0,-2) {\small (Batch = 32)};
                \node[circle, fill=green, inner sep=1.5pt] at (0,0) {};
            \end{scope}
            
            \begin{scope}[shift={(7,0)}]
                \foreach \r in {0.3, 0.6, 0.9, 1.2} {
                    \draw[gray!40] (0,0) circle (\r);
                }
                \draw[->, ultra thick, red] (1.2,0) -- (0.9,0) -- (0.6,0) -- (0.3,0) -- (0,0);
                \node[below] at (0,-1.6) {\textbf{Batch Completo}};
                \node[below] at (0,-2) {\small (Batch = N)};
                \node[circle, fill=green, inner sep=1.5pt] at (0,0) {};
            \end{scope}
            
            % Barra de gradiente
            \draw[thick] (0,-3) -- (7,-3);
            \fill[blue!60] (0,-3.2) rectangle (1,-2.8);
            \fill[blue!60!green!60] (3,-3.2) rectangle (4,-2.8);
            \fill[red!60] (6,-3.2) rectangle (7,-2.8);
            
            \node[below] at (0,-3.3) {\small Rápido};
            \node[below] at (7,-3.3) {\small Preciso};
        \end{tikzpicture}
    \end{center}
\end{frame}

\begin{frame}[c]{Terminologia: Época e Iteração}
    \begin{columns}[c]
        \begin{column}{0.5\textwidth}
            \textbf{Época (Epoch):}
            \begin{itemize}
                \item Uma passada completa pelos dados
                \item Todos exemplos vistos uma vez
                \item Métrica comum de progresso
            \end{itemize}
            
            \vspace{0.5cm}
            \textbf{Passos por Época:}
            \begin{itemize}
                \item \highlight{Batch completo}: 1 passo
                \item \highlight{SGD}: N passos
                \item \highlight{Mini-batch}: N/batch\_size passos
            \end{itemize}
        \end{column}
        \begin{column}{0.5\textwidth}
            \begin{tikzpicture}[scale=0.8]
                % Dataset cilindro
                \draw[fill=blue!30] (-1.5,0) -- (-1.5,-2) -- (1.5,-2) -- (1.5,0);
                \draw[fill=blue!20] (0,0) ellipse (1.5 and 0.5);
                
                % Divisões do dataset
                \foreach \y in {-0.6, -1, -1.4, -1.8} {
                    \draw[white, thick] (-1.5,\y) -- (1.5,\y);
                    }
                    
                \draw[fill=blue!20] (-1.5,-2) arc(180:360:1.5 and 0.5);
                % Labels
                \node at (0,0.8) {\textbf{Dataset Completo}};
                \node[right] at (1.7,-0.3) {Batch 1};
                \node[right] at (1.7,-0.7) {Batch 2};
                \node[right] at (1.7,-1.1) {Batch 3};
                \node[right] at (1.7,-1.5) {Batch 4};
                \node[right] at (1.7,-1.9) {Batch 5};
                
            \end{tikzpicture}
        \end{column}
    \end{columns}
    
    \vspace{0.5cm}
    \begin{center}
        \highlight{Dica:} Embaralhar os dados após cada época melhora a convergência!
    \end{center}
\end{frame}

\begin{frame}[c]{Fluxo de Treinamento: Visão Prática}
    \begin{center}
        \begin{tikzpicture}[scale=0.9, node distance=1.5cm]
            % Nós do fluxograma
            \node[rectangle, draw, fill=cyan!20, rounded corners] (data) {Carregar Dados};
            \node[rectangle, draw, fill=orange!20, rounded corners, below of=data] (shuffle) {Embaralhar};
            \node[rectangle, draw, fill=yellow!20, rounded corners, below of=shuffle] (batch) {Criar Mini-batches};
            \node[rectangle, draw, fill=green!20, rounded corners, below of=batch] (forward) {Forward Pass};
            \node[rectangle, draw, fill=red!20, rounded corners, below of=forward] (loss) {Calcular Perda};
            \node[rectangle, draw, fill=purple!20, rounded corners, right=2cm of loss] (backward) {Backward Pass};
            \node[rectangle, draw, fill=blue!20, rounded corners, above of=backward] (update) {Atualizar Pesos};
            
            % Setas
            \draw[->, thick] (data) -- (shuffle);
            \draw[->, thick] (shuffle) -- (batch);
            \draw[->, thick] (batch) -- (forward);
            \draw[->, thick] (forward) -- (loss);
            \draw[->, thick] (loss) -- (backward);
            \draw[->, thick] (backward) -- (update);
            \draw[->, thick] (update) -- ++(0,1.5) -| (forward);
            
            % Loop de época
            \draw[->, thick, dashed, blue] (update) -- ++(2,0) |- (shuffle);
            
            % Anotações
            \node[right] at (4.5,0) {\small Repetir para};
            \node[right] at (4.5,-0.4) {\small cada batch};
            
            \node[right, blue] at (4.5,-3) {\small Nova época};
        \end{tikzpicture}
    \end{center}
\end{frame}

\begin{frame}[c, fragile]{Implementação em Keras/TensorFlow}
    \begin{code}[Estrutura Típica de Código]{python}
# 1. Construir o modelo
model = Sequential()
model.add(Dense(64, activation='relu'))
model.add(Dense(32, activation='relu'))
model.add(Dense(10, activation='softmax'))
# 2. Compilar
model.compile(
    optimizer='adam',
    loss='categorical_crossentropy',
    metrics=['accuracy']
)
# 3. Treinar
history = model.fit(
    X_train, y_train,
    batch_size=32,      # Mini-batch
    epochs=50,          # Épocas
    validation_split=0.2
)
    \end{code}
\end{frame}

\begin{frame}[c]{Implementação em Keras/TensorFlow}
        \textbf{Parâmetros Importantes:}
            \begin{itemize}
                \item \highlight{batch\_size}: Tamanho do mini-batch
                \item \highlight{epochs}: Número de épocas
                \item \highlight{shuffle}: Embaralhar dados (padrão: True)
                \item \highlight{validation\_split}: Proporção para validação
            \end{itemize}
            
            \vspace{0.5cm}
            \textbf{Valores Típicos:}
            \begin{itemize}
                \item Batch size: 16, 32, 64, 128
                \item Épocas: 10-100 (depende do problema)
                \item Learning rate: 0.001 (Adam), 0.01 (SGD)
            \end{itemize}
\end{frame}

\begin{frame}[c]{Normalização de Entrada}
    \begin{columns}[c]
        \begin{column}{0.5\textwidth}
            \textbf{Por que normalizar?}
            \begin{itemize}
                \item Acelera convergência
                \item Evita saturação de neurônios
                \item Estabiliza o treinamento
                \item Permite learning rates maiores
            \end{itemize}
            
            \vspace{0.5cm}
            \textbf{Métodos Comuns:}
            \begin{itemize}
                \item \highlight{Min-Max}: $[0, 1]$ ou $[-1, 1]$
                \item \highlight{Padronização}: média 0, desvio 1
                \item \highlight{Normalização L2}: vetores unitários
            \end{itemize}
        \end{column}
        \begin{column}{0.5\textwidth}
            \textbf{Fórmulas:}
            
            \vspace{0.3cm}
            Min-Max para $[0,1]$:
            $$x' = \frac{x - x_{min}}{x_{max} - x_{min}}$$
            
            \vspace{0.3cm}
            Padronização (Z-score):
            $$x' = \frac{x - \mu}{\sigma}$$
        \end{column}
    \end{columns}
\end{frame}


\begin{frame}[c]{Ativação Softmax para Classificação Multiclasse}
    \begin{columns}[c]
        \begin{column}{0.5\textwidth}
            \textbf{Extensão da Sigmoid:}
            \begin{itemize}
                \item Saída: vetor de probabilidades
                \item Soma sempre igual a 1
                \item Interpretação probabilística
            \end{itemize}
            
            \textbf{Fórmula:}
            $$\text{softmax}(z_i) = \frac{e^{z_i}}{\sum_{k=1}^{K} e^{z_k}}$$
            
            \textbf{Com Cross-Entropy:}
            $$J = -\sum_{i=1}^{n} y_i \log(\hat{y}_i)$$
        \end{column}
        \begin{column}{0.5\textwidth}
            \textbf{Exemplo Visual:}
            \begin{tikzpicture}[scale=0.8]
                % Entrada
                \node[rectangle, draw] (z1) at (0,2) {$z_1 = 2.0$};
                \node[rectangle, draw] (z2) at (0,0.5) {$z_2 = 1.0$};
                \node[rectangle, draw] (z3) at (0,-1) {$z_3 = 0.1$};
                
                % Softmax
                \node[rectangle, draw, fill=yellow!30] (soft) at (3,0.5) {Softmax};
                
                % Saída
                \node[rectangle, draw, fill=green!30] (p1) at (6,2) {$p_1 = 0.66$};
                \node[rectangle, draw, fill=green!20] (p2) at (6,0.5) {$p_2 = 0.24$};
                \node[rectangle, draw, fill=green!10] (p3) at (6,-1) {$p_3 = 0.10$};
                
                % Conexões
                \draw[->] (z1) -- (soft);
                \draw[->] (z2) -- (soft);
                \draw[->] (z3) -- (soft);
                \draw[->] (soft) -- (p1);
                \draw[->] (soft) -- (p2);
                \draw[->] (soft) -- (p3);
                
            \end{tikzpicture}
            
            \vspace{0.3cm}
            \begin{center}
                \highlight{Classe prevista: Argmax = Classe 1}
            \end{center}
        \end{column}
    \end{columns}
\end{frame}

\begin{frame}[c]{Resumo: Detalhes Práticos do Treinamento}
    \begin{columns}[c]
        \begin{column}{0.5\textwidth}
            \textbf{Decisões Importantes:}
            \begin{itemize}
                \item \highlight{Estratégia de batch}
                \begin{itemize}
                    \item Mini-batch (padrão)
                    \item Tamanho: 32-128
                \end{itemize}
                \item \highlight{Número de épocas}
                \begin{itemize}
                    \item Monitorar validação
                    \item Early stopping
                \end{itemize}
                \item \highlight{Normalização de entrada}
                \begin{itemize}
                    \item Sempre normalizar!
                    \item Método depende dos dados
                \end{itemize}
            \end{itemize}
        \end{column}
        \begin{column}{0.5\textwidth}
            \textbf{Boas Práticas:}
            \begin{itemize}
                \item[$\checkmark$] Embaralhar dados a cada época
                \item[$\checkmark$] Usar validação para monitorar
                \item[$\checkmark$] Começar com learning rate pequeno
                \item[$\checkmark$] Visualizar curvas de aprendizado
                \item[$\checkmark$] Salvar checkpoints do modelo
            \end{itemize}
            
            \vspace{0.5cm}
            \textbf{Próximos Tópicos:}
            \begin{itemize}
                \item Otimizadores avançados (Adam, RMSprop)
                \item Regularização (Dropout, L2)
                \item Learning rate scheduling
            \end{itemize}
        \end{column}
    \end{columns}
\end{frame}

\begin{frame}[c, noframenumbering, plain]
    \frametitle{~}
    \vfill
    \begin{center}
        {\Huge Obrigado!}\vspace{1.5em}\\
        {\Large \highlight{Dúvidas?}}\\
    \end{center}
    \vfill
    \begin{center}
        {\small Vamos para as implementações!}
    \end{center}
\end{frame}

\end{document}