\documentclass[xcolor=dvipsnames,t,aspectratio=169]{beamer} %t para ficar alinhado no topo do slide

\usecolortheme{rose}
\usecolortheme{dolphin}
\usetheme{Boadilla}

% Encoding packages MUST come first
\usepackage[utf8]{inputenc}
\usepackage[brazil]{babel}
\usepackage[T1]{fontenc}

\usepackage{amsfonts}
\usepackage{amsmath}
\usepackage{amssymb}
% \usepackage{latexsym}
% \usepackage{graphicx}
% \usepackage{listings} 
\usepackage{xcolor}
\usepackage{amsthm}
\usepackage{url}
\usepackage{textpos}
% \usepackage{amssymb}
% \usepackage{caption}
% \usepackage{subcaption}
% \usepackage{multicol}
% \usepackage{mathrsfs}
\usepackage{float}
\setbeamersize{text margin left=1cm,text margin right=1cm,} 
\usepackage{subcaption}
\usepackage{dsfont}
\usepackage{physics}
\usepackage{mathpazo}
% \usepackage{tikz}
\usepackage{lipsum}
\usepackage{minted}
\usepackage[minted, most]{tcolorbox}

\input{../../templates/slides/settings}

\input{../../templates/slides/commands}

\titlegraphic{
    \includegraphics[scale = 0.5]{../../templates/slides/logo}
}

\logo{
\begin{tikzpicture}[overlay,remember picture]
\node[left=1.1cm, below=0.2cm] at (current page.30){
    \includegraphics[width=0.1\textwidth]{../../templates/slides/logo}};
\end{tikzpicture}
}

\newcommand{\highlight}[1]{{\color{nes_dark_orange} #1}}

\title{Revisão de Machine Learning e Descida de Gradiente} 

\author{
    Eduardo Adame
}

\date{{\color{nes_dark_purple}  \textbf{Redes Neurais}\\[0.5em] 13 de agosto de 2025 }}

\begin{document}

\frame[plain]{\titlepage}
\setcounter{framenumber}{0}


\section{Sobre mim}
\begin{frame}[c]{Sobre mim}
    \begin{columns}[c]
        \begin{column}{0.7\textwidth}
            \begin{itemize}
                \item \textbf{Nome:} Eduardo Adame
                \item \textbf{Formação:}
                \begin{itemize}
                    \item Mestrando em Matemática Aplicada e Ciência de Dados (FGV EMAp); 
                    \item Bacharel em Ciência de Dados e Inteligência Artificial (FGV EMAp).
                \end{itemize}
                \item \textbf{Experiência:} Diversos cursos produzidos/ministrados, ex-consultor por 2 anos e meio no Banco Mundial.
                \item \textbf{Pesquisa:} Estatística bayesiana; quantificação de incerteza, modelos não-paramétricos, inferência exata.
                \item \textbf{Contato:} eadamesalles@gmail.com
            \end{itemize}
        \end{column}
        \begin{column}{0.3\textwidth}
            \centering
            \includegraphics[width=0.9\textwidth]{images/me.jpg}
        \end{column}
    \end{columns}
\end{frame}

\section{Sobre o curso}
\begin{frame}[c]{Sobre o curso}
    \begin{display}[Objetivos]
        \begin{itemize}
            \item Fornecer \highlight{fundamentos sólidos} em redes neurais e aprendizado profundo
            \item Desenvolver \highlight{intuição} sobre funcionamento dos algoritmos
            \item Preparar estudantes para \highlight{projetos reais} de deep learning
        \end{itemize}
    \end{display}
    
    \begin{display}[Metodologia]
        \begin{itemize}
            \item Aulas teóricas (slides próprios + outros materiais)
            \item Exercícios práticos em \highlight{Jupyter notebooks}
            \item Implementação de algoritmos \highlight{do zero}
            \item Uso de frameworks modernos (\highlight{Keras/TensorFlow})
        \end{itemize}
    \end{display}
\end{frame}

\begin{frame}[c]{Estrutura do curso - 12 semanas}
    \begin{columns}[c]
        \begin{column}{0.5\textwidth}
            \textbf{Fundamentos (Semanas 1-8):}
            \begin{enumerate}
                \item Revisão ML + Gradiente
                \item Introdução às Redes Neurais
                \item Retropropagação
                \item Keras + Primeiras Redes
                \item Regularização + Dropout
                \item Introdução às CNNs
                \item Revisão + Dúvidas
                \item Avaliação 1
            \end{enumerate}
        \end{column}
        \begin{column}{0.5\textwidth}
            \textbf{Tópicos Avançados (Semanas 8-16):}
            \begin{enumerate}
                \setcounter{enumi}{8}
                \item Transfer Learning
                \item Arquiteturas de CNNs
                \item Técnicas Avançadas (Data Aug.)
                \item Texto + Word Vectors
                \item RNNs para Sequências
                \item LSTMs + Aplicações
                \item Revisão + Dúvidas
                \item Avaliação 2
            \end{enumerate}
        \end{column}
    \end{columns}
\end{frame}

\begin{frame}[c]{Logística do curso}
    \begin{columns}[c]
        \begin{column}{0.5\textwidth}
            \textbf{Horário e Local:}
            \begin{itemize}
                \item Quartas-feiras, 15h-18h
                \item Primeira aula: 13/08/2025
                \item Última aula: 26/11/2025
            \end{itemize}
            
            \textbf{Material:}
            \begin{itemize}
                \item Slides (PDF)
                \item Notebooks interativos
                \item Notas de aula
                \item Repositório GitHub
            \end{itemize}
        \end{column}
        \begin{column}{0.5\textwidth}
            \textbf{Pré-requisitos:}
            \begin{itemize}
                \item Python
                \item Álgebra linear
                \item Cálculo diferencial
                \item ML básico 
            \end{itemize}
            
            \textbf{Avaliação:}
            \begin{itemize}
                \item Exercícios semanais
                \item Projeto final 
            \end{itemize}
        \end{column}
    \end{columns}
\end{frame}

\section{Revisão de Machine Learning}
\begin{frame}[c]{Por que revisar Machine Learning?}
    \begin{itemize}
        \item Redes neurais são uma \highlight{extensão} dos métodos de ML tradicionais
        \item Muitos conceitos fundamentais são compartilhados:
        \begin{itemize}
            \item \highlight{Função de custo/perda}
            \item \highlight{Otimização via gradiente}
            \item Overfitting vs. underfitting
            \item Validação cruzada
        \end{itemize}
        \item Entender o \highlight{gradiente} é crucial para entender redes neurais
        \item Comparação de performance: NN vs. métodos tradicionais
    \end{itemize}
    
    \vspace{0.5cm}
    \begin{display}[Objetivo da Semana 1]
        Implementar \highlight{descida de gradiente} do zero e entender sua dinâmica
    \end{display}
\end{frame}



\begin{frame}[c]{O que é Machine Learning?}
    \begin{center}
        \Large
        \highlight{Machine Learning permite que computadores aprendam e façam inferências a partir de dados.}
    \end{center}
    
    \vspace{1cm}
    
    \begin{columns}[c]
        \begin{column}{0.5\textwidth}
            \textbf{Programação Tradicional:}
            \begin{itemize}
                \item Regras explícitas
                \item Comportamento determinístico
                \item Limitado a cenários previstos
            \end{itemize}
        \end{column}
        \begin{column}{0.5\textwidth}
            \textbf{Machine Learning:}
            \begin{itemize}
                \item Aprende padrões dos dados
                \item Generaliza para novos casos
                \item Adapta-se a complexidade
            \end{itemize}
        \end{column}
    \end{columns}
\end{frame}

\begin{frame}[c]{Tipos de Machine Learning}
    \begin{columns}[c]
        \begin{column}{0.5\textwidth}
            \begin{display}[Aprendizado Supervisionado]
                \begin{itemize}
                    \item Dados com \highlight{respostas conhecidas}
                    \item Objetivo: prever respostas para novos dados
                    \item Exemplos:
                    \begin{itemize}
                        \item Classificação de e-mails (spam/não spam)
                        \item Previsão de preços de imóveis
                        \item Diagnóstico médico
                    \end{itemize}
                \end{itemize}
            \end{display}
        \end{column}
        \begin{column}{0.5\textwidth}
            \begin{display}[Aprendizado Não-Supervisionado]
                \vspace{.4cm}
                \begin{itemize}
                    \item Dados \highlight{sem respostas}
                    \item Objetivo: encontrar estruturas/padrões
                    \item Exemplos:
                    \begin{itemize}
                        \item Segmentação de clientes
                        \item Detecção de anomalias
                        \item Redução de dimensionalidade
                    \end{itemize}
                \end{itemize}
                \vspace{.4cm}
            \end{display}
        \end{column}
    \end{columns}
\end{frame}

\begin{frame}[c]{Tipos de Aprendizado Supervisionado}
    \begin{columns}[c]
        \begin{column}{0.5\textwidth}
            \begin{display}[Regressão]
                \vspace{.2cm}
                \begin{itemize}
                    \item Saída: valor \highlight{contínuo/numérico}
                    \item Exemplos:
                    \begin{itemize}
                        \item Preço de uma casa: R\$ 450.000
                        \item Temperatura amanhã: 23.5°C
                        \item Vendas do próximo mês: 1.250 unidades
                    \end{itemize}
                    \item Métricas: RMSE, MAE, R²
                \end{itemize}
                \vspace{.2cm}
            \end{display}
        \end{column}
        \begin{column}{0.5\textwidth}
            \begin{display}[Classificação]
                \begin{itemize}
                    \item Saída: \highlight{categoria/classe}
                    \item Exemplos:
                    \begin{itemize}
                        \item E-mail: spam ou não spam
                        \item Imagem: gato, cachorro ou pássaro
                        \item Transação: fraudulenta ou legítima
                    \end{itemize}
                    \item Métricas: Acurácia, Precisão, Recall, F1
                \end{itemize}
            \end{display}
        \end{column}
    \end{columns}
\end{frame}

\begin{frame}[c]{Vocabulário de Machine Learning}
    \begin{itemize}
        \item \textbf{Target (Alvo):} Variável que queremos prever
        \begin{itemize}
            \item Sinônimos: Resposta, Output, Variável Dependente, Labels
        \end{itemize}
        
        \item \textbf{Features (Características):} Variáveis usadas para fazer a previsão
        \begin{itemize}
            \item Sinônimos: Preditores, Input, Variáveis Independentes, Atributos
        \end{itemize}
        
        \item \textbf{Example (Exemplo):} Um único ponto de dados
        \begin{itemize}
            \item Sinônimos: Observação, Registro, Instância, Linha
        \end{itemize}
        
        \item \textbf{Label (Rótulo):} Valor do target para um exemplo específico
        \begin{itemize}
            \item Sinônimos: Resposta, Valor-y, Categoria
        \end{itemize}
    \end{itemize}
\end{frame}

\begin{frame}[c]{Pipeline de Aprendizado Supervisionado}
    \begin{center}
        \begin{tikzpicture}[
            box/.style={rectangle, draw, minimum width=3cm, minimum height=1cm, align=center},
            arrow/.style={->, thick}
        ]
        % Primeira linha
        \node[box] (data) at (0,3) {Dados com\\respostas};
        \node (plus1) at (2.5,3) {+};
        \node[box] (model1) at (5,3) {Modelo};
        \node (fit) at (7.5,3) {$\xrightarrow{\text{Ajustar}}$};
        \node[box, fill=gray!30] (trained) at (10.5,3) {MODELO\\TREINADO};
        
        % Segunda linha
        \node[box] (newdata) at (0,0) {Dados sem\\respostas};
        \node (plus2) at (2.5,0) {+};
        \node[box, fill=gray!30] (model2) at (5,0) {MODELO\\TREINADO};
        \node (predict) at (7.5,0) {$\xrightarrow{\text{Predizer}}$};
        \node[box] (predictions) at (10.5,0) {Previsões};
        
        % Seta conectando
        \draw[arrow, dashed] (trained) -- (model2);
        
        % Labels
        \node at (5,4.5) {\textbf{Fase de Treinamento}};
        \node at (5,-1.5) {\textbf{Fase de Predição}};
        \end{tikzpicture}
    \end{center}
\end{frame}

\begin{frame}[c]{Dados de Treino e Teste}
    \begin{columns}[c]
        \begin{column}{0.5\textwidth}
            \textbf{Por que dividir os dados?}
            \begin{itemize}
                \item Avaliar \highlight{generalização} do modelo
                \item Detectar \highlight{overfitting}
                \item Simular cenário real de uso
            \end{itemize}
            
            \vspace{0.5cm}
            \textbf{Divisão típica:}
            \begin{itemize}
                \item 70-80\% para treino
                \item 20-30\% para teste
                \item Opcional: conjunto de validação
            \end{itemize}
        \end{column}
        \begin{column}{0.5\textwidth}
            \begin{display}[Processo]
                \begin{enumerate}
                    \item \textbf{Treinar} com dados de treino
                    \item \textbf{Prever} nos dados de teste
                    \item \textbf{Comparar} previsões com valores reais
                    \item \textbf{Calcular} métricas de erro
                \end{enumerate}
            \end{display}
            
            \begin{attention}[Importante]
                NUNCA use dados de teste durante o treinamento!
            \end{attention}
        \end{column}
    \end{columns}
\end{frame}

\begin{frame}[c]{Underfitting vs. Overfitting}
    \begin{columns}[c]
        \begin{column}{0.33\textwidth}
            \centering
            \textbf{Underfitting}\\
            \vspace{0.1cm}
            \begin{tikzpicture}[scale=0.8]
                % Pontos de dados
                \foreach \x/\y in {0.5/1.2, 1/1.5, 1.5/2.1, 2/2.8, 2.5/3.2, 3/3.8, 3.5/4.1}
                    \fill (\x,\y) circle (2pt);
                % Linha reta (underfitting)
                \draw[thick, blue] (0,0.8) -- (4,4.5);
                \draw[->] (0,0) -- (4.5,0);
                \draw[->] (0,0) -- (0,5);
            \end{tikzpicture}
            \begin{itemize}
                \small
                \item Modelo muito simples
                \item Alto bias
                \item Baixa variância
                \item Erro alto no treino e teste
            \end{itemize}
        \end{column}
        \begin{column}{0.33\textwidth}
            \centering
            \textbf{Na medida certa}\\
            \vspace{0.1cm}
            \begin{tikzpicture}[scale=0.8]
                % Pontos de dados
                \foreach \x/\y in {0.5/1.2, 1/1.5, 1.5/2.1, 2/2.8, 2.5/3.2, 3/3.8, 3.5/4.1}
                    \fill (\x,\y) circle (2pt);
                % Curva suave
                \draw[thick, green!60!black] plot[smooth,tension=0.6] coordinates {(0.3,1) (1,1.5) (2,2.8) (3,3.8) (3.7,4.2)};
                \draw[->] (0,0) -- (4.5,0);
                \draw[->] (0,0) -- (0,5);
            \end{tikzpicture}
            \begin{itemize}
                \small
                \item Complexidade adequada
                \item Balanceado
                \item Boa generalização
                \item Erro baixo no treino e teste
            \end{itemize}
        \end{column}
        \begin{column}{0.33\textwidth}
            \centering
            \textbf{Overfitting}\\
            \vspace{0.1cm}
            \begin{tikzpicture}[scale=0.8]
                % Pontos de dados
                \foreach \x/\y in {0.5/1.2, 1/1.5, 1.5/2.1, 2/2.8, 2.5/3.2, 3/3.8, 3.5/4.1}
                    \fill (\x,\y) circle (2pt);
                % Curva complexa
                \draw[thick, red] plot[smooth,tension=0.3] coordinates {(0.5,1.2) (0.8,1.8) (1,1.5) (1.3,1.9) (1.5,2.1) (1.8,2.5) (2,2.8) (2.3,3.0) (2.5,3.2) (2.8,3.5) (3,3.8) (3.2,3.9) (3.5,4.1)};
                \draw[->] (0,0) -- (4.5,0);
                \draw[->] (0,0) -- (0,5);
            \end{tikzpicture}
            \begin{itemize}
                \small
                \item Modelo muito complexo
                \item Baixo bias
                \item Alta variância
                \item Erro baixo no treino, alto no teste
            \end{itemize}
        \end{column}
    \end{columns}
\end{frame}

\begin{frame}[c]{Métricas de Avaliação Básicas}
    \begin{columns}[c]
        \begin{column}{0.5\textwidth}
            \textbf{Para Regressão:}
            \begin{itemize}
                \item RMSE (Root Mean Square Error) $$\text{RMSE} = \sqrt{\frac{1}{n}\sum_{i=1}^{n}(y_i - \hat{y}_i)^2}$$
                \item MAE (Mean Absolute Error) $$\text{MAE} = \frac{1}{n}\sum_{i=1}^{n}|y_i - \hat{y}_i|$$
            \end{itemize}
        \end{column}
        \begin{column}{0.5\textwidth}
            \textbf{Para Classificação:}
            \begin{itemize}
                \item Acurácia $$\text{Acc} = \frac{\text{Previsões Corretas}}{\text{Total de Previsões}}$$
                \item Precisão e Recall $$\text{Prec} = \frac{TP}{TP + FP}, \quad \text{Rec} = \frac{TP}{TP + FN}$$
            \end{itemize}
        \end{column}
    \end{columns}
\end{frame}

\begin{frame}[c]{Problema de Regressão Linear}
    \textbf{Modelo:} $y = \beta_0 + \beta_1 x_1 + \beta_2 x_2 + \varepsilon$
    \vspace{.2cm}
    \begin{columns}[c]
        \begin{column}{0.5\textwidth}
            \textbf{Dados conhecidos:}
            \begin{itemize}
                \item $\beta_0 = 1.5$ (intercepto)
                \item $\beta_1 = 2$ (coef. de $x_1$)
                \item $\beta_2 = 5$ (coef. de $x_2$)
                \item $x_1, x_2 \sim \text{Uniforme}[0,10]$
                \item $\varepsilon \sim \mathcal{N}(0, 0.5^2)$
            \end{itemize}
        \end{column}
        \begin{column}{0.5\textwidth}
            \textbf{Objetivo:}
            \begin{itemize}
                \item Estimar $\hat{\beta_0}, \hat{\beta_1}, \hat{\beta_2}$
                \item Métodos a comparar:
                \begin{enumerate}
                    \item Solução analítica
                    \item Scikit-learn
                    \item Descida de gradiente
                \end{enumerate}
            \end{itemize}
        \end{column}
    \end{columns}
    
    \vspace{0.5cm}
    \begin{display}[Função de Custo (MSE)]
        \begin{equation}
            J(\boldsymbol{\beta}) = \frac{1}{2n}\sum_{i=1}^{n}(y_i - \hat{y_i})^2 = \frac{1}{2n}\sum_{i=1}^{n}(y_i - \boldsymbol{x_i}^T\boldsymbol{\beta})^2
        \end{equation}
    \end{display}
\end{frame}

\begin{frame}[c]{Solução Analítica vs. Numérica}
    \begin{columns}[c]
        \begin{column}{0.5\textwidth}
            \textbf{Solução Analítica:}
            \begin{equation}
                \hat{\boldsymbol{\beta}} = (X^TX)^{-1}X^T\mathbf{y}
            \end{equation}
            
            \textbf{Vantagens:}
            \begin{itemize}
                \item Solução exata
                \item Uma única operação
                \item Computacionalmente rápida (problemas pequenos)
            \end{itemize}
            
            \textbf{Desvantagens:}
            \begin{itemize}
                \item Inversão de matriz: $O(n^3)$
                \item Problemas numéricos
                \item Não escala para grandes dados
            \end{itemize}
        \end{column}
        \begin{column}{0.5\textwidth}
            \textbf{Descida de Gradiente:}
            \begin{equation}
                \boldsymbol{\beta}^{(t+1)} = \boldsymbol{\beta}^{(t)} - \alpha \nabla J(\boldsymbol{\beta}^{(t)})
            \end{equation}
            
            \textbf{Vantagens:}
            \begin{itemize}
                \item Escala para grandes dados
                \item Aplicável a funções não-lineares
                \item Base para redes neurais
                \item Flexibilidade
            \end{itemize}
            
            \textbf{Desvantagens:}
            \begin{itemize}
                \item Solução aproximada
                \item Hiperparâmetros ($\alpha$, iterações)
                \item Convergência pode ser lenta
            \end{itemize}
        \end{column}
    \end{columns}
\end{frame}

\section{Descida de Gradiente}

\begin{frame}[c]{Intuição Geométrica do Gradiente}
    \begin{columns}[c]
        \begin{column}{0.5\textwidth}
            \textbf{Imagine:} Você está no topo de uma montanha e quer descer até o vale mais baixo.
            
            \vspace{0.5cm}
            \textbf{Estratégia:}
            \begin{enumerate}
                \item Olhe ao seu redor
                \item Encontre a direção mais íngreme
                \item Dê um passo nessa direção
                \item Repita até chegar ao vale
            \end{enumerate}
            
            \vspace{0.5cm}
            O \highlight{gradiente} aponta para a direção de maior subida. Logo, o \highlight{gradiente negativo} aponta para a maior descida!
        \end{column}
        \begin{column}{0.5\textwidth}
            \centering
            \begin{tikzpicture}[scale=1.2]
                % Função de custo (parábola)
                \draw[thick, blue!60] plot[smooth,domain=-2:2] (\x, {0.5*\x*\x + 0.5});
                
                % Pontos da trajetória
                \fill[red] (1.5, 1.625) circle (2pt) node[above] {Início};
                \draw[->, red, thick] (1.5, 1.625) -- (1.0, 1.3);
                \fill[red] (1.0, 1.0) circle (2pt);
                \draw[->, red, thick] (1.0, 1.0) -- (0.5, 0.8);
                \fill[red] (0.5, 0.625) circle (2pt);
                \draw[->, red, thick] (0.5, 0.625) -- (0.1, 0.55);
                \fill[green] (0, 0.5) circle (2pt) node[below] {Mínimo};
                
                % Eixos
                \draw[->] (-2.5, 0) -- (2.5, 0) node[right] {$\beta$};
                \draw[->] (0, 0) -- (0, 3) node[above] {$J(\beta)$};
                
                % Label
                \node at (0, -0.8) {Trajetória da Descida de Gradiente};
            \end{tikzpicture}
        \end{column}
    \end{columns}
\end{frame}

% Seção: Descida de Gradiente
\section{Descida de Gradiente}
\begin{frame}
    \frametitle{Algoritmo de Descida de Gradiente}
    \vspace{.25cm}
    \begin{display}[Algoritmo Básico]
        \begin{enumerate}
            \item \textbf{Inicialização:} $\boldsymbol{\beta}^{(0)} = \text{valores iniciais}$
            \item \textbf{Para} $t = 0, 1, 2, \ldots$ \textbf{até convergência:}
            \begin{enumerate}
                \item Calcular predições: $\hat{\mathbf{y}} = X\boldsymbol{\beta}^{(t)}$
                \item Calcular erro: $\mathbf{e} = \mathbf{y} - \hat{\mathbf{y}}$
                \item Calcular gradiente: $\nabla J = -\frac{1}{n}X^T\mathbf{e}$
                \item Atualizar parâmetros: $\boldsymbol{\beta}^{(t+1)} = \boldsymbol{\beta}^{(t)} - \alpha \nabla J$
            \end{enumerate}
            \item \textbf{Retornar:} $\boldsymbol{\beta}^{(T)}$
        \end{enumerate}
    \end{display}
    
    \vspace{0.3cm}
    \textbf{Gradiente do MSE:}
    \begin{equation}
        \frac{\partial J}{\partial \beta_j} = -\frac{1}{n}\sum_{i=1}^{n}(y_i - \hat{y_i}) \cdot x_{ij}
    \end{equation}
\end{frame}

% Seção: Gradiente Estocástico
\section{Gradiente Estocástico (SGD)}
\begin{frame}
    \frametitle{Batch vs. Stochastic Gradient Descent}
    
    \begin{columns}[c]
        \begin{column}{0.5\textwidth}
            \textbf{Batch Gradient Descent:}
            \begin{itemize}
                \item Usa \textbf{todos} os dados por iteração
                \item Gradiente exato
                \item Convergência suave
                \item Computacionalmente caro
            \end{itemize}
            
            \begin{equation}
                \nabla J = \frac{1}{n}\sum_{i=1}^{n} \nabla J_i
            \end{equation}
        \end{column}
        \begin{column}{0.5\textwidth}
            \textbf{Stochastic Gradient Descent:}
            \begin{itemize}
                \item Usa \textbf{uma amostra} por iteração
                \item Gradiente aproximado
                \item Convergência com ruído
                \item Computacionalmente eficiente
            \end{itemize}
            
            \begin{equation}
                \nabla J \approx \nabla J_i
            \end{equation}
        \end{column}
    \end{columns}

    \begin{display}[Vantagens do SGD]
        \begin{itemize}
            \item Escala para milhões de amostras
            \item Pode escapar de mínimos locais (ruído = benefício!)
            \item Convergência online (dados chegando continuamente)
        \end{itemize}
    \end{display}
\end{frame}

\begin{frame}
    \frametitle{SGD: Ordem dos Dados Importa!}
    
    \begin{columns}[c]
        \begin{column}{0.5\textwidth}
            \textbf{Sem embaralhamento:}
            \begin{itemize}
                \item Mesma ordem a cada época
                \item Padrões sistemáticos
                \item Convergência tendenciosa
                \item Ciclos na trajetória
            \end{itemize}
            
            \textbf{Problema:} Se os dados têm ordem específica, o algoritmo pode ``memorizar'' essa ordem.
        \end{column}
        \begin{column}{0.5\textwidth}
            \textbf{Com embaralhamento:}
            \begin{itemize}
                \item Ordem aleatória a cada época
                \item Reduz viés sistemático
                \item Convergência mais robusta
                \item Melhor exploração
            \end{itemize}
            
            \textbf{Solução:} Embaralhar dados no início de cada época.
        \end{column}
    \end{columns}
    
    \vspace{0.5cm}
    \begin{attention}[Exercício no Notebook]
        Implemente SGD com e sem embaralhamento e compare as trajetórias!
    \end{attention}
\end{frame}


\begin{frame}
    \frametitle{Análises a Fazer}
    
    \begin{enumerate}
        \item \textbf{Comparação de Métodos:}
        \begin{itemize}
            \item Os três métodos convergem para o mesmo resultado?
            \item Qual é o mais rápido? Qual é o mais preciso?
        \end{itemize}
        
        \item \textbf{Experimentos com Taxa de Aprendizado:}
        \begin{itemize}
            \item $\alpha = 0.00001$ (muito pequeno)
            \item $\alpha = 0.001$ (adequado)
            \item $\alpha = 0.1$ (muito grande)
        \end{itemize}
        
        \item \textbf{Visualização da Trajetória:}
        \begin{itemize}
            \item Plotar o caminho no espaço de parâmetros
            \item Plotar a evolução da função de custo
            \item Identificar ponto de convergência
        \end{itemize}
        
        \item \textbf{SGD com Embaralhamento:}
        \begin{itemize}
            \item Implementar versão estocástica
            \item Comparar com e sem embaralhamento
        \end{itemize}
    \end{enumerate}
\end{frame}



\begin{frame}[c]{Critérios de Convergência}
    
    \textbf{Como saber quando parar o algoritmo?}
    
    \begin{enumerate}
        \item \textbf{Convergência do gradiente:}
        \begin{equation}
            \|\nabla J(\boldsymbol{\beta}^{(t)})\| < \varepsilon_1
        \end{equation}
        
        \item \textbf{Convergência dos parâmetros:}
        \begin{equation}
            \|\boldsymbol{\beta}^{(t+1)} - \boldsymbol{\beta}^{(t)}\| < \varepsilon_2
        \end{equation}
        
        \item \textbf{Convergência da função de custo:}
        \begin{equation}
            |J(\boldsymbol{\beta}^{(t+1)}) - J(\boldsymbol{\beta}^{(t)})| < \varepsilon_3
        \end{equation}
        
        \item \textbf{Número máximo de iterações:} $t > T_{\max}$
    \end{enumerate}

\end{frame}

\begin{frame}[c]{Visualização das Trajetórias}
    
    \textbf{Análise visual é fundamental para entender convergência:}
        \vspace{0.5cm}

    \begin{columns}[c]
        \begin{column}{0.5\textwidth}
            \textbf{Plots importantes:}
            \begin{itemize}
                \item Trajetória no espaço de parâmetros
                \item Evolução da função de custo
                \item Curvas de nível + trajetória
                \item Comparação SGD vs. Batch GD
            \end{itemize}
        \end{column}
        \begin{column}{0.5\textwidth}
            \textbf{O que observar:}
            \begin{itemize}
                \item Convergência suave vs. oscilante
                \item Velocidade de convergência
                \item Ponto final vs. solução analítica
                \item Efeito da taxa de aprendizado
            \end{itemize}
        \end{column}
    \end{columns}
    

\end{frame}

\section{Conceitos Avançados}

\begin{frame}[c]{Momentum e Algoritmos Avançados}
    
    \textbf{Limitações do GD básico:}
    \begin{itemize}
        \item Pode ficar \highlight{oscilando} em vales estreitos
        \item Convergência \highlight{lenta} em direções de pouca curvatura
        \item Dificuldade com \highlight{mínimos locais}
    \end{itemize}
    
    \vspace{0.5cm}
    
    \begin{defi}{SGD com Momentum}{momentum}
        \begin{align}
            \mathbf{v}^{(t+1)} &= \gamma \mathbf{v}^{(t)} + \alpha \nabla J(\boldsymbol{\beta}^{(t)}) \\
            \boldsymbol{\beta}^{(t+1)} &= \boldsymbol{\beta}^{(t)} - \mathbf{v}^{(t+1)}
        \end{align}
        onde $\gamma \in [0,1)$ é o coeficiente de momentum.
    \end{defi}
    
    \textbf{Algoritmos mais modernos:} Adam, RMSprop, AdaGrad (veremos nas próximas aulas)
\end{frame}

\begin{frame}[c]{Resumo da Aula}
    
    \begin{display}[O que aprendemos hoje]
        \begin{itemize}
            \item Revisão de conceitos fundamentais de ML
            \item Regressão linear como base para redes neurais
            \item \highlight{Descida de gradiente}: algoritmo e implementação
            \item Importância da \highlight{taxa de aprendizado}
            \item \highlight{SGD vs. Batch GD}
            \item Efeito do \highlight{embaralhamento} em SGD
            \item Critérios de convergência e visualização
            \item Introdução ao \highlight{momentum}
        \end{itemize}
    \end{display}

\end{frame}   
\begin{frame}[c]{Resumo da Aula}
    \begin{display}[Próxima semana: Introdução às Redes Neurais]
        \begin{itemize}
            \item Do neurônio biológico ao artificial
            \item \highlight{Perceptron} e funções de ativação
            \item Primeira rede neural \highlight{multi-camadas}
            \item Por que precisamos de \highlight{não-linearidade}?
        \end{itemize}
    \end{display}
\end{frame}

\begin{frame}[c, noframenumbering, plain]
    \frametitle{~}
        \vfill
        \begin{center}
            {\Huge Obrigado!}\vspace{1.5em}\\
            {\Large \highlight{Alguma dúvida?}}\\
        \end{center}
        \vfill
        \begin{center}
            {\small Agora vamos para os exercícios!}
        \end{center}
\end{frame}

\end{document}