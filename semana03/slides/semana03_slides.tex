\documentclass[xcolor=dvipsnames,t,aspectratio=169]{beamer}

\usecolortheme{rose}
\usecolortheme{dolphin}
\usetheme{Boadilla}

% Encoding packages MUST come first
\usepackage[utf8]{inputenc}
\usepackage[brazil]{babel}
\usepackage[T1]{fontenc}

\usepackage{amsfonts}
\usepackage{amsmath}
\usepackage{amssymb}
% \usepackage{latexsym}
% \usepackage{graphicx}
% \usepackage{listings} 
\usepackage{xcolor}
\usepackage{amsthm}
\usepackage{url}
\usepackage{textpos}
% \usepackage{amssymb}
% \usepackage{caption}
% \usepackage{subcaption}
% \usepackage{multicol}
% \usepackage{mathrsfs}
\usepackage{float}
\setbeamersize{text margin left=1cm,text margin right=1cm,} 
\usepackage{subcaption}
\usepackage{dsfont}
\usepackage{physics}
\usepackage{mathpazo}
% \usepackage{tikz}
\usepackage{lipsum}
\usepackage{minted}
\usepackage[minted, most]{tcolorbox}
\tcbuselibrary{theorems}
\setbeamertemplate{enumerate items}[default]
\usefonttheme{professionalfonts}

\setlength{\parindent}{0pt} % zero indent

\setbeamertemplate{itemize items}[circle]
\setbeamertemplate{itemize subitem}{$\blacktriangleright$}
\definecolor{nes_dark_purple}{HTML}{4E006B}
\definecolor{nes_dark_orange}{HTML}{EE4F09}
\setbeamercolor{normal text}{fg=nes_dark_purple}\usebeamercolor*{normal text}
\setbeamercolor{math text}{fg=nes_dark_purple}\usebeamercolor[fg]{math text}


\setbeamercolor{title}{fg=nes_dark_purple}
\setbeamercolor{frametitle}{fg=nes_dark_purple}
\setbeamercolor{structure}{fg=nes_dark_purple}
\setbeamercolor{author}{fg=nes_dark_orange}
\setbeamercolor{footline}{fg=nes_dark_purple} 

\setbeamertemplate{footline}[frame number]
\setbeamertemplate{navigation symbols}{}

\definecolor{codegreen}{rgb}{0,0.6,0.2}
\definecolor{codegray}{rgb}{0.5,0.5,0.5}
\definecolor{codepurple}{rgb}{0.58,0,0.82}
\definecolor{backcolour}{rgb}{0.95,0.95,0.92}

\lstdefinestyle{mystyle}{
    backgroundcolor=\color{backcolour},   
    commentstyle=\color{codegreen},
    keywordstyle=\color{magenta},
    numberstyle=\tiny\color{codegray},
    stringstyle=\color{codepurple},
    basicstyle=\ttfamily\footnotesize,
    breakatwhitespace=false,  
    breaklines=true,                 
    captionpos=b,                    
    keepspaces=true,                 
    numbers=left,                    
    numbersep=5pt,                  
    showspaces=false,                
    showstringspaces=false,
    showtabs=false,                  
    tabsize=2
}
\input{../../templates/slides/commands}

\usepackage{tikz}
\usetikzlibrary{arrows.meta, shapes}
\usepackage{algorithmic}

\titlegraphic{
    \includegraphics[scale = 0.5]{../../templates/slides/logo}
}

\logo{
\begin{tikzpicture}[overlay,remember picture]
\node[left=1.1cm, below=0.2cm] at (current page.30){
    \includegraphics[width=0.1\textwidth]{../../templates/slides/logo}};
\end{tikzpicture}
}

\newcommand{\highlight}[1]{{\color{nes_dark_orange} #1}}

\title{Retropropagação em Redes Neurais} 

\author{
    Eduardo Adame
}

\date{{\color{nes_dark_purple}  \textbf{Redes Neurais}\\[0.5em] 20 de agosto de 2025 }}

\begin{document}

\frame[plain]{\titlepage}
\setcounter{framenumber}{0}

\section{Introdução ao Backpropagation}
\begin{frame}[c]{Como Treinar uma Rede Neural?}
    \begin{columns}[c]
        \begin{column}{0.5\textwidth}
            \textbf{Processo de Treinamento:}
            \begin{itemize}
                \item Inserir dados de treino
                \item Obter a saída da rede
                \item Comparar com respostas corretas
                \item Calcular função de perda $J$
                \item \highlight{Ajustar pesos e repetir!}
            \end{itemize}
            
            \vspace{0.5cm}
            \textbf{Backpropagation:}\\
            Usa cálculo para determinar como ajustar cada peso
        \end{column}
        \begin{column}{0.5\textwidth}
            \begin{tikzpicture}[scale=0.8]
                % Rede simples
                \node[circle, draw, fill=cyan!30] (x1) at (0, 1.5) {$x_1$};
                \node[circle, draw, fill=cyan!30] (x2) at (0, 0) {$x_2$};
                \node[circle, draw, fill=cyan!30] (x3) at (0, -1.5) {$x_3$};
                
                \foreach \i in {1,2,3} {
                    \node[circle, draw, fill=orange!40] (h\i) at (3, 2-\i) {$\sigma$};
                }
                
                \node[circle, draw, fill=gray!50] (y) at (6, 0) {$\hat{y}$};
                
                % Conexões
                \foreach \x in {1,2,3} {
                    \foreach \h in {1,2,3} {
                        \draw[-latex, yellow!70!black] (x\x) -- (h\h);
                    }
                }
                
                \foreach \h in {1,2,3} {
                    \draw[-latex, yellow!70!black] (h\h) -- (y);
                }
                
                % Labels
                \node[above] at (1.5, 2) {$W^{(1)}$};
                \node[above] at (4.5, 1.5) {$W^{(2)}$};
            \end{tikzpicture}
        \end{column}
    \end{columns}
\end{frame}

\begin{frame}[c]{Descida do Gradiente em Redes Neurais}
    \begin{display}[Algoritmo de Treinamento]
        \begin{enumerate}
            \item \textbf{Fazer predição} (forward pass)
            \item \textbf{Calcular perda} $J(\mathbf{y}, \hat{\mathbf{y}})$
            \item \highlight{Calcular gradiente} da função de perda em relação aos parâmetros
            \item \textbf{Atualizar parâmetros} dando um passo na direção oposta
            \item \textbf{Iterar} até convergência
        \end{enumerate}
    \end{display}
    
    \begin{attention}[Questão Central]
        Como calcular $\dfrac{\partial J}{\partial W_k}$ para cada peso $W_k$ na rede?
    \end{attention}
    
    \begin{center}
        \textbf{Resposta:} \highlight{Regra da Cadeia!}
    \end{center}
\end{frame}

\begin{frame}[c]{Forward Propagation}
    \begin{columns}[c]
        \begin{column}{0.4\textwidth}
            \textbf{Etapas do Forward Pass:}
            \begin{enumerate}
                \item Inserir entrada $\mathbf{x}$
                \item Calcular cada camada sequencialmente
                \item Obter saída $\hat{\mathbf{y}}$
                \item Avaliar perda $J(\mathbf{y}, \hat{\mathbf{y}})$
            \end{enumerate}
        \end{column}
        \begin{column}{0.6\textwidth}
            \begin{tikzpicture}[scale=0.9]
                % Camadas
                \node[circle, draw, fill=cyan!30] (x1) at (0, 1) {$x_1$};
                \node[circle, draw, fill=cyan!30] (x2) at (0, -1) {$x_2$};
                
                \foreach \i in {1,2} {
                    \node[circle, draw, fill=orange!40] (h1\i) at (2.5, 1.5-\i*1.5) {};
                }
                
                \foreach \i in {1,2} {
                    \node[circle, draw, fill=orange!40] (h2\i) at (5, 1.5-\i*1.5) {};
                }
                
                \node[circle, draw, fill=gray!50] (y) at (7.5, 0) {$\hat{y}$};
                
                % Setas indicando direção
                \draw[->, ultra thick, blue] (0, -2.5) -- (7.5, -2.5) node[midway, below] {\textbf{Direção do Forward Pass}};
                
                % Conexões
                \foreach \x in {1,2} {
                    \foreach \h in {1,2} {
                        \draw[-latex, yellow!60!black] (x\x) -- (h1\h);
                    }
                }
                
                \foreach \h in {1,2} {
                    \foreach \hh in {1,2} {
                        \draw[-latex, yellow!60!black] (h1\h) -- (h2\hh);
                    }
                }
                
                \foreach \h in {1,2} {
                    \draw[-latex, yellow!60!black] (h2\h) -- (y);
                }
                
                % Anotações
                \node at (2.5, 2.5) {$z^{(2)} = XW^{(1)}$};
                \node at (2.5, -2) {$a^{(2)} = \sigma(z^{(2)})$};
            \end{tikzpicture}
        \end{column}
    \end{columns}
\end{frame}

\begin{frame}[c]{Backpropagation: A Intuição}
    \begin{columns}[c]
        \begin{column}{0.5\textwidth}
            \textbf{O que queremos calcular:}
            $$\frac{\partial J}{\partial W_k} \text{ para todo peso } W_k$$
            
            \vspace{0.5cm}
            \textbf{Estratégia:}
            \begin{itemize}
                \item Começar do \highlight{erro na saída}
                \item Propagar o erro \highlight{de trás para frente}
                \item Usar a \highlight{regra da cadeia} repetidamente
                \item Calcular gradientes camada por camada
            \end{itemize}
        \end{column}
        \begin{column}{0.5\textwidth}
            \begin{tikzpicture}[scale=0.8]
                % Rede
                \node[circle, draw, fill=cyan!30] (x) at (0, 0) {$x$};
                \node[circle, draw, fill=orange!40] (h1) at (2, 0) {$h_1$};
                \node[circle, draw, fill=orange!40] (h2) at (4, 0) {$h_2$};
                \node[circle, draw, fill=gray!50] (y) at (6, 0) {$\hat{y}$};
                
                \draw[-latex] (x) -- (h1) node[midway, above] {$W^{(1)}$};
                \draw[-latex] (h1) -- (h2) node[midway, above] {$W^{(2)}$};
                \draw[-latex] (h2) -- (y) node[midway, above] {$W^{(3)}$};
                
                % Backprop arrows
                \draw[->, ultra thick, red, dashed] (6, -1.5) -- (0, -1.5) node[midway, below] {\textbf{Direção do Backprop}};
                
                % Gradientes
                \node[red] at (6, 1) {$\frac{\partial J}{\partial \hat{y}}$};
                \node[red] at (4, 1) {$\frac{\partial J}{\partial h_2}$};
                \node[red] at (2, 1) {$\frac{\partial J}{\partial h_1}$};
            \end{tikzpicture}
        \end{column}
    \end{columns}
\end{frame}

\begin{frame}[c]{Cálculo dos Gradientes}
    \begin{display}[Fórmulas do Backpropagation]
        Para uma rede de 3 camadas com ativação sigmoid:
        
        \vspace{0.3cm}
        \begin{align}
            \frac{\partial J}{\partial W^{(3)}} &= (\hat{\mathbf{y}} - \mathbf{y}) \cdot a^{(3)} \\[0.5em]
            \frac{\partial J}{\partial W^{(2)}} &= (\hat{\mathbf{y}} - \mathbf{y}) \cdot W^{(3)} \cdot \sigma'(z^{(3)}) \cdot a^{(2)} \\[0.5em]
            \frac{\partial J}{\partial W^{(1)}} &= (\hat{\mathbf{y}} - \mathbf{y}) \cdot W^{(3)} \cdot \sigma'(z^{(3)}) \cdot W^{(2)} \cdot \sigma'(z^{(2)}) \cdot \mathbf{x}
        \end{align}
    \end{display}
    
    % \begin{attention}[Lembre-se]
    %     Para a função sigmoid: $\sigma'(z) = \sigma(z) \cdot (1 - \sigma(z))$
    % \end{attention}
\end{frame}

\begin{frame}[c]{Problema do Gradiente Desvanecente}
    \begin{columns}[c]
        \begin{column}{0.5\textwidth}
            \textbf{O Problema:}
            \begin{itemize}
                \item $\sigma'(z) = \sigma(z)(1-\sigma(z)) \leq 0.25$
                \item Gradientes ficam \highlight{muito pequenos} em camadas iniciais
                \item Multiplicação de muitos valores $< 1$
                \item Aprendizado \highlight{extremamente lento} ou \highlight{para}
            \end{itemize}
            
            \vspace{0.5cm}
            \textbf{Solução:}
            \begin{itemize}
                \item Usar outras funções de ativação
                \item ReLU: $f(z) = \max(0, z)$
                \item Leaky ReLU, tanh, etc.
            \end{itemize}
        \end{column}
        \begin{column}{0.5\textwidth}
            \begin{tikzpicture}[scale=0.9]
                % Gráfico sigmoid
                \draw[->] (-2, 0) -- (2, 0) node[right] {$z$};
                \draw[->] (0, -0.5) -- (0, 2) node[above] {$\sigma'(z)$};
                
                % Curva da derivada
                \draw[blue, ultra thick, domain=-2:2, smooth, variable=\x]
                    plot (\x, {1.5*exp(-\x*\x)});
                
                % Valor máximo
                \draw[dashed, red] (-2, 0.375) -- (2, 0.375);
                \node[red] at (2.5, 0.375) {$0.25$};
                
                % Anotação
                \node[text width=3cm, align=center] at (0, -1.5) {
                    \small Derivada sempre $\leq 0.25$
                };
            \end{tikzpicture}
        \end{column}
    \end{columns}
\end{frame}

\begin{frame}[c]{Funções de Ativação Alternativas}
    \begin{columns}[c]
        \begin{column}{0.33\textwidth}
            \centering
            \textbf{Sigmoid}
            \begin{tikzpicture}[scale=0.6]
                \draw[->] (-2, 0) -- (2, 0);
                \draw[->] (0, -0.5) -- (0, 1.5);
                \draw[blue, ultra thick, domain=-2:2, smooth, variable=\x]
                    plot (\x, {1/(1 + exp(-2*\x))});
            \end{tikzpicture}
            
            $$\sigma(z) = \frac{1}{1+e^{-z}}$$
            \textcolor{red}{Gradiente desvanece}
        \end{column}
        \begin{column}{0.33\textwidth}
            \centering
            \textbf{ReLU}
            \begin{tikzpicture}[scale=0.6]
                \draw[->] (-2, 0) -- (2, 0);
                \draw[->] (0, -0.5) -- (0, 1.5);
                \draw[blue, ultra thick] (-2, 0) -- (0, 0);
                \draw[blue, ultra thick] (0, 0) -- (2, 1.5);
            \end{tikzpicture}
            
            $$f(z) = \max(0, z)$$
            \textcolor{green}{Gradiente preservado}
        \end{column}
        \begin{column}{0.33\textwidth}
            \centering
            \textbf{Tanh}
            \begin{tikzpicture}[scale=0.6]
                \draw[->] (-2, 0) -- (2, 0);
                \draw[->] (0, -1.5) -- (0, 1.5);
                \draw[blue, ultra thick, domain=-2:2, smooth, variable=\x]
                    plot (\x, {tanh(\x)});
            \end{tikzpicture}
            
            $$\tanh(z) = \frac{e^{2z}-1}{e^{2z}+1}$$
            \textcolor{orange}{Centrado em zero}
        \end{column}
    \end{columns}
\end{frame}

\begin{frame}[c]{Como Escolher a Função de Ativação}
    \begin{columns}[c]
        \begin{column}{0.5\textwidth}
            \textbf{Recomendações Práticas:}
            \begin{itemize}
                \item \highlight{Camadas ocultas:} ReLU (default), Leaky ReLU, ELU ou SiLU
                \item \highlight{Camada de saída:}
                \begin{itemize}
                    \item Regressão: Linear ou nenhuma ativação
                    \item Classificação binária: Sigmoid
                    \item Classificação multiclasse: Softmax
                    \item Valores positivos: ReLU
                \end{itemize}
                \item Experimente diferentes funções para seu problema específico
                \item Monitore a proporção de neurônios mortos (para ReLU)
            \end{itemize}
        \end{column}
        \begin{column}{0.5\textwidth}
        \textbf{Regra de Ouro:}
            \begin{center}
                \highlight{Não existe uma função ideal para todos os casos!}
            \end{center}
        \end{column}
    \end{columns}
\end{frame}

\begin{frame}[c]{Algoritmo Completo de Backpropagation}
    \begin{display}[Pseudocódigo]
        \begin{algorithmic}
            \STATE \textbf{Para} cada época de treinamento:
            \STATE \quad \textbf{Para} cada batch de dados:
            \STATE \quad \quad 1. \textbf{Forward Pass:}
            \STATE \quad \quad \quad Calcular $a^{(l)} = \sigma(z^{(l)})$ para cada camada $l$
            \STATE \quad \quad 2. \textbf{Calcular Perda:}
            \STATE \quad \quad \quad $J = \frac{1}{m}\sum_{i=1}^{m} L(y_i, \hat{y}_i)$
            \STATE \quad \quad 3. \textbf{Backward Pass:}
            \STATE \quad \quad \quad Calcular $\delta^{(L)} = \nabla_a J \odot \sigma'(z^{(L)})$
            \STATE \quad \quad \quad \textbf{Para} $l = L-1, L-2, ..., 1$:
            \STATE \quad \quad \quad \quad $\delta^{(l)} = (W^{(l+1)})^T \delta^{(l+1)} \odot \sigma'(z^{(l)})$
            \STATE \quad \quad 4. \textbf{Atualizar Pesos:}
            \STATE \quad \quad \quad $W^{(l)} = W^{(l)} - \alpha \cdot \delta^{(l)} \cdot (a^{(l-1)})^T$
        \end{algorithmic}
    \end{display}
\end{frame}


\begin{frame}[c]{Exemplo Prático: Cálculo de um Gradiente}
    \begin{columns}[c]
        \begin{column}{0.5\textwidth}
            \textbf{Cenário Simplificado:}
            \begin{itemize}
                \item Rede com apenas 1 neurônio
                \item Entrada $x$, peso $w$, bias $b$
                \item $\hat{y} = \sigma(z)$ onde $z = w \cdot x + b$
                \item Função de perda: $J = \frac{1}{2}(y - \hat{y})^2$
            \end{itemize}
            
            \vspace{0.5cm}
            \textbf{Cálculo do Gradiente:}
            \footnotesize
            \begin{align*}
                \frac{\partial J}{\partial w} &= \frac{\partial J}{\partial \hat{y}} \cdot \frac{\partial \hat{y}}{\partial z} \cdot \frac{\partial z}{\partial w} \\
                &= - (y - \hat{y}) \cdot \sigma'(z) \cdot x
            \end{align*}
        \end{column}
        \begin{column}{0.5\textwidth}
            \begin{tikzpicture}[scale=0.9]
                % Neurônio único
                \node[circle, draw, fill=cyan!30] (x) at (0, 0) {$x$};
                \node[circle, draw, fill=orange!40, minimum size=1.2cm] (neuron) at (2.5, 0) {$\sigma$};
                \node[circle, draw, fill=gray!50] (y) at (5, 0) {$\hat{y}$};
                
                % Parâmetros
                \draw[-latex] (x) -- (neuron) node[midway, above] {$w$};
                \draw[-latex] (neuron) -- (y);
                
                % Bias
                \node[circle, fill=cyan!10] (b) at (2.5, -1.5) {$b$};
                \draw[-latex] (b) -- (neuron);
                
                % Fórmulas
                \node[text width=3cm] at (2.5, -2.5) {
                    \footnotesize
                    $z = w \cdot x + b$ \\
                    $\hat{y} = \sigma(z)$
                };
            \end{tikzpicture}
        \end{column}
    \end{columns}
    
    \vspace{0.5cm}
    \begin{center}
        \textbf{Esta mesma lógica se aplica a redes complexas através da regra da cadeia!}
    \end{center}
\end{frame}


\begin{frame}[c]{Resumo e Próximos Passos}
    \begin{columns}[c]
        \begin{column}{0.5\textwidth}
            \textbf{O que aprendemos:}
            \begin{itemize}
                \item Forward propagation
                \item \highlight{Backpropagation}
                \item Cálculo de gradientes
                \item Regra da cadeia
                \item Problema do gradiente desvanecente
                \item Funções de ativação alternativas
            \end{itemize}
        \end{column}
        \begin{column}{0.5\textwidth}
            \textbf{Próxima aula:}
            \begin{itemize}
                \item Otimizadores
                \item Regularização (Dropout, L2)
                \item Batch Normalization
                \item Inicialização de pesos
                \item Implementação prática
            \end{itemize}
        \end{column}
    \end{columns}
    
    \vspace{0.5cm}
    
    \begin{center}
        \Large
        \highlight{Backpropagation é o coração do aprendizado profundo!}
    \end{center}
\end{frame}

\begin{frame}[c, noframenumbering, plain]
    \frametitle{~}
    \vfill
    \begin{center}
        {\Huge Obrigado!}\vspace{1.5em}\\
        {\Large \highlight{Alguma dúvida?}}\\
    \end{center}
    \vfill
    \begin{center}
        {\small Agora vamos implementar backpropagation!}
    \end{center}
\end{frame}

\end{document}